\documentclass{article}
\usepackage{graphicx} % Required for inserting images
\usepackage[left=2cm, right=2cm, top=0.5cm, bottom=0.5cm]{geometry}
\title{Verifica Elettrostatica}
\author{4B LSA}
\date{22 Gennaio 2025}
\pagestyle{empty}
\begin{document}
\pagestyle{empty}
\renewcommand{\tablename}{}
\maketitle
\centering
\textbf{Griglia di valutazione}. Eventuali mezzi punti saranno arrotondati all'intero precedente.
\begin{table}[h]
    \centering
\begin{tabular}{|c|c|c|c|c|c|c|c|c|c|c|c|c|c|c|c|c|c|c|c|}
\hline
Punti &  $\leq 4$ & 5 & 6 & 7 & 8 & 9 & 10 & 11 & \textbf{12} & 13 & 14 & 15 & 16 & 17 & 18 & 19 & 20 \\
\hline
Voto & 2 & 2.5 & 3 & 3.5 & 4 & 4.5 & 5 & 5.5 & \textbf{6} & 6.5 & 7 & 7.5 & 8 & 8.5 & 9 & 9.5 & 10 \\
\hline
\end{tabular}
\end{table}


\textbf{Codice}: srjvbtgcwlnfoxbgirffp


\begin{enumerate}
  \item Un corpo di massa $m_1$ e un corpo di massa $m_2$ sono lanciati in contemporanea da una torre alta $h$. Quali sono le velocità dei due corpi appena prima di toccare terra?
  \begin{enumerate}[label=\Alph*.]
    \item $v_1=\sqrt{\frac{2gh}{m_1}}, v_2=\sqrt{\frac{2gh}{m_2}}$.
    \item $v_1=\sqrt{2m_1gh}, v_2=\sqrt{2m_2gh}$.
    \item v_1=0, v_2=0
    \item $v_1=\sqrt{2gh}, v_2=\sqrt{2gh}$.
  \end{enumerate}
  \item Un corpo, partendo da fermo, rotola giù dalla cima di un piano inclinato di un angolo $\alpha$, alto $h$. Quale è la sua velocità quando arriva in fondo?
  \begin{enumerate}[label=\Alph*.]
    \item $\sqrt{2gh}$.
    \item $\sqrt{mgh\cos\alpha}$.
    \item $\sqrt{2gh}\sin\alpha$.
    \item $\sqrt{2gh\sin\alpha}$.
  \end{enumerate}
  \item Sia \vec{F} una forza conservativa. Allora
  \begin{enumerate}[label=\Alph*.]
    \item il lavoro di \vec{F} è sempre nullo.
    \item il lavoro di \vec{F} può essere nullo anche se il percorso non è chiuso.
    \item il lavoro di \vec{F} non dipende dal percorso considerato.
    \item il lavoro di \vec{F} è nullo solo se il percorso è chiuso.
  \end{enumerate}
  \item Un corpo viene lasciato cadere da fermo da un altezza $h$. Quando raggiunge il suolo ha una velocità $v$. Quale relazione esprime correttamente la conservazione dell'energia?
  \begin{enumerate}[label=\Alph*.]
    \item $mgh=\frac{1}{2}mv^2$.
    \item $\frac{1}{2}mh^2=\frac{1}{2}mv^2$.
    \item $\frac{1}{2}mv^2+mgh=0.$
    \item $mgh+\frac{1}{2}mv^2=\frac{1}{2}mv^2.$
  \end{enumerate}
  \item L'energia cinetica di un corpo fermo che si trova a un'altezza $h$
  \begin{enumerate}[label=\Alph*.]
    \item dipende da $h$
    \item è nulla.
    \item è $mgh$.
    \item si conserva.
  \end{enumerate}
  \item Un corpo impatta una molla con una velocità $v$. La molla di costante elastica $k$, opponendosi all'avanzare del corpo, lo ferma. Quale è il suo allungamento nel momento in cui si ferma, trascurando l'attrito?
  \begin{enumerate}[label=\Alph*.]
    \item $\sqrt{\frac{m}{k}}v$.
    \item $\sqrt{\frac{m}{k}v}$.
    \item $\sqrt{\frac{k}{m}}v$.
    \item $\sqrt{\frac{k}{m}}v$.
  \end{enumerate}
  \item Un corpo di massa $m$ scivola su un piano con attrito, partendo da una velocità v. Il coefficiente di attrito è \mu. Dopo quanto spazio si ferma?
  \begin{enumerate}[label=\Alph*.]
    \item $\frac{2v^2}{g\mu}}$.
    \item $\frac{1}{2}v^2+\mu g$.
    \item $\frac{v^2}{2g\mu}}$.
    \item $\frac{1}{2}v^2-\mu g$.
  \end{enumerate}
  \item Sia $E_m$ l'energia meccanica, $K$ l'energia cinetica, $U$ quella potenziale e $W_{NC}$ il lavoro delle forze non conservative agenti. Quale delle seguenti relazioni è vera?
  \begin{enumerate}[label=\Alph*.]
    \item $\Delta U=W_{NC}$.
    \item $\Delta K=W_{NC}.$
    \item $\Delta K = \Delta U$.
    \item $\Delta E_m=W_{NC}$.
  \end{enumerate}
  \item Sia \vec{F} una forza generica, che agisce su un corpo che effettua uno spostamento $\Delta \vec{s}$. Allora
  \begin{enumerate}[label=\Alph*.]
    \item il suo lavoro non ha una direzione perché è scalare.
    \item il suo lavoro ha la stessa direzione dello spostamento.
    \item il suo lavoro ha la stessa direzione della forza.
    \item il suo lavoro ha una direzione data dalla regola del parallelogramma.
  \end{enumerate}
  \item Il lavoro della forza peso
  \begin{enumerate}[label=\Alph*.]
    \item è sempre positivo.
    \item è sempre negativo.
    \item è nullo solo se il percorso è chiuso.
    \item è nullo se il percorso ha gli estremi posti alla stessa altitudine.
  \end{enumerate}
  \item Se calcoliamo il lavoro di una forza lungo un percorso chiuso e troviamo zero, cosa possiamo concludere?
  \begin{enumerate}[label=\Alph*.]
    \item niente.
    \item la forza non è conservativa.
    \item la forza è conservativa.
    \item la forza è nulla.
  \end{enumerate}
  \item Il lavoro della reazione vincolare
  \begin{enumerate}[label=\Alph*.]
    \item può assumere qualsiasi valore.
    \item è nullo solo se il percorso è chiuso.
    \item è sempre nullo.
    \item è sempre diverso da zero.
  \end{enumerate}
  \item Sia $E_m$ l'energia meccanica, $K$ l'energia cinetica, $U$ quella potenziale e $W$ il lavoro di tutte le forze agenti. Quale delle seguenti relazioni è vera?
  \begin{enumerate}[label=\Alph*.]
    \item $\Delta K=W.$
    \item $\Delta E_m=W$.
    \item $\Delta U=W$.
    \item $\Delta U=W$.
  \end{enumerate}
  \item Un corpo sale lungo un piano inclinato di un angolo $\alpha$ e alto $h$, partendo da terra e arrivando fino alla cima del piano. Il lavoro della forza peso è:
  \begin{enumerate}[label=\Alph*.]
    \item $mgh\sin\alpha$
    \item $-mgh$
    \item $mgh\cos\alpha$
    \item $mgh$
  \end{enumerate}
  \item Il lavoro della forza d'attrito
  \begin{enumerate}[label=\Alph*.]
    \item è nullo se il percorso è rettilineo.
    \item non può essere calcolato perché la forza non è conservativa.
    \item è sempre negativo.
    \item è sempre positivo.
  \end{enumerate}
  \item Un corpo sale da terra fino a un'altezza $h$. Il lavoro della forza peso è:
  \begin{enumerate}[label=\Alph*.]
    \item positivo.
    \item negativo.
    \item nullo
    \item $mgh$
  \end{enumerate}
  \item Un corpo sale da terra fino a un'altezza $h$. Il lavoro della forza peso è:
  \begin{enumerate}[label=\Alph*.]
    \item negativo.
    \item positivo.
    \item nullo
    \item $mgh$
  \end{enumerate}
  \item L'energia meccanica di un corpo che si trova a un'altezza $h$
  \begin{enumerate}[label=\Alph*.]
    \item dipende dalla velocità del corpo.
    \item è nulla.
    \item è $mgh$.
    \item è negativa
  \end{enumerate}
  \item Sia \vec{F} una forza non conservativa. Allora
  \begin{enumerate}[label=\Alph*.]
    \item la sua energia potenziale è sempre negativa.
    \item la sua energia potenziale non esiste.
    \item la sua energia potenziale è tale che lungo un percorso $\Delta U=-W$.
    \item la sua energia potenziale è tale che lungo un percorso $\Delta U=W$.
  \end{enumerate}
  \item Un corpo ha inizialmente una velocità $v$ e dopo un certo tempo si ferma. La variazione di energia cinetica è
  \begin{enumerate}[label=\Alph*.]
    \item è negativa
    \item dipende dal tempo in cui si ferma.
    \item è positiva.
    \item dipende dallo spazio percorso.
  \end{enumerate}
  \item Sia \vec{F} una forza non conservativa. Allora
  \begin{enumerate}[label=\Alph*.]
    \item il lavoro non dipende dagli estremi del percorso.
    \item esiste un percorso chiuso il cui lavoro non è nullo.
    \item il lavoro non dipende dal percorso.
    \item per ogni percorso aperto il lavoro è nullo.
  \end{enumerate}
\end{enumerate}








\newpage \maketitle \centering \textbf{Griglia di valutazione}. Eventuali mezzi punti saranno arrotondati all'intero precedente. \begin{table}[h]     \centering \begin{tabular}{|c|c|c|c|c|c|c|c|c|c|c|c|c|c|c|c|c|c|c|c|} \hline Punti &  $\leq 4$ & 5 & 6 & 7 & 8 & 9 & 10 & 11 & \textbf{12} & 13 & 14 & 15 & 16 & 17 & 18 & 19 & 20 \\ \hline Voto & 2 & 2.5 & 3 & 3.5 & 4 & 4.5 & 5 & 5.5 & \textbf{6} & 6.5 & 7 & 7.5 & 8 & 8.5 & 9 & 9.5 & 10 \\ \hline \end{tabular} \end{table} 
\textbf{Codice}: rrkwdueaxloeoxchltefo


\begin{enumerate}
  \item Il lavoro della forza peso
  \begin{enumerate}[label=\Alph*.]
    \item è nullo solo se il percorso è chiuso.
    \item è sempre positivo.
    \item è nullo se il percorso ha gli estremi posti alla stessa altitudine.
    \item è sempre negativo.
  \end{enumerate}
  \item L'energia meccanica di un corpo che si trova a un'altezza $h$
  \begin{enumerate}[label=\Alph*.]
    \item dipende dalla velocità del corpo.
    \item è $mgh$.
    \item è nulla.
    \item è negativa
  \end{enumerate}
  \item Un corpo viene lasciato cadere da fermo da un altezza $h$. Quando raggiunge il suolo ha una velocità $v$. Quale relazione esprime correttamente la conservazione dell'energia?
  \begin{enumerate}[label=\Alph*.]
    \item $\frac{1}{2}mv^2+mgh=0.$
    \item $\frac{1}{2}mh^2=\frac{1}{2}mv^2$.
    \item $mgh=\frac{1}{2}mv^2$.
    \item $mgh+\frac{1}{2}mv^2=\frac{1}{2}mv^2.$
  \end{enumerate}
  \item L'energia cinetica di un corpo fermo che si trova a un'altezza $h$
  \begin{enumerate}[label=\Alph*.]
    \item è $mgh$.
    \item è nulla.
    \item si conserva.
    \item dipende da $h$
  \end{enumerate}
  \item Il lavoro della reazione vincolare
  \begin{enumerate}[label=\Alph*.]
    \item è sempre diverso da zero.
    \item è nullo solo se il percorso è chiuso.
    \item può assumere qualsiasi valore.
    \item è sempre nullo.
  \end{enumerate}
  \item Un corpo, partendo da fermo, rotola giù dalla cima di un piano inclinato di un angolo $\alpha$, alto $h$. Quale è la sua velocità quando arriva in fondo?
  \begin{enumerate}[label=\Alph*.]
    \item $\sqrt{mgh\cos\alpha}$.
    \item $\sqrt{2gh}$.
    \item $\sqrt{2gh\sin\alpha}$.
    \item $\sqrt{2gh}\sin\alpha$.
  \end{enumerate}
  \item Un corpo ha inizialmente una velocità $v$ e dopo un certo tempo si ferma. La variazione di energia cinetica è
  \begin{enumerate}[label=\Alph*.]
    \item è negativa
    \item dipende dallo spazio percorso.
    \item è positiva.
    \item dipende dal tempo in cui si ferma.
  \end{enumerate}
  \item Un corpo sale da terra fino a un'altezza $h$. Il lavoro della forza peso è:
  \begin{enumerate}[label=\Alph*.]
    \item nullo
    \item negativo.
    \item positivo.
    \item $mgh$
  \end{enumerate}
  \item Un corpo di massa $m_1$ e un corpo di massa $m_2$ sono lanciati in contemporanea da una torre alta $h$. Quali sono le velocità dei due corpi appena prima di toccare terra?
  \begin{enumerate}[label=\Alph*.]
    \item $v_1=\sqrt{2m_1gh}, v_2=\sqrt{2m_2gh}$.
    \item $v_1=\sqrt{2gh}, v_2=\sqrt{2gh}$.
    \item $v_1=\sqrt{\frac{2gh}{m_1}}, v_2=\sqrt{\frac{2gh}{m_2}}$.
    \item v_1=0, v_2=0
  \end{enumerate}
  \item Sia $E_m$ l'energia meccanica, $K$ l'energia cinetica, $U$ quella potenziale e $W$ il lavoro di tutte le forze agenti. Quale delle seguenti relazioni è vera?
  \begin{enumerate}[label=\Alph*.]
    \item $\Delta E_m=W$.
    \item $\Delta U=W$.
    \item $\Delta U=W$.
    \item $\Delta K=W.$
  \end{enumerate}
  \item Un corpo impatta una molla con una velocità $v$. La molla di costante elastica $k$, opponendosi all'avanzare del corpo, lo ferma. Quale è il suo allungamento nel momento in cui si ferma, trascurando l'attrito?
  \begin{enumerate}[label=\Alph*.]
    \item $\sqrt{\frac{m}{k}v}$.
    \item $\sqrt{\frac{m}{k}}v$.
    \item $\sqrt{\frac{k}{m}}v$.
    \item $\sqrt{\frac{k}{m}}v$.
  \end{enumerate}
  \item Un corpo sale da terra fino a un'altezza $h$. Il lavoro della forza peso è:
  \begin{enumerate}[label=\Alph*.]
    \item nullo
    \item negativo.
    \item positivo.
    \item $mgh$
  \end{enumerate}
  \item Un corpo di massa $m$ scivola su un piano con attrito, partendo da una velocità v. Il coefficiente di attrito è \mu. Dopo quanto spazio si ferma?
  \begin{enumerate}[label=\Alph*.]
    \item $\frac{v^2}{2g\mu}}$.
    \item $\frac{2v^2}{g\mu}}$.
    \item $\frac{1}{2}v^2-\mu g$.
    \item $\frac{1}{2}v^2+\mu g$.
  \end{enumerate}
  \item Sia \vec{F} una forza conservativa. Allora
  \begin{enumerate}[label=\Alph*.]
    \item il lavoro di \vec{F} è nullo solo se il percorso è chiuso.
    \item il lavoro di \vec{F} può essere nullo anche se il percorso non è chiuso.
    \item il lavoro di \vec{F} non dipende dal percorso considerato.
    \item il lavoro di \vec{F} è sempre nullo.
  \end{enumerate}
  \item Un corpo sale lungo un piano inclinato di un angolo $\alpha$ e alto $h$, partendo da terra e arrivando fino alla cima del piano. Il lavoro della forza peso è:
  \begin{enumerate}[label=\Alph*.]
    \item $mgh\sin\alpha$
    \item $mgh$
    \item $mgh\cos\alpha$
    \item $-mgh$
  \end{enumerate}
  \item Sia \vec{F} una forza non conservativa. Allora
  \begin{enumerate}[label=\Alph*.]
    \item la sua energia potenziale è tale che lungo un percorso $\Delta U=W$.
    \item la sua energia potenziale è tale che lungo un percorso $\Delta U=-W$.
    \item la sua energia potenziale non esiste.
    \item la sua energia potenziale è sempre negativa.
  \end{enumerate}
  \item Il lavoro della forza d'attrito
  \begin{enumerate}[label=\Alph*.]
    \item è sempre positivo.
    \item è nullo se il percorso è rettilineo.
    \item non può essere calcolato perché la forza non è conservativa.
    \item è sempre negativo.
  \end{enumerate}
  \item Se calcoliamo il lavoro di una forza lungo un percorso chiuso e troviamo zero, cosa possiamo concludere?
  \begin{enumerate}[label=\Alph*.]
    \item la forza non è conservativa.
    \item la forza è conservativa.
    \item niente.
    \item la forza è nulla.
  \end{enumerate}
  \item Sia \vec{F} una forza non conservativa. Allora
  \begin{enumerate}[label=\Alph*.]
    \item esiste un percorso chiuso il cui lavoro non è nullo.
    \item il lavoro non dipende dal percorso.
    \item il lavoro non dipende dagli estremi del percorso.
    \item per ogni percorso aperto il lavoro è nullo.
  \end{enumerate}
  \item Sia \vec{F} una forza generica, che agisce su un corpo che effettua uno spostamento $\Delta \vec{s}$. Allora
  \begin{enumerate}[label=\Alph*.]
    \item il suo lavoro non ha una direzione perché è scalare.
    \item il suo lavoro ha la stessa direzione della forza.
    \item il suo lavoro ha la stessa direzione dello spostamento.
    \item il suo lavoro ha una direzione data dalla regola del parallelogramma.
  \end{enumerate}
  \item Sia $E_m$ l'energia meccanica, $K$ l'energia cinetica, $U$ quella potenziale e $W_{NC}$ il lavoro delle forze non conservative agenti. Quale delle seguenti relazioni è vera?
  \begin{enumerate}[label=\Alph*.]
    \item $\Delta E_m=W_{NC}$.
    \item $\Delta K = \Delta U$.
    \item $\Delta U=W_{NC}$.
    \item $\Delta K=W_{NC}.$
  \end{enumerate}
\end{enumerate}








\newpage \maketitle \centering \textbf{Griglia di valutazione}. Eventuali mezzi punti saranno arrotondati all'intero precedente. \begin{table}[h]     \centering \begin{tabular}{|c|c|c|c|c|c|c|c|c|c|c|c|c|c|c|c|c|c|c|c|} \hline Punti &  $\leq 4$ & 5 & 6 & 7 & 8 & 9 & 10 & 11 & \textbf{12} & 13 & 14 & 15 & 16 & 17 & 18 & 19 & 20 \\ \hline Voto & 2 & 2.5 & 3 & 3.5 & 4 & 4.5 & 5 & 5.5 & \textbf{6} & 6.5 & 7 & 7.5 & 8 & 8.5 & 9 & 9.5 & 10 \\ \hline \end{tabular} \end{table} 
\textbf{Codice}: ptjvbuhbwlofryailtghp


\begin{enumerate}
  \item Un corpo di massa $m_1$ e un corpo di massa $m_2$ sono lanciati in contemporanea da una torre alta $h$. Quali sono le velocità dei due corpi appena prima di toccare terra?
  \begin{enumerate}[label=\Alph*.]
    \item $v_1=\sqrt{2gh}, v_2=\sqrt{2gh}$.
    \item $v_1=\sqrt{2m_1gh}, v_2=\sqrt{2m_2gh}$.
    \item v_1=0, v_2=0
    \item $v_1=\sqrt{\frac{2gh}{m_1}}, v_2=\sqrt{\frac{2gh}{m_2}}$.
  \end{enumerate}
  \item Sia \vec{F} una forza non conservativa. Allora
  \begin{enumerate}[label=\Alph*.]
    \item la sua energia potenziale è sempre negativa.
    \item la sua energia potenziale è tale che lungo un percorso $\Delta U=-W$.
    \item la sua energia potenziale non esiste.
    \item la sua energia potenziale è tale che lungo un percorso $\Delta U=W$.
  \end{enumerate}
  \item Un corpo, partendo da fermo, rotola giù dalla cima di un piano inclinato di un angolo $\alpha$, alto $h$. Quale è la sua velocità quando arriva in fondo?
  \begin{enumerate}[label=\Alph*.]
    \item $\sqrt{mgh\cos\alpha}$.
    \item $\sqrt{2gh}$.
    \item $\sqrt{2gh\sin\alpha}$.
    \item $\sqrt{2gh}\sin\alpha$.
  \end{enumerate}
  \item Sia $E_m$ l'energia meccanica, $K$ l'energia cinetica, $U$ quella potenziale e $W_{NC}$ il lavoro delle forze non conservative agenti. Quale delle seguenti relazioni è vera?
  \begin{enumerate}[label=\Alph*.]
    \item $\Delta E_m=W_{NC}$.
    \item $\Delta U=W_{NC}$.
    \item $\Delta K = \Delta U$.
    \item $\Delta K=W_{NC}.$
  \end{enumerate}
  \item Se calcoliamo il lavoro di una forza lungo un percorso chiuso e troviamo zero, cosa possiamo concludere?
  \begin{enumerate}[label=\Alph*.]
    \item la forza è conservativa.
    \item niente.
    \item la forza è nulla.
    \item la forza non è conservativa.
  \end{enumerate}
  \item Un corpo sale da terra fino a un'altezza $h$. Il lavoro della forza peso è:
  \begin{enumerate}[label=\Alph*.]
    \item $mgh$
    \item negativo.
    \item positivo.
    \item nullo
  \end{enumerate}
  \item Un corpo viene lasciato cadere da fermo da un altezza $h$. Quando raggiunge il suolo ha una velocità $v$. Quale relazione esprime correttamente la conservazione dell'energia?
  \begin{enumerate}[label=\Alph*.]
    \item $\frac{1}{2}mh^2=\frac{1}{2}mv^2$.
    \item $mgh+\frac{1}{2}mv^2=\frac{1}{2}mv^2.$
    \item $\frac{1}{2}mv^2+mgh=0.$
    \item $mgh=\frac{1}{2}mv^2$.
  \end{enumerate}
  \item Un corpo di massa $m$ scivola su un piano con attrito, partendo da una velocità v. Il coefficiente di attrito è \mu. Dopo quanto spazio si ferma?
  \begin{enumerate}[label=\Alph*.]
    \item $\frac{1}{2}v^2+\mu g$.
    \item $\frac{1}{2}v^2-\mu g$.
    \item $\frac{v^2}{2g\mu}}$.
    \item $\frac{2v^2}{g\mu}}$.
  \end{enumerate}
  \item Sia \vec{F} una forza generica, che agisce su un corpo che effettua uno spostamento $\Delta \vec{s}$. Allora
  \begin{enumerate}[label=\Alph*.]
    \item il suo lavoro non ha una direzione perché è scalare.
    \item il suo lavoro ha la stessa direzione della forza.
    \item il suo lavoro ha una direzione data dalla regola del parallelogramma.
    \item il suo lavoro ha la stessa direzione dello spostamento.
  \end{enumerate}
  \item L'energia meccanica di un corpo che si trova a un'altezza $h$
  \begin{enumerate}[label=\Alph*.]
    \item è negativa
    \item è nulla.
    \item è $mgh$.
    \item dipende dalla velocità del corpo.
  \end{enumerate}
  \item Un corpo impatta una molla con una velocità $v$. La molla di costante elastica $k$, opponendosi all'avanzare del corpo, lo ferma. Quale è il suo allungamento nel momento in cui si ferma, trascurando l'attrito?
  \begin{enumerate}[label=\Alph*.]
    \item $\sqrt{\frac{k}{m}}v$.
    \item $\sqrt{\frac{m}{k}}v$.
    \item $\sqrt{\frac{k}{m}}v$.
    \item $\sqrt{\frac{m}{k}v}$.
  \end{enumerate}
  \item Un corpo sale da terra fino a un'altezza $h$. Il lavoro della forza peso è:
  \begin{enumerate}[label=\Alph*.]
    \item $mgh$
    \item nullo
    \item negativo.
    \item positivo.
  \end{enumerate}
  \item Sia $E_m$ l'energia meccanica, $K$ l'energia cinetica, $U$ quella potenziale e $W$ il lavoro di tutte le forze agenti. Quale delle seguenti relazioni è vera?
  \begin{enumerate}[label=\Alph*.]
    \item $\Delta U=W$.
    \item $\Delta E_m=W$.
    \item $\Delta U=W$.
    \item $\Delta K=W.$
  \end{enumerate}
  \item Sia \vec{F} una forza conservativa. Allora
  \begin{enumerate}[label=\Alph*.]
    \item il lavoro di \vec{F} è sempre nullo.
    \item il lavoro di \vec{F} non dipende dal percorso considerato.
    \item il lavoro di \vec{F} può essere nullo anche se il percorso non è chiuso.
    \item il lavoro di \vec{F} è nullo solo se il percorso è chiuso.
  \end{enumerate}
  \item Un corpo sale lungo un piano inclinato di un angolo $\alpha$ e alto $h$, partendo da terra e arrivando fino alla cima del piano. Il lavoro della forza peso è:
  \begin{enumerate}[label=\Alph*.]
    \item $mgh\cos\alpha$
    \item $-mgh$
    \item $mgh$
    \item $mgh\sin\alpha$
  \end{enumerate}
  \item Il lavoro della reazione vincolare
  \begin{enumerate}[label=\Alph*.]
    \item è sempre diverso da zero.
    \item è nullo solo se il percorso è chiuso.
    \item può assumere qualsiasi valore.
    \item è sempre nullo.
  \end{enumerate}
  \item L'energia cinetica di un corpo fermo che si trova a un'altezza $h$
  \begin{enumerate}[label=\Alph*.]
    \item dipende da $h$
    \item è $mgh$.
    \item si conserva.
    \item è nulla.
  \end{enumerate}
  \item Sia \vec{F} una forza non conservativa. Allora
  \begin{enumerate}[label=\Alph*.]
    \item il lavoro non dipende dagli estremi del percorso.
    \item il lavoro non dipende dal percorso.
    \item esiste un percorso chiuso il cui lavoro non è nullo.
    \item per ogni percorso aperto il lavoro è nullo.
  \end{enumerate}
  \item Il lavoro della forza peso
  \begin{enumerate}[label=\Alph*.]
    \item è nullo solo se il percorso è chiuso.
    \item è sempre negativo.
    \item è nullo se il percorso ha gli estremi posti alla stessa altitudine.
    \item è sempre positivo.
  \end{enumerate}
  \item Un corpo ha inizialmente una velocità $v$ e dopo un certo tempo si ferma. La variazione di energia cinetica è
  \begin{enumerate}[label=\Alph*.]
    \item è positiva.
    \item dipende dallo spazio percorso.
    \item è negativa
    \item dipende dal tempo in cui si ferma.
  \end{enumerate}
  \item Il lavoro della forza d'attrito
  \begin{enumerate}[label=\Alph*.]
    \item è nullo se il percorso è rettilineo.
    \item è sempre negativo.
    \item non può essere calcolato perché la forza non è conservativa.
    \item è sempre positivo.
  \end{enumerate}
\end{enumerate}








\newpage \maketitle \centering \textbf{Griglia di valutazione}. Eventuali mezzi punti saranno arrotondati all'intero precedente. \begin{table}[h]     \centering \begin{tabular}{|c|c|c|c|c|c|c|c|c|c|c|c|c|c|c|c|c|c|c|c|} \hline Punti &  $\leq 4$ & 5 & 6 & 7 & 8 & 9 & 10 & 11 & \textbf{12} & 13 & 14 & 15 & 16 & 17 & 18 & 19 & 20 \\ \hline Voto & 2 & 2.5 & 3 & 3.5 & 4 & 4.5 & 5 & 5.5 & \textbf{6} & 6.5 & 7 & 7.5 & 8 & 8.5 & 9 & 9.5 & 10 \\ \hline \end{tabular} \end{table}
\textbf{Codice}: prjwawhbylqeqz fiuhgp


\begin{enumerate}
  \item Un corpo sale da terra fino a un'altezza $h$. Il lavoro della forza peso è:
  \begin{enumerate}[label=\Alph*.]
    \item negativo.
    \item positivo.
    \item nullo
    \item $mgh$
  \end{enumerate}
  \item Sia $E_m$ l'energia meccanica, $K$ l'energia cinetica, $U$ quella potenziale e $W$ il lavoro di tutte le forze agenti. Quale delle seguenti relazioni è vera?
  \begin{enumerate}[label=\Alph*.]
    \item $\Delta K=W.$
    \item $\Delta U=W$.
    \item $\Delta E_m=W$.
    \item $\Delta U=W$.
  \end{enumerate}
  \item Sia \vec{F} una forza conservativa. Allora
  \begin{enumerate}[label=\Alph*.]
    \item il lavoro di \vec{F} è nullo solo se il percorso è chiuso.
    \item il lavoro di \vec{F} può essere nullo anche se il percorso non è chiuso.
    \item il lavoro di \vec{F} è sempre nullo.
    \item il lavoro di \vec{F} non dipende dal percorso considerato.
  \end{enumerate}
  \item Un corpo ha inizialmente una velocità $v$ e dopo un certo tempo si ferma. La variazione di energia cinetica è
  \begin{enumerate}[label=\Alph*.]
    \item dipende dallo spazio percorso.
    \item è negativa
    \item dipende dal tempo in cui si ferma.
    \item è positiva.
  \end{enumerate}
  \item Un corpo di massa $m$ scivola su un piano con attrito, partendo da una velocità v. Il coefficiente di attrito è \mu. Dopo quanto spazio si ferma?
  \begin{enumerate}[label=\Alph*.]
    \item $\frac{v^2}{2g\mu}}$.
    \item $\frac{1}{2}v^2-\mu g$.
    \item $\frac{1}{2}v^2+\mu g$.
    \item $\frac{2v^2}{g\mu}}$.
  \end{enumerate}
  \item Un corpo viene lasciato cadere da fermo da un altezza $h$. Quando raggiunge il suolo ha una velocità $v$. Quale relazione esprime correttamente la conservazione dell'energia?
  \begin{enumerate}[label=\Alph*.]
    \item $\frac{1}{2}mv^2+mgh=0.$
    \item $mgh+\frac{1}{2}mv^2=\frac{1}{2}mv^2.$
    \item $\frac{1}{2}mh^2=\frac{1}{2}mv^2$.
    \item $mgh=\frac{1}{2}mv^2$.
  \end{enumerate}
  \item Il lavoro della reazione vincolare
  \begin{enumerate}[label=\Alph*.]
    \item è sempre diverso da zero.
    \item è nullo solo se il percorso è chiuso.
    \item può assumere qualsiasi valore.
    \item è sempre nullo.
  \end{enumerate}
  \item Sia \vec{F} una forza non conservativa. Allora
  \begin{enumerate}[label=\Alph*.]
    \item la sua energia potenziale è tale che lungo un percorso $\Delta U=-W$.
    \item la sua energia potenziale è sempre negativa.
    \item la sua energia potenziale non esiste.
    \item la sua energia potenziale è tale che lungo un percorso $\Delta U=W$.
  \end{enumerate}
  \item Il lavoro della forza d'attrito
  \begin{enumerate}[label=\Alph*.]
    \item è sempre positivo.
    \item non può essere calcolato perché la forza non è conservativa.
    \item è sempre negativo.
    \item è nullo se il percorso è rettilineo.
  \end{enumerate}
  \item Il lavoro della forza peso
  \begin{enumerate}[label=\Alph*.]
    \item è sempre positivo.
    \item è sempre negativo.
    \item è nullo solo se il percorso è chiuso.
    \item è nullo se il percorso ha gli estremi posti alla stessa altitudine.
  \end{enumerate}
  \item Un corpo di massa $m_1$ e un corpo di massa $m_2$ sono lanciati in contemporanea da una torre alta $h$. Quali sono le velocità dei due corpi appena prima di toccare terra?
  \begin{enumerate}[label=\Alph*.]
    \item v_1=0, v_2=0
    \item $v_1=\sqrt{\frac{2gh}{m_1}}, v_2=\sqrt{\frac{2gh}{m_2}}$.
    \item $v_1=\sqrt{2m_1gh}, v_2=\sqrt{2m_2gh}$.
    \item $v_1=\sqrt{2gh}, v_2=\sqrt{2gh}$.
  \end{enumerate}
  \item Sia $E_m$ l'energia meccanica, $K$ l'energia cinetica, $U$ quella potenziale e $W_{NC}$ il lavoro delle forze non conservative agenti. Quale delle seguenti relazioni è vera?
  \begin{enumerate}[label=\Alph*.]
    \item $\Delta K = \Delta U$.
    \item $\Delta E_m=W_{NC}$.
    \item $\Delta U=W_{NC}$.
    \item $\Delta K=W_{NC}.$
  \end{enumerate}
  \item Un corpo impatta una molla con una velocità $v$. La molla di costante elastica $k$, opponendosi all'avanzare del corpo, lo ferma. Quale è il suo allungamento nel momento in cui si ferma, trascurando l'attrito?
  \begin{enumerate}[label=\Alph*.]
    \item $\sqrt{\frac{k}{m}}v$.
    \item $\sqrt{\frac{k}{m}}v$.
    \item $\sqrt{\frac{m}{k}}v$.
    \item $\sqrt{\frac{m}{k}v}$.
  \end{enumerate}
  \item L'energia cinetica di un corpo fermo che si trova a un'altezza $h$
  \begin{enumerate}[label=\Alph*.]
    \item si conserva.
    \item è $mgh$.
    \item dipende da $h$
    \item è nulla.
  \end{enumerate}
  \item Sia \vec{F} una forza generica, che agisce su un corpo che effettua uno spostamento $\Delta \vec{s}$. Allora
  \begin{enumerate}[label=\Alph*.]
    \item il suo lavoro non ha una direzione perché è scalare.
    \item il suo lavoro ha la stessa direzione dello spostamento.
    \item il suo lavoro ha la stessa direzione della forza.
    \item il suo lavoro ha una direzione data dalla regola del parallelogramma.
  \end{enumerate}
  \item Un corpo sale lungo un piano inclinato di un angolo $\alpha$ e alto $h$, partendo da terra e arrivando fino alla cima del piano. Il lavoro della forza peso è:
  \begin{enumerate}[label=\Alph*.]
    \item $-mgh$
    \item $mgh\sin\alpha$
    \item $mgh$
    \item $mgh\cos\alpha$
  \end{enumerate}
  \item Se calcoliamo il lavoro di una forza lungo un percorso chiuso e troviamo zero, cosa possiamo concludere?
  \begin{enumerate}[label=\Alph*.]
    \item niente.
    \item la forza non è conservativa.
    \item la forza è nulla.
    \item la forza è conservativa.
  \end{enumerate}
  \item Sia \vec{F} una forza non conservativa. Allora
  \begin{enumerate}[label=\Alph*.]
    \item il lavoro non dipende dal percorso.
    \item per ogni percorso aperto il lavoro è nullo.
    \item il lavoro non dipende dagli estremi del percorso.
    \item esiste un percorso chiuso il cui lavoro non è nullo.
  \end{enumerate}
  \item Un corpo sale da terra fino a un'altezza $h$. Il lavoro della forza peso è:
  \begin{enumerate}[label=\Alph*.]
    \item positivo.
    \item $mgh$
    \item nullo
    \item negativo.
  \end{enumerate}
  \item L'energia meccanica di un corpo che si trova a un'altezza $h$
  \begin{enumerate}[label=\Alph*.]
    \item è nulla.
    \item dipende dalla velocità del corpo.
    \item è negativa
    \item è $mgh$.
  \end{enumerate}
  \item Un corpo, partendo da fermo, rotola giù dalla cima di un piano inclinato di un angolo $\alpha$, alto $h$. Quale è la sua velocità quando arriva in fondo?
  \begin{enumerate}[label=\Alph*.]
    \item $\sqrt{2gh\sin\alpha}$.
    \item $\sqrt{2gh}$.
    \item $\sqrt{mgh\cos\alpha}$.
    \item $\sqrt{2gh}\sin\alpha$.
  \end{enumerate}
\end{enumerate}








\newpage \maketitle \centering \textbf{Griglia di valutazione}. Eventuali mezzi punti saranno arrotondati all'intero precedente. \begin{table}[h]     \centering \begin{tabular}{|c|c|c|c|c|c|c|c|c|c|c|c|c|c|c|c|c|c|c|c|} \hline Punti &  $\leq 4$ & 5 & 6 & 7 & 8 & 9 & 10 & 11 & \textbf{12} & 13 & 14 & 15 & 16 & 17 & 18 & 19 & 20 \\ \hline Voto & 2 & 2.5 & 3 & 3.5 & 4 & 4.5 & 5 & 5.5 & \textbf{6} & 6.5 & 7 & 7.5 & 8 & 8.5 & 9 & 9.5 & 10 \\ \hline \end{tabular} \end{table}
\textbf{Codice}: ruixavgcwjnepzbiishgp


\begin{enumerate}
  \item L'energia cinetica di un corpo fermo che si trova a un'altezza $h$
  \begin{enumerate}[label=\Alph*.]
    \item dipende da $h$
    \item si conserva.
    \item è nulla.
    \item è $mgh$.
  \end{enumerate}
  \item Sia \vec{F} una forza non conservativa. Allora
  \begin{enumerate}[label=\Alph*.]
    \item la sua energia potenziale è tale che lungo un percorso $\Delta U=W$.
    \item la sua energia potenziale è sempre negativa.
    \item la sua energia potenziale è tale che lungo un percorso $\Delta U=-W$.
    \item la sua energia potenziale non esiste.
  \end{enumerate}
  \item Il lavoro della forza peso
  \begin{enumerate}[label=\Alph*.]
    \item è nullo se il percorso ha gli estremi posti alla stessa altitudine.
    \item è sempre positivo.
    \item è sempre negativo.
    \item è nullo solo se il percorso è chiuso.
  \end{enumerate}
  \item Un corpo, partendo da fermo, rotola giù dalla cima di un piano inclinato di un angolo $\alpha$, alto $h$. Quale è la sua velocità quando arriva in fondo?
  \begin{enumerate}[label=\Alph*.]
    \item $\sqrt{2gh\sin\alpha}$.
    \item $\sqrt{mgh\cos\alpha}$.
    \item $\sqrt{2gh}$.
    \item $\sqrt{2gh}\sin\alpha$.
  \end{enumerate}
  \item Il lavoro della reazione vincolare
  \begin{enumerate}[label=\Alph*.]
    \item è sempre nullo.
    \item è nullo solo se il percorso è chiuso.
    \item può assumere qualsiasi valore.
    \item è sempre diverso da zero.
  \end{enumerate}
  \item Un corpo sale da terra fino a un'altezza $h$. Il lavoro della forza peso è:
  \begin{enumerate}[label=\Alph*.]
    \item $mgh$
    \item positivo.
    \item negativo.
    \item nullo
  \end{enumerate}
  \item Se calcoliamo il lavoro di una forza lungo un percorso chiuso e troviamo zero, cosa possiamo concludere?
  \begin{enumerate}[label=\Alph*.]
    \item la forza è nulla.
    \item la forza non è conservativa.
    \item niente.
    \item la forza è conservativa.
  \end{enumerate}
  \item Sia \vec{F} una forza non conservativa. Allora
  \begin{enumerate}[label=\Alph*.]
    \item per ogni percorso aperto il lavoro è nullo.
    \item il lavoro non dipende dagli estremi del percorso.
    \item il lavoro non dipende dal percorso.
    \item esiste un percorso chiuso il cui lavoro non è nullo.
  \end{enumerate}
  \item Un corpo impatta una molla con una velocità $v$. La molla di costante elastica $k$, opponendosi all'avanzare del corpo, lo ferma. Quale è il suo allungamento nel momento in cui si ferma, trascurando l'attrito?
  \begin{enumerate}[label=\Alph*.]
    \item $\sqrt{\frac{m}{k}}v$.
    \item $\sqrt{\frac{k}{m}}v$.
    \item $\sqrt{\frac{m}{k}v}$.
    \item $\sqrt{\frac{k}{m}}v$.
  \end{enumerate}
  \item Il lavoro della forza d'attrito
  \begin{enumerate}[label=\Alph*.]
    \item è nullo se il percorso è rettilineo.
    \item è sempre negativo.
    \item è sempre positivo.
    \item non può essere calcolato perché la forza non è conservativa.
  \end{enumerate}
  \item Un corpo ha inizialmente una velocità $v$ e dopo un certo tempo si ferma. La variazione di energia cinetica è
  \begin{enumerate}[label=\Alph*.]
    \item è negativa
    \item dipende dallo spazio percorso.
    \item dipende dal tempo in cui si ferma.
    \item è positiva.
  \end{enumerate}
  \item Sia \vec{F} una forza conservativa. Allora
  \begin{enumerate}[label=\Alph*.]
    \item il lavoro di \vec{F} è nullo solo se il percorso è chiuso.
    \item il lavoro di \vec{F} può essere nullo anche se il percorso non è chiuso.
    \item il lavoro di \vec{F} è sempre nullo.
    \item il lavoro di \vec{F} non dipende dal percorso considerato.
  \end{enumerate}
  \item Un corpo di massa $m$ scivola su un piano con attrito, partendo da una velocità v. Il coefficiente di attrito è \mu. Dopo quanto spazio si ferma?
  \begin{enumerate}[label=\Alph*.]
    \item $\frac{1}{2}v^2+\mu g$.
    \item $\frac{v^2}{2g\mu}}$.
    \item $\frac{2v^2}{g\mu}}$.
    \item $\frac{1}{2}v^2-\mu g$.
  \end{enumerate}
  \item Sia $E_m$ l'energia meccanica, $K$ l'energia cinetica, $U$ quella potenziale e $W_{NC}$ il lavoro delle forze non conservative agenti. Quale delle seguenti relazioni è vera?
  \begin{enumerate}[label=\Alph*.]
    \item $\Delta U=W_{NC}$.
    \item $\Delta K=W_{NC}.$
    \item $\Delta K = \Delta U$.
    \item $\Delta E_m=W_{NC}$.
  \end{enumerate}
  \item Un corpo sale lungo un piano inclinato di un angolo $\alpha$ e alto $h$, partendo da terra e arrivando fino alla cima del piano. Il lavoro della forza peso è:
  \begin{enumerate}[label=\Alph*.]
    \item $mgh$
    \item $mgh\cos\alpha$
    \item $-mgh$
    \item $mgh\sin\alpha$
  \end{enumerate}
  \item Un corpo sale da terra fino a un'altezza $h$. Il lavoro della forza peso è:
  \begin{enumerate}[label=\Alph*.]
    \item nullo
    \item positivo.
    \item $mgh$
    \item negativo.
  \end{enumerate}
  \item Un corpo viene lasciato cadere da fermo da un altezza $h$. Quando raggiunge il suolo ha una velocità $v$. Quale relazione esprime correttamente la conservazione dell'energia?
  \begin{enumerate}[label=\Alph*.]
    \item $mgh=\frac{1}{2}mv^2$.
    \item $mgh+\frac{1}{2}mv^2=\frac{1}{2}mv^2.$
    \item $\frac{1}{2}mh^2=\frac{1}{2}mv^2$.
    \item $\frac{1}{2}mv^2+mgh=0.$
  \end{enumerate}
  \item Sia \vec{F} una forza generica, che agisce su un corpo che effettua uno spostamento $\Delta \vec{s}$. Allora
  \begin{enumerate}[label=\Alph*.]
    \item il suo lavoro ha la stessa direzione dello spostamento.
    \item il suo lavoro non ha una direzione perché è scalare.
    \item il suo lavoro ha la stessa direzione della forza.
    \item il suo lavoro ha una direzione data dalla regola del parallelogramma.
  \end{enumerate}
  \item L'energia meccanica di un corpo che si trova a un'altezza $h$
  \begin{enumerate}[label=\Alph*.]
    \item è $mgh$.
    \item è negativa
    \item è nulla.
    \item dipende dalla velocità del corpo.
  \end{enumerate}
  \item Un corpo di massa $m_1$ e un corpo di massa $m_2$ sono lanciati in contemporanea da una torre alta $h$. Quali sono le velocità dei due corpi appena prima di toccare terra?
  \begin{enumerate}[label=\Alph*.]
    \item $v_1=\sqrt{2m_1gh}, v_2=\sqrt{2m_2gh}$.
    \item $v_1=\sqrt{2gh}, v_2=\sqrt{2gh}$.
    \item v_1=0, v_2=0
    \item $v_1=\sqrt{\frac{2gh}{m_1}}, v_2=\sqrt{\frac{2gh}{m_2}}$.
  \end{enumerate}
  \item Sia $E_m$ l'energia meccanica, $K$ l'energia cinetica, $U$ quella potenziale e $W$ il lavoro di tutte le forze agenti. Quale delle seguenti relazioni è vera?
  \begin{enumerate}[label=\Alph*.]
    \item $\Delta E_m=W$.
    \item $\Delta K=W.$
    \item $\Delta U=W$.
    \item $\Delta U=W$.
  \end{enumerate}
\end{enumerate}








\newpage \maketitle \centering \textbf{Griglia di valutazione}. Eventuali mezzi punti saranno arrotondati all'intero precedente. \begin{table}[h]     \centering \begin{tabular}{|c|c|c|c|c|c|c|c|c|c|c|c|c|c|c|c|c|c|c|c|} \hline Punti &  $\leq 4$ & 5 & 6 & 7 & 8 & 9 & 10 & 11 & \textbf{12} & 13 & 14 & 15 & 16 & 17 & 18 & 19 & 20 \\ \hline Voto & 2 & 2.5 & 3 & 3.5 & 4 & 4.5 & 5 & 5.5 & \textbf{6} & 6.5 & 7 & 7.5 & 8 & 8.5 & 9 & 9.5 & 10 \\ \hline \end{tabular} \end{table}
\textbf{Codice}: rriwatg zlnepy iluggp


\begin{enumerate}
  \item Un corpo di massa $m$ scivola su un piano con attrito, partendo da una velocità v. Il coefficiente di attrito è \mu. Dopo quanto spazio si ferma?
  \begin{enumerate}[label=\Alph*.]
    \item $\frac{1}{2}v^2+\mu g$.
    \item $\frac{2v^2}{g\mu}}$.
    \item $\frac{v^2}{2g\mu}}$.
    \item $\frac{1}{2}v^2-\mu g$.
  \end{enumerate}
  \item Un corpo, partendo da fermo, rotola giù dalla cima di un piano inclinato di un angolo $\alpha$, alto $h$. Quale è la sua velocità quando arriva in fondo?
  \begin{enumerate}[label=\Alph*.]
    \item $\sqrt{2gh}$.
    \item $\sqrt{2gh\sin\alpha}$.
    \item $\sqrt{mgh\cos\alpha}$.
    \item $\sqrt{2gh}\sin\alpha$.
  \end{enumerate}
  \item Sia \vec{F} una forza conservativa. Allora
  \begin{enumerate}[label=\Alph*.]
    \item il lavoro di \vec{F} può essere nullo anche se il percorso non è chiuso.
    \item il lavoro di \vec{F} non dipende dal percorso considerato.
    \item il lavoro di \vec{F} è nullo solo se il percorso è chiuso.
    \item il lavoro di \vec{F} è sempre nullo.
  \end{enumerate}
  \item Un corpo sale lungo un piano inclinato di un angolo $\alpha$ e alto $h$, partendo da terra e arrivando fino alla cima del piano. Il lavoro della forza peso è:
  \begin{enumerate}[label=\Alph*.]
    \item $mgh\cos\alpha$
    \item $-mgh$
    \item $mgh\sin\alpha$
    \item $mgh$
  \end{enumerate}
  \item Il lavoro della reazione vincolare
  \begin{enumerate}[label=\Alph*.]
    \item è sempre nullo.
    \item è sempre diverso da zero.
    \item può assumere qualsiasi valore.
    \item è nullo solo se il percorso è chiuso.
  \end{enumerate}
  \item Un corpo sale da terra fino a un'altezza $h$. Il lavoro della forza peso è:
  \begin{enumerate}[label=\Alph*.]
    \item negativo.
    \item $mgh$
    \item nullo
    \item positivo.
  \end{enumerate}
  \item Sia \vec{F} una forza non conservativa. Allora
  \begin{enumerate}[label=\Alph*.]
    \item la sua energia potenziale è tale che lungo un percorso $\Delta U=W$.
    \item la sua energia potenziale è tale che lungo un percorso $\Delta U=-W$.
    \item la sua energia potenziale non esiste.
    \item la sua energia potenziale è sempre negativa.
  \end{enumerate}
  \item Sia \vec{F} una forza non conservativa. Allora
  \begin{enumerate}[label=\Alph*.]
    \item esiste un percorso chiuso il cui lavoro non è nullo.
    \item il lavoro non dipende dal percorso.
    \item il lavoro non dipende dagli estremi del percorso.
    \item per ogni percorso aperto il lavoro è nullo.
  \end{enumerate}
  \item Un corpo sale da terra fino a un'altezza $h$. Il lavoro della forza peso è:
  \begin{enumerate}[label=\Alph*.]
    \item nullo
    \item positivo.
    \item $mgh$
    \item negativo.
  \end{enumerate}
  \item Un corpo viene lasciato cadere da fermo da un altezza $h$. Quando raggiunge il suolo ha una velocità $v$. Quale relazione esprime correttamente la conservazione dell'energia?
  \begin{enumerate}[label=\Alph*.]
    \item $mgh+\frac{1}{2}mv^2=\frac{1}{2}mv^2.$
    \item $\frac{1}{2}mh^2=\frac{1}{2}mv^2$.
    \item $\frac{1}{2}mv^2+mgh=0.$
    \item $mgh=\frac{1}{2}mv^2$.
  \end{enumerate}
  \item Un corpo di massa $m_1$ e un corpo di massa $m_2$ sono lanciati in contemporanea da una torre alta $h$. Quali sono le velocità dei due corpi appena prima di toccare terra?
  \begin{enumerate}[label=\Alph*.]
    \item $v_1=\sqrt{2gh}, v_2=\sqrt{2gh}$.
    \item $v_1=\sqrt{\frac{2gh}{m_1}}, v_2=\sqrt{\frac{2gh}{m_2}}$.
    \item $v_1=\sqrt{2m_1gh}, v_2=\sqrt{2m_2gh}$.
    \item v_1=0, v_2=0
  \end{enumerate}
  \item Un corpo impatta una molla con una velocità $v$. La molla di costante elastica $k$, opponendosi all'avanzare del corpo, lo ferma. Quale è il suo allungamento nel momento in cui si ferma, trascurando l'attrito?
  \begin{enumerate}[label=\Alph*.]
    \item $\sqrt{\frac{k}{m}}v$.
    \item $\sqrt{\frac{m}{k}}v$.
    \item $\sqrt{\frac{m}{k}v}$.
    \item $\sqrt{\frac{k}{m}}v$.
  \end{enumerate}
  \item Un corpo ha inizialmente una velocità $v$ e dopo un certo tempo si ferma. La variazione di energia cinetica è
  \begin{enumerate}[label=\Alph*.]
    \item è positiva.
    \item è negativa
    \item dipende dal tempo in cui si ferma.
    \item dipende dallo spazio percorso.
  \end{enumerate}
  \item Se calcoliamo il lavoro di una forza lungo un percorso chiuso e troviamo zero, cosa possiamo concludere?
  \begin{enumerate}[label=\Alph*.]
    \item la forza è nulla.
    \item la forza è conservativa.
    \item niente.
    \item la forza non è conservativa.
  \end{enumerate}
  \item Sia $E_m$ l'energia meccanica, $K$ l'energia cinetica, $U$ quella potenziale e $W_{NC}$ il lavoro delle forze non conservative agenti. Quale delle seguenti relazioni è vera?
  \begin{enumerate}[label=\Alph*.]
    \item $\Delta E_m=W_{NC}$.
    \item $\Delta K = \Delta U$.
    \item $\Delta K=W_{NC}.$
    \item $\Delta U=W_{NC}$.
  \end{enumerate}
  \item L'energia cinetica di un corpo fermo che si trova a un'altezza $h$
  \begin{enumerate}[label=\Alph*.]
    \item si conserva.
    \item è $mgh$.
    \item dipende da $h$
    \item è nulla.
  \end{enumerate}
  \item Sia \vec{F} una forza generica, che agisce su un corpo che effettua uno spostamento $\Delta \vec{s}$. Allora
  \begin{enumerate}[label=\Alph*.]
    \item il suo lavoro ha la stessa direzione dello spostamento.
    \item il suo lavoro ha una direzione data dalla regola del parallelogramma.
    \item il suo lavoro ha la stessa direzione della forza.
    \item il suo lavoro non ha una direzione perché è scalare.
  \end{enumerate}
  \item Il lavoro della forza d'attrito
  \begin{enumerate}[label=\Alph*.]
    \item è nullo se il percorso è rettilineo.
    \item è sempre positivo.
    \item non può essere calcolato perché la forza non è conservativa.
    \item è sempre negativo.
  \end{enumerate}
  \item Sia $E_m$ l'energia meccanica, $K$ l'energia cinetica, $U$ quella potenziale e $W$ il lavoro di tutte le forze agenti. Quale delle seguenti relazioni è vera?
  \begin{enumerate}[label=\Alph*.]
    \item $\Delta E_m=W$.
    \item $\Delta U=W$.
    \item $\Delta K=W.$
    \item $\Delta U=W$.
  \end{enumerate}
  \item L'energia meccanica di un corpo che si trova a un'altezza $h$
  \begin{enumerate}[label=\Alph*.]
    \item è $mgh$.
    \item dipende dalla velocità del corpo.
    \item è nulla.
    \item è negativa
  \end{enumerate}
  \item Il lavoro della forza peso
  \begin{enumerate}[label=\Alph*.]
    \item è nullo solo se il percorso è chiuso.
    \item è nullo se il percorso ha gli estremi posti alla stessa altitudine.
    \item è sempre negativo.
    \item è sempre positivo.
  \end{enumerate}
\end{enumerate}








\newpage \maketitle \centering \textbf{Griglia di valutazione}. Eventuali mezzi punti saranno arrotondati all'intero precedente. \begin{table}[h]     \centering \begin{tabular}{|c|c|c|c|c|c|c|c|c|c|c|c|c|c|c|c|c|c|c|c|} \hline Punti &  $\leq 4$ & 5 & 6 & 7 & 8 & 9 & 10 & 11 & \textbf{12} & 13 & 14 & 15 & 16 & 17 & 18 & 19 & 20 \\ \hline Voto & 2 & 2.5 & 3 & 3.5 & 4 & 4.5 & 5 & 5.5 & \textbf{6} & 6.5 & 7 & 7.5 & 8 & 8.5 & 9 & 9.5 & 10 \\ \hline \end{tabular} \end{table}
\textbf{Codice}: pulvbvhczjpgrwaiksego


\begin{enumerate}
  \item Un corpo, partendo da fermo, rotola giù dalla cima di un piano inclinato di un angolo $\alpha$, alto $h$. Quale è la sua velocità quando arriva in fondo?
  \begin{enumerate}[label=\Alph*.]
    \item $\sqrt{2gh}$.
    \item $\sqrt{2gh}\sin\alpha$.
    \item $\sqrt{mgh\cos\alpha}$.
    \item $\sqrt{2gh\sin\alpha}$.
  \end{enumerate}
  \item Il lavoro della reazione vincolare
  \begin{enumerate}[label=\Alph*.]
    \item è nullo solo se il percorso è chiuso.
    \item è sempre diverso da zero.
    \item può assumere qualsiasi valore.
    \item è sempre nullo.
  \end{enumerate}
  \item Un corpo viene lasciato cadere da fermo da un altezza $h$. Quando raggiunge il suolo ha una velocità $v$. Quale relazione esprime correttamente la conservazione dell'energia?
  \begin{enumerate}[label=\Alph*.]
    \item $mgh+\frac{1}{2}mv^2=\frac{1}{2}mv^2.$
    \item $\frac{1}{2}mh^2=\frac{1}{2}mv^2$.
    \item $\frac{1}{2}mv^2+mgh=0.$
    \item $mgh=\frac{1}{2}mv^2$.
  \end{enumerate}
  \item L'energia cinetica di un corpo fermo che si trova a un'altezza $h$
  \begin{enumerate}[label=\Alph*.]
    \item è nulla.
    \item dipende da $h$
    \item è $mgh$.
    \item si conserva.
  \end{enumerate}
  \item Un corpo ha inizialmente una velocità $v$ e dopo un certo tempo si ferma. La variazione di energia cinetica è
  \begin{enumerate}[label=\Alph*.]
    \item è positiva.
    \item è negativa
    \item dipende dal tempo in cui si ferma.
    \item dipende dallo spazio percorso.
  \end{enumerate}
  \item Il lavoro della forza d'attrito
  \begin{enumerate}[label=\Alph*.]
    \item è nullo se il percorso è rettilineo.
    \item è sempre positivo.
    \item è sempre negativo.
    \item non può essere calcolato perché la forza non è conservativa.
  \end{enumerate}
  \item Un corpo di massa $m_1$ e un corpo di massa $m_2$ sono lanciati in contemporanea da una torre alta $h$. Quali sono le velocità dei due corpi appena prima di toccare terra?
  \begin{enumerate}[label=\Alph*.]
    \item $v_1=\sqrt{2m_1gh}, v_2=\sqrt{2m_2gh}$.
    \item v_1=0, v_2=0
    \item $v_1=\sqrt{\frac{2gh}{m_1}}, v_2=\sqrt{\frac{2gh}{m_2}}$.
    \item $v_1=\sqrt{2gh}, v_2=\sqrt{2gh}$.
  \end{enumerate}
  \item Sia \vec{F} una forza conservativa. Allora
  \begin{enumerate}[label=\Alph*.]
    \item il lavoro di \vec{F} non dipende dal percorso considerato.
    \item il lavoro di \vec{F} è nullo solo se il percorso è chiuso.
    \item il lavoro di \vec{F} è sempre nullo.
    \item il lavoro di \vec{F} può essere nullo anche se il percorso non è chiuso.
  \end{enumerate}
  \item L'energia meccanica di un corpo che si trova a un'altezza $h$
  \begin{enumerate}[label=\Alph*.]
    \item è negativa
    \item è nulla.
    \item è $mgh$.
    \item dipende dalla velocità del corpo.
  \end{enumerate}
  \item Sia \vec{F} una forza non conservativa. Allora
  \begin{enumerate}[label=\Alph*.]
    \item il lavoro non dipende dagli estremi del percorso.
    \item esiste un percorso chiuso il cui lavoro non è nullo.
    \item il lavoro non dipende dal percorso.
    \item per ogni percorso aperto il lavoro è nullo.
  \end{enumerate}
  \item Un corpo impatta una molla con una velocità $v$. La molla di costante elastica $k$, opponendosi all'avanzare del corpo, lo ferma. Quale è il suo allungamento nel momento in cui si ferma, trascurando l'attrito?
  \begin{enumerate}[label=\Alph*.]
    \item $\sqrt{\frac{k}{m}}v$.
    \item $\sqrt{\frac{k}{m}}v$.
    \item $\sqrt{\frac{m}{k}}v$.
    \item $\sqrt{\frac{m}{k}v}$.
  \end{enumerate}
  \item Sia \vec{F} una forza generica, che agisce su un corpo che effettua uno spostamento $\Delta \vec{s}$. Allora
  \begin{enumerate}[label=\Alph*.]
    \item il suo lavoro ha la stessa direzione della forza.
    \item il suo lavoro ha la stessa direzione dello spostamento.
    \item il suo lavoro ha una direzione data dalla regola del parallelogramma.
    \item il suo lavoro non ha una direzione perché è scalare.
  \end{enumerate}
  \item Un corpo di massa $m$ scivola su un piano con attrito, partendo da una velocità v. Il coefficiente di attrito è \mu. Dopo quanto spazio si ferma?
  \begin{enumerate}[label=\Alph*.]
    \item $\frac{1}{2}v^2-\mu g$.
    \item $\frac{2v^2}{g\mu}}$.
    \item $\frac{1}{2}v^2+\mu g$.
    \item $\frac{v^2}{2g\mu}}$.
  \end{enumerate}
  \item Sia $E_m$ l'energia meccanica, $K$ l'energia cinetica, $U$ quella potenziale e $W$ il lavoro di tutte le forze agenti. Quale delle seguenti relazioni è vera?
  \begin{enumerate}[label=\Alph*.]
    \item $\Delta K=W.$
    \item $\Delta E_m=W$.
    \item $\Delta U=W$.
    \item $\Delta U=W$.
  \end{enumerate}
  \item Un corpo sale lungo un piano inclinato di un angolo $\alpha$ e alto $h$, partendo da terra e arrivando fino alla cima del piano. Il lavoro della forza peso è:
  \begin{enumerate}[label=\Alph*.]
    \item $mgh\sin\alpha$
    \item $-mgh$
    \item $mgh$
    \item $mgh\cos\alpha$
  \end{enumerate}
  \item Sia $E_m$ l'energia meccanica, $K$ l'energia cinetica, $U$ quella potenziale e $W_{NC}$ il lavoro delle forze non conservative agenti. Quale delle seguenti relazioni è vera?
  \begin{enumerate}[label=\Alph*.]
    \item $\Delta U=W_{NC}$.
    \item $\Delta K=W_{NC}.$
    \item $\Delta K = \Delta U$.
    \item $\Delta E_m=W_{NC}$.
  \end{enumerate}
  \item Un corpo sale da terra fino a un'altezza $h$. Il lavoro della forza peso è:
  \begin{enumerate}[label=\Alph*.]
    \item nullo
    \item positivo.
    \item negativo.
    \item $mgh$
  \end{enumerate}
  \item Sia \vec{F} una forza non conservativa. Allora
  \begin{enumerate}[label=\Alph*.]
    \item la sua energia potenziale è sempre negativa.
    \item la sua energia potenziale non esiste.
    \item la sua energia potenziale è tale che lungo un percorso $\Delta U=W$.
    \item la sua energia potenziale è tale che lungo un percorso $\Delta U=-W$.
  \end{enumerate}
  \item Il lavoro della forza peso
  \begin{enumerate}[label=\Alph*.]
    \item è nullo se il percorso ha gli estremi posti alla stessa altitudine.
    \item è sempre negativo.
    \item è sempre positivo.
    \item è nullo solo se il percorso è chiuso.
  \end{enumerate}
  \item Se calcoliamo il lavoro di una forza lungo un percorso chiuso e troviamo zero, cosa possiamo concludere?
  \begin{enumerate}[label=\Alph*.]
    \item la forza è nulla.
    \item niente.
    \item la forza è conservativa.
    \item la forza non è conservativa.
  \end{enumerate}
  \item Un corpo sale da terra fino a un'altezza $h$. Il lavoro della forza peso è:
  \begin{enumerate}[label=\Alph*.]
    \item negativo.
    \item positivo.
    \item nullo
    \item $mgh$
  \end{enumerate}
\end{enumerate}








\newpage \maketitle \centering \textbf{Griglia di valutazione}. Eventuali mezzi punti saranno arrotondati all'intero precedente. \begin{table}[h]     \centering \begin{tabular}{|c|c|c|c|c|c|c|c|c|c|c|c|c|c|c|c|c|c|c|c|} \hline Punti &  $\leq 4$ & 5 & 6 & 7 & 8 & 9 & 10 & 11 & \textbf{12} & 13 & 14 & 15 & 16 & 17 & 18 & 19 & 20 \\ \hline Voto & 2 & 2.5 & 3 & 3.5 & 4 & 4.5 & 5 & 5.5 & \textbf{6} & 6.5 & 7 & 7.5 & 8 & 8.5 & 9 & 9.5 & 10 \\ \hline \end{tabular} \end{table}
\textbf{Codice}: rrkycuhayipfryagltfgr


\begin{enumerate}
  \item L'energia meccanica di un corpo che si trova a un'altezza $h$
  \begin{enumerate}[label=\Alph*.]
    \item è negativa
    \item è $mgh$.
    \item dipende dalla velocità del corpo.
    \item è nulla.
  \end{enumerate}
  \item Un corpo sale da terra fino a un'altezza $h$. Il lavoro della forza peso è:
  \begin{enumerate}[label=\Alph*.]
    \item negativo.
    \item nullo
    \item positivo.
    \item $mgh$
  \end{enumerate}
  \item Sia \vec{F} una forza generica, che agisce su un corpo che effettua uno spostamento $\Delta \vec{s}$. Allora
  \begin{enumerate}[label=\Alph*.]
    \item il suo lavoro ha una direzione data dalla regola del parallelogramma.
    \item il suo lavoro ha la stessa direzione dello spostamento.
    \item il suo lavoro non ha una direzione perché è scalare.
    \item il suo lavoro ha la stessa direzione della forza.
  \end{enumerate}
  \item Il lavoro della forza peso
  \begin{enumerate}[label=\Alph*.]
    \item è nullo solo se il percorso è chiuso.
    \item è sempre negativo.
    \item è sempre positivo.
    \item è nullo se il percorso ha gli estremi posti alla stessa altitudine.
  \end{enumerate}
  \item Se calcoliamo il lavoro di una forza lungo un percorso chiuso e troviamo zero, cosa possiamo concludere?
  \begin{enumerate}[label=\Alph*.]
    \item la forza è nulla.
    \item la forza non è conservativa.
    \item niente.
    \item la forza è conservativa.
  \end{enumerate}
  \item Il lavoro della forza d'attrito
  \begin{enumerate}[label=\Alph*.]
    \item è sempre positivo.
    \item è sempre negativo.
    \item è nullo se il percorso è rettilineo.
    \item non può essere calcolato perché la forza non è conservativa.
  \end{enumerate}
  \item Un corpo ha inizialmente una velocità $v$ e dopo un certo tempo si ferma. La variazione di energia cinetica è
  \begin{enumerate}[label=\Alph*.]
    \item dipende dallo spazio percorso.
    \item è positiva.
    \item dipende dal tempo in cui si ferma.
    \item è negativa
  \end{enumerate}
  \item Un corpo, partendo da fermo, rotola giù dalla cima di un piano inclinato di un angolo $\alpha$, alto $h$. Quale è la sua velocità quando arriva in fondo?
  \begin{enumerate}[label=\Alph*.]
    \item $\sqrt{2gh}\sin\alpha$.
    \item $\sqrt{2gh}$.
    \item $\sqrt{2gh\sin\alpha}$.
    \item $\sqrt{mgh\cos\alpha}$.
  \end{enumerate}
  \item Un corpo impatta una molla con una velocità $v$. La molla di costante elastica $k$, opponendosi all'avanzare del corpo, lo ferma. Quale è il suo allungamento nel momento in cui si ferma, trascurando l'attrito?
  \begin{enumerate}[label=\Alph*.]
    \item $\sqrt{\frac{m}{k}v}$.
    \item $\sqrt{\frac{k}{m}}v$.
    \item $\sqrt{\frac{m}{k}}v$.
    \item $\sqrt{\frac{k}{m}}v$.
  \end{enumerate}
  \item Un corpo sale da terra fino a un'altezza $h$. Il lavoro della forza peso è:
  \begin{enumerate}[label=\Alph*.]
    \item negativo.
    \item nullo
    \item $mgh$
    \item positivo.
  \end{enumerate}
  \item Sia \vec{F} una forza non conservativa. Allora
  \begin{enumerate}[label=\Alph*.]
    \item la sua energia potenziale è tale che lungo un percorso $\Delta U=-W$.
    \item la sua energia potenziale è sempre negativa.
    \item la sua energia potenziale non esiste.
    \item la sua energia potenziale è tale che lungo un percorso $\Delta U=W$.
  \end{enumerate}
  \item L'energia cinetica di un corpo fermo che si trova a un'altezza $h$
  \begin{enumerate}[label=\Alph*.]
    \item dipende da $h$
    \item si conserva.
    \item è nulla.
    \item è $mgh$.
  \end{enumerate}
  \item Un corpo di massa $m_1$ e un corpo di massa $m_2$ sono lanciati in contemporanea da una torre alta $h$. Quali sono le velocità dei due corpi appena prima di toccare terra?
  \begin{enumerate}[label=\Alph*.]
    \item $v_1=\sqrt{2m_1gh}, v_2=\sqrt{2m_2gh}$.
    \item $v_1=\sqrt{\frac{2gh}{m_1}}, v_2=\sqrt{\frac{2gh}{m_2}}$.
    \item v_1=0, v_2=0
    \item $v_1=\sqrt{2gh}, v_2=\sqrt{2gh}$.
  \end{enumerate}
  \item Un corpo di massa $m$ scivola su un piano con attrito, partendo da una velocità v. Il coefficiente di attrito è \mu. Dopo quanto spazio si ferma?
  \begin{enumerate}[label=\Alph*.]
    \item $\frac{2v^2}{g\mu}}$.
    \item $\frac{1}{2}v^2+\mu g$.
    \item $\frac{v^2}{2g\mu}}$.
    \item $\frac{1}{2}v^2-\mu g$.
  \end{enumerate}
  \item Sia $E_m$ l'energia meccanica, $K$ l'energia cinetica, $U$ quella potenziale e $W_{NC}$ il lavoro delle forze non conservative agenti. Quale delle seguenti relazioni è vera?
  \begin{enumerate}[label=\Alph*.]
    \item $\Delta K = \Delta U$.
    \item $\Delta E_m=W_{NC}$.
    \item $\Delta U=W_{NC}$.
    \item $\Delta K=W_{NC}.$
  \end{enumerate}
  \item Sia \vec{F} una forza non conservativa. Allora
  \begin{enumerate}[label=\Alph*.]
    \item il lavoro non dipende dal percorso.
    \item esiste un percorso chiuso il cui lavoro non è nullo.
    \item il lavoro non dipende dagli estremi del percorso.
    \item per ogni percorso aperto il lavoro è nullo.
  \end{enumerate}
  \item Sia $E_m$ l'energia meccanica, $K$ l'energia cinetica, $U$ quella potenziale e $W$ il lavoro di tutte le forze agenti. Quale delle seguenti relazioni è vera?
  \begin{enumerate}[label=\Alph*.]
    \item $\Delta U=W$.
    \item $\Delta E_m=W$.
    \item $\Delta U=W$.
    \item $\Delta K=W.$
  \end{enumerate}
  \item Sia \vec{F} una forza conservativa. Allora
  \begin{enumerate}[label=\Alph*.]
    \item il lavoro di \vec{F} è nullo solo se il percorso è chiuso.
    \item il lavoro di \vec{F} non dipende dal percorso considerato.
    \item il lavoro di \vec{F} può essere nullo anche se il percorso non è chiuso.
    \item il lavoro di \vec{F} è sempre nullo.
  \end{enumerate}
  \item Il lavoro della reazione vincolare
  \begin{enumerate}[label=\Alph*.]
    \item può assumere qualsiasi valore.
    \item è sempre nullo.
    \item è nullo solo se il percorso è chiuso.
    \item è sempre diverso da zero.
  \end{enumerate}
  \item Un corpo viene lasciato cadere da fermo da un altezza $h$. Quando raggiunge il suolo ha una velocità $v$. Quale relazione esprime correttamente la conservazione dell'energia?
  \begin{enumerate}[label=\Alph*.]
    \item $mgh+\frac{1}{2}mv^2=\frac{1}{2}mv^2.$
    \item $mgh=\frac{1}{2}mv^2$.
    \item $\frac{1}{2}mv^2+mgh=0.$
    \item $\frac{1}{2}mh^2=\frac{1}{2}mv^2$.
  \end{enumerate}
  \item Un corpo sale lungo un piano inclinato di un angolo $\alpha$ e alto $h$, partendo da terra e arrivando fino alla cima del piano. Il lavoro della forza peso è:
  \begin{enumerate}[label=\Alph*.]
    \item $mgh\cos\alpha$
    \item $mgh\sin\alpha$
    \item $mgh$
    \item $-mgh$
  \end{enumerate}
\end{enumerate}








\newpage \maketitle \centering \textbf{Griglia di valutazione}. Eventuali mezzi punti saranno arrotondati all'intero precedente. \begin{table}[h]     \centering \begin{tabular}{|c|c|c|c|c|c|c|c|c|c|c|c|c|c|c|c|c|c|c|c|} \hline Punti &  $\leq 4$ & 5 & 6 & 7 & 8 & 9 & 10 & 11 & \textbf{12} & 13 & 14 & 15 & 16 & 17 & 18 & 19 & 20 \\ \hline Voto & 2 & 2.5 & 3 & 3.5 & 4 & 4.5 & 5 & 5.5 & \textbf{6} & 6.5 & 7 & 7.5 & 8 & 8.5 & 9 & 9.5 & 10 \\ \hline \end{tabular} \end{table}
\textbf{Codice}: psjxdufczingqwbijtfhr


\begin{enumerate}
  \item Un corpo di massa $m$ scivola su un piano con attrito, partendo da una velocità v. Il coefficiente di attrito è \mu. Dopo quanto spazio si ferma?
  \begin{enumerate}[label=\Alph*.]
    \item $\frac{v^2}{2g\mu}}$.
    \item $\frac{2v^2}{g\mu}}$.
    \item $\frac{1}{2}v^2-\mu g$.
    \item $\frac{1}{2}v^2+\mu g$.
  \end{enumerate}
  \item Un corpo impatta una molla con una velocità $v$. La molla di costante elastica $k$, opponendosi all'avanzare del corpo, lo ferma. Quale è il suo allungamento nel momento in cui si ferma, trascurando l'attrito?
  \begin{enumerate}[label=\Alph*.]
    \item $\sqrt{\frac{k}{m}}v$.
    \item $\sqrt{\frac{m}{k}}v$.
    \item $\sqrt{\frac{m}{k}v}$.
    \item $\sqrt{\frac{k}{m}}v$.
  \end{enumerate}
  \item Un corpo viene lasciato cadere da fermo da un altezza $h$. Quando raggiunge il suolo ha una velocità $v$. Quale relazione esprime correttamente la conservazione dell'energia?
  \begin{enumerate}[label=\Alph*.]
    \item $mgh+\frac{1}{2}mv^2=\frac{1}{2}mv^2.$
    \item $mgh=\frac{1}{2}mv^2$.
    \item $\frac{1}{2}mh^2=\frac{1}{2}mv^2$.
    \item $\frac{1}{2}mv^2+mgh=0.$
  \end{enumerate}
  \item Sia $E_m$ l'energia meccanica, $K$ l'energia cinetica, $U$ quella potenziale e $W$ il lavoro di tutte le forze agenti. Quale delle seguenti relazioni è vera?
  \begin{enumerate}[label=\Alph*.]
    \item $\Delta U=W$.
    \item $\Delta U=W$.
    \item $\Delta K=W.$
    \item $\Delta E_m=W$.
  \end{enumerate}
  \item Sia \vec{F} una forza generica, che agisce su un corpo che effettua uno spostamento $\Delta \vec{s}$. Allora
  \begin{enumerate}[label=\Alph*.]
    \item il suo lavoro ha una direzione data dalla regola del parallelogramma.
    \item il suo lavoro ha la stessa direzione dello spostamento.
    \item il suo lavoro ha la stessa direzione della forza.
    \item il suo lavoro non ha una direzione perché è scalare.
  \end{enumerate}
  \item Il lavoro della forza d'attrito
  \begin{enumerate}[label=\Alph*.]
    \item è nullo se il percorso è rettilineo.
    \item è sempre negativo.
    \item non può essere calcolato perché la forza non è conservativa.
    \item è sempre positivo.
  \end{enumerate}
  \item L'energia cinetica di un corpo fermo che si trova a un'altezza $h$
  \begin{enumerate}[label=\Alph*.]
    \item è $mgh$.
    \item è nulla.
    \item dipende da $h$
    \item si conserva.
  \end{enumerate}
  \item Un corpo, partendo da fermo, rotola giù dalla cima di un piano inclinato di un angolo $\alpha$, alto $h$. Quale è la sua velocità quando arriva in fondo?
  \begin{enumerate}[label=\Alph*.]
    \item $\sqrt{2gh\sin\alpha}$.
    \item $\sqrt{2gh}\sin\alpha$.
    \item $\sqrt{mgh\cos\alpha}$.
    \item $\sqrt{2gh}$.
  \end{enumerate}
  \item Un corpo sale lungo un piano inclinato di un angolo $\alpha$ e alto $h$, partendo da terra e arrivando fino alla cima del piano. Il lavoro della forza peso è:
  \begin{enumerate}[label=\Alph*.]
    \item $mgh$
    \item $mgh\cos\alpha$
    \item $mgh\sin\alpha$
    \item $-mgh$
  \end{enumerate}
  \item Sia \vec{F} una forza conservativa. Allora
  \begin{enumerate}[label=\Alph*.]
    \item il lavoro di \vec{F} può essere nullo anche se il percorso non è chiuso.
    \item il lavoro di \vec{F} non dipende dal percorso considerato.
    \item il lavoro di \vec{F} è nullo solo se il percorso è chiuso.
    \item il lavoro di \vec{F} è sempre nullo.
  \end{enumerate}
  \item Un corpo di massa $m_1$ e un corpo di massa $m_2$ sono lanciati in contemporanea da una torre alta $h$. Quali sono le velocità dei due corpi appena prima di toccare terra?
  \begin{enumerate}[label=\Alph*.]
    \item $v_1=\sqrt{2gh}, v_2=\sqrt{2gh}$.
    \item $v_1=\sqrt{\frac{2gh}{m_1}}, v_2=\sqrt{\frac{2gh}{m_2}}$.
    \item v_1=0, v_2=0
    \item $v_1=\sqrt{2m_1gh}, v_2=\sqrt{2m_2gh}$.
  \end{enumerate}
  \item Il lavoro della reazione vincolare
  \begin{enumerate}[label=\Alph*.]
    \item è nullo solo se il percorso è chiuso.
    \item è sempre diverso da zero.
    \item può assumere qualsiasi valore.
    \item è sempre nullo.
  \end{enumerate}
  \item L'energia meccanica di un corpo che si trova a un'altezza $h$
  \begin{enumerate}[label=\Alph*.]
    \item è negativa
    \item è $mgh$.
    \item dipende dalla velocità del corpo.
    \item è nulla.
  \end{enumerate}
  \item Il lavoro della forza peso
  \begin{enumerate}[label=\Alph*.]
    \item è nullo se il percorso ha gli estremi posti alla stessa altitudine.
    \item è sempre negativo.
    \item è nullo solo se il percorso è chiuso.
    \item è sempre positivo.
  \end{enumerate}
  \item Sia \vec{F} una forza non conservativa. Allora
  \begin{enumerate}[label=\Alph*.]
    \item la sua energia potenziale è tale che lungo un percorso $\Delta U=-W$.
    \item la sua energia potenziale è tale che lungo un percorso $\Delta U=W$.
    \item la sua energia potenziale non esiste.
    \item la sua energia potenziale è sempre negativa.
  \end{enumerate}
  \item Un corpo sale da terra fino a un'altezza $h$. Il lavoro della forza peso è:
  \begin{enumerate}[label=\Alph*.]
    \item nullo
    \item positivo.
    \item $mgh$
    \item negativo.
  \end{enumerate}
  \item Un corpo sale da terra fino a un'altezza $h$. Il lavoro della forza peso è:
  \begin{enumerate}[label=\Alph*.]
    \item $mgh$
    \item negativo.
    \item positivo.
    \item nullo
  \end{enumerate}
  \item Se calcoliamo il lavoro di una forza lungo un percorso chiuso e troviamo zero, cosa possiamo concludere?
  \begin{enumerate}[label=\Alph*.]
    \item la forza è conservativa.
    \item la forza è nulla.
    \item niente.
    \item la forza non è conservativa.
  \end{enumerate}
  \item Un corpo ha inizialmente una velocità $v$ e dopo un certo tempo si ferma. La variazione di energia cinetica è
  \begin{enumerate}[label=\Alph*.]
    \item dipende dal tempo in cui si ferma.
    \item è negativa
    \item dipende dallo spazio percorso.
    \item è positiva.
  \end{enumerate}
  \item Sia $E_m$ l'energia meccanica, $K$ l'energia cinetica, $U$ quella potenziale e $W_{NC}$ il lavoro delle forze non conservative agenti. Quale delle seguenti relazioni è vera?
  \begin{enumerate}[label=\Alph*.]
    \item $\Delta U=W_{NC}$.
    \item $\Delta K = \Delta U$.
    \item $\Delta E_m=W_{NC}$.
    \item $\Delta K=W_{NC}.$
  \end{enumerate}
  \item Sia \vec{F} una forza non conservativa. Allora
  \begin{enumerate}[label=\Alph*.]
    \item per ogni percorso aperto il lavoro è nullo.
    \item il lavoro non dipende dal percorso.
    \item il lavoro non dipende dagli estremi del percorso.
    \item esiste un percorso chiuso il cui lavoro non è nullo.
  \end{enumerate}
\end{enumerate}








\newpage \maketitle \centering \textbf{Griglia di valutazione}. Eventuali mezzi punti saranno arrotondati all'intero precedente. \begin{table}[h]     \centering \begin{tabular}{|c|c|c|c|c|c|c|c|c|c|c|c|c|c|c|c|c|c|c|c|} \hline Punti &  $\leq 4$ & 5 & 6 & 7 & 8 & 9 & 10 & 11 & \textbf{12} & 13 & 14 & 15 & 16 & 17 & 18 & 19 & 20 \\ \hline Voto & 2 & 2.5 & 3 & 3.5 & 4 & 4.5 & 5 & 5.5 & \textbf{6} & 6.5 & 7 & 7.5 & 8 & 8.5 & 9 & 9.5 & 10 \\ \hline \end{tabular} \end{table}
\textbf{Codice}: puixatebxlpdoxchkshfp


\begin{enumerate}
  \item Sia \vec{F} una forza non conservativa. Allora
  \begin{enumerate}[label=\Alph*.]
    \item esiste un percorso chiuso il cui lavoro non è nullo.
    \item il lavoro non dipende dal percorso.
    \item il lavoro non dipende dagli estremi del percorso.
    \item per ogni percorso aperto il lavoro è nullo.
  \end{enumerate}
  \item Un corpo di massa $m$ scivola su un piano con attrito, partendo da una velocità v. Il coefficiente di attrito è \mu. Dopo quanto spazio si ferma?
  \begin{enumerate}[label=\Alph*.]
    \item $\frac{1}{2}v^2-\mu g$.
    \item $\frac{2v^2}{g\mu}}$.
    \item $\frac{1}{2}v^2+\mu g$.
    \item $\frac{v^2}{2g\mu}}$.
  \end{enumerate}
  \item Il lavoro della reazione vincolare
  \begin{enumerate}[label=\Alph*.]
    \item è sempre nullo.
    \item è nullo solo se il percorso è chiuso.
    \item può assumere qualsiasi valore.
    \item è sempre diverso da zero.
  \end{enumerate}
  \item Un corpo sale da terra fino a un'altezza $h$. Il lavoro della forza peso è:
  \begin{enumerate}[label=\Alph*.]
    \item positivo.
    \item nullo
    \item negativo.
    \item $mgh$
  \end{enumerate}
  \item Sia $E_m$ l'energia meccanica, $K$ l'energia cinetica, $U$ quella potenziale e $W_{NC}$ il lavoro delle forze non conservative agenti. Quale delle seguenti relazioni è vera?
  \begin{enumerate}[label=\Alph*.]
    \item $\Delta E_m=W_{NC}$.
    \item $\Delta U=W_{NC}$.
    \item $\Delta K = \Delta U$.
    \item $\Delta K=W_{NC}.$
  \end{enumerate}
  \item Un corpo, partendo da fermo, rotola giù dalla cima di un piano inclinato di un angolo $\alpha$, alto $h$. Quale è la sua velocità quando arriva in fondo?
  \begin{enumerate}[label=\Alph*.]
    \item $\sqrt{2gh}$.
    \item $\sqrt{mgh\cos\alpha}$.
    \item $\sqrt{2gh\sin\alpha}$.
    \item $\sqrt{2gh}\sin\alpha$.
  \end{enumerate}
  \item Un corpo ha inizialmente una velocità $v$ e dopo un certo tempo si ferma. La variazione di energia cinetica è
  \begin{enumerate}[label=\Alph*.]
    \item è negativa
    \item dipende dallo spazio percorso.
    \item dipende dal tempo in cui si ferma.
    \item è positiva.
  \end{enumerate}
  \item Il lavoro della forza d'attrito
  \begin{enumerate}[label=\Alph*.]
    \item è nullo se il percorso è rettilineo.
    \item è sempre positivo.
    \item è sempre negativo.
    \item non può essere calcolato perché la forza non è conservativa.
  \end{enumerate}
  \item Sia \vec{F} una forza non conservativa. Allora
  \begin{enumerate}[label=\Alph*.]
    \item la sua energia potenziale è tale che lungo un percorso $\Delta U=-W$.
    \item la sua energia potenziale non esiste.
    \item la sua energia potenziale è sempre negativa.
    \item la sua energia potenziale è tale che lungo un percorso $\Delta U=W$.
  \end{enumerate}
  \item Se calcoliamo il lavoro di una forza lungo un percorso chiuso e troviamo zero, cosa possiamo concludere?
  \begin{enumerate}[label=\Alph*.]
    \item la forza è conservativa.
    \item la forza è nulla.
    \item la forza non è conservativa.
    \item niente.
  \end{enumerate}
  \item Un corpo sale da terra fino a un'altezza $h$. Il lavoro della forza peso è:
  \begin{enumerate}[label=\Alph*.]
    \item positivo.
    \item nullo
    \item negativo.
    \item $mgh$
  \end{enumerate}
  \item Il lavoro della forza peso
  \begin{enumerate}[label=\Alph*.]
    \item è nullo se il percorso ha gli estremi posti alla stessa altitudine.
    \item è nullo solo se il percorso è chiuso.
    \item è sempre negativo.
    \item è sempre positivo.
  \end{enumerate}
  \item Sia $E_m$ l'energia meccanica, $K$ l'energia cinetica, $U$ quella potenziale e $W$ il lavoro di tutte le forze agenti. Quale delle seguenti relazioni è vera?
  \begin{enumerate}[label=\Alph*.]
    \item $\Delta K=W.$
    \item $\Delta E_m=W$.
    \item $\Delta U=W$.
    \item $\Delta U=W$.
  \end{enumerate}
  \item Un corpo impatta una molla con una velocità $v$. La molla di costante elastica $k$, opponendosi all'avanzare del corpo, lo ferma. Quale è il suo allungamento nel momento in cui si ferma, trascurando l'attrito?
  \begin{enumerate}[label=\Alph*.]
    \item $\sqrt{\frac{k}{m}}v$.
    \item $\sqrt{\frac{m}{k}}v$.
    \item $\sqrt{\frac{k}{m}}v$.
    \item $\sqrt{\frac{m}{k}v}$.
  \end{enumerate}
  \item Sia \vec{F} una forza conservativa. Allora
  \begin{enumerate}[label=\Alph*.]
    \item il lavoro di \vec{F} è sempre nullo.
    \item il lavoro di \vec{F} non dipende dal percorso considerato.
    \item il lavoro di \vec{F} è nullo solo se il percorso è chiuso.
    \item il lavoro di \vec{F} può essere nullo anche se il percorso non è chiuso.
  \end{enumerate}
  \item Un corpo sale lungo un piano inclinato di un angolo $\alpha$ e alto $h$, partendo da terra e arrivando fino alla cima del piano. Il lavoro della forza peso è:
  \begin{enumerate}[label=\Alph*.]
    \item $mgh\sin\alpha$
    \item $mgh\cos\alpha$
    \item $-mgh$
    \item $mgh$
  \end{enumerate}
  \item Sia \vec{F} una forza generica, che agisce su un corpo che effettua uno spostamento $\Delta \vec{s}$. Allora
  \begin{enumerate}[label=\Alph*.]
    \item il suo lavoro ha la stessa direzione della forza.
    \item il suo lavoro ha una direzione data dalla regola del parallelogramma.
    \item il suo lavoro non ha una direzione perché è scalare.
    \item il suo lavoro ha la stessa direzione dello spostamento.
  \end{enumerate}
  \item L'energia cinetica di un corpo fermo che si trova a un'altezza $h$
  \begin{enumerate}[label=\Alph*.]
    \item si conserva.
    \item è nulla.
    \item dipende da $h$
    \item è $mgh$.
  \end{enumerate}
  \item Un corpo viene lasciato cadere da fermo da un altezza $h$. Quando raggiunge il suolo ha una velocità $v$. Quale relazione esprime correttamente la conservazione dell'energia?
  \begin{enumerate}[label=\Alph*.]
    \item $mgh+\frac{1}{2}mv^2=\frac{1}{2}mv^2.$
    \item $\frac{1}{2}mh^2=\frac{1}{2}mv^2$.
    \item $\frac{1}{2}mv^2+mgh=0.$
    \item $mgh=\frac{1}{2}mv^2$.
  \end{enumerate}
  \item L'energia meccanica di un corpo che si trova a un'altezza $h$
  \begin{enumerate}[label=\Alph*.]
    \item dipende dalla velocità del corpo.
    \item è nulla.
    \item è $mgh$.
    \item è negativa
  \end{enumerate}
  \item Un corpo di massa $m_1$ e un corpo di massa $m_2$ sono lanciati in contemporanea da una torre alta $h$. Quali sono le velocità dei due corpi appena prima di toccare terra?
  \begin{enumerate}[label=\Alph*.]
    \item $v_1=\sqrt{2m_1gh}, v_2=\sqrt{2m_2gh}$.
    \item $v_1=\sqrt{2gh}, v_2=\sqrt{2gh}$.
    \item $v_1=\sqrt{\frac{2gh}{m_1}}, v_2=\sqrt{\frac{2gh}{m_2}}$.
    \item v_1=0, v_2=0
  \end{enumerate}
\end{enumerate}








\newpage \maketitle \centering \textbf{Griglia di valutazione}. Eventuali mezzi punti saranno arrotondati all'intero precedente. \begin{table}[h]     \centering \begin{tabular}{|c|c|c|c|c|c|c|c|c|c|c|c|c|c|c|c|c|c|c|c|} \hline Punti &  $\leq 4$ & 5 & 6 & 7 & 8 & 9 & 10 & 11 & \textbf{12} & 13 & 14 & 15 & 16 & 17 & 18 & 19 & 20 \\ \hline Voto & 2 & 2.5 & 3 & 3.5 & 4 & 4.5 & 5 & 5.5 & \textbf{6} & 6.5 & 7 & 7.5 & 8 & 8.5 & 9 & 9.5 & 10 \\ \hline \end{tabular} \end{table}
\textbf{Codice}: prixbvgbxlodpyaijseir


\begin{enumerate}
  \item Sia \vec{F} una forza non conservativa. Allora
  \begin{enumerate}[label=\Alph*.]
    \item la sua energia potenziale non esiste.
    \item la sua energia potenziale è tale che lungo un percorso $\Delta U=-W$.
    \item la sua energia potenziale è sempre negativa.
    \item la sua energia potenziale è tale che lungo un percorso $\Delta U=W$.
  \end{enumerate}
  \item Sia \vec{F} una forza generica, che agisce su un corpo che effettua uno spostamento $\Delta \vec{s}$. Allora
  \begin{enumerate}[label=\Alph*.]
    \item il suo lavoro non ha una direzione perché è scalare.
    \item il suo lavoro ha la stessa direzione dello spostamento.
    \item il suo lavoro ha la stessa direzione della forza.
    \item il suo lavoro ha una direzione data dalla regola del parallelogramma.
  \end{enumerate}
  \item Il lavoro della reazione vincolare
  \begin{enumerate}[label=\Alph*.]
    \item è sempre nullo.
    \item è sempre diverso da zero.
    \item è nullo solo se il percorso è chiuso.
    \item può assumere qualsiasi valore.
  \end{enumerate}
  \item Sia $E_m$ l'energia meccanica, $K$ l'energia cinetica, $U$ quella potenziale e $W_{NC}$ il lavoro delle forze non conservative agenti. Quale delle seguenti relazioni è vera?
  \begin{enumerate}[label=\Alph*.]
    \item $\Delta K=W_{NC}.$
    \item $\Delta U=W_{NC}$.
    \item $\Delta E_m=W_{NC}$.
    \item $\Delta K = \Delta U$.
  \end{enumerate}
  \item Un corpo, partendo da fermo, rotola giù dalla cima di un piano inclinato di un angolo $\alpha$, alto $h$. Quale è la sua velocità quando arriva in fondo?
  \begin{enumerate}[label=\Alph*.]
    \item $\sqrt{2gh\sin\alpha}$.
    \item $\sqrt{2gh}$.
    \item $\sqrt{mgh\cos\alpha}$.
    \item $\sqrt{2gh}\sin\alpha$.
  \end{enumerate}
  \item Un corpo ha inizialmente una velocità $v$ e dopo un certo tempo si ferma. La variazione di energia cinetica è
  \begin{enumerate}[label=\Alph*.]
    \item dipende dal tempo in cui si ferma.
    \item dipende dallo spazio percorso.
    \item è negativa
    \item è positiva.
  \end{enumerate}
  \item Un corpo viene lasciato cadere da fermo da un altezza $h$. Quando raggiunge il suolo ha una velocità $v$. Quale relazione esprime correttamente la conservazione dell'energia?
  \begin{enumerate}[label=\Alph*.]
    \item $\frac{1}{2}mh^2=\frac{1}{2}mv^2$.
    \item $\frac{1}{2}mv^2+mgh=0.$
    \item $mgh=\frac{1}{2}mv^2$.
    \item $mgh+\frac{1}{2}mv^2=\frac{1}{2}mv^2.$
  \end{enumerate}
  \item L'energia meccanica di un corpo che si trova a un'altezza $h$
  \begin{enumerate}[label=\Alph*.]
    \item è $mgh$.
    \item è nulla.
    \item dipende dalla velocità del corpo.
    \item è negativa
  \end{enumerate}
  \item Un corpo impatta una molla con una velocità $v$. La molla di costante elastica $k$, opponendosi all'avanzare del corpo, lo ferma. Quale è il suo allungamento nel momento in cui si ferma, trascurando l'attrito?
  \begin{enumerate}[label=\Alph*.]
    \item $\sqrt{\frac{m}{k}v}$.
    \item $\sqrt{\frac{m}{k}}v$.
    \item $\sqrt{\frac{k}{m}}v$.
    \item $\sqrt{\frac{k}{m}}v$.
  \end{enumerate}
  \item Un corpo sale da terra fino a un'altezza $h$. Il lavoro della forza peso è:
  \begin{enumerate}[label=\Alph*.]
    \item nullo
    \item $mgh$
    \item positivo.
    \item negativo.
  \end{enumerate}
  \item Un corpo di massa $m$ scivola su un piano con attrito, partendo da una velocità v. Il coefficiente di attrito è \mu. Dopo quanto spazio si ferma?
  \begin{enumerate}[label=\Alph*.]
    \item $\frac{1}{2}v^2-\mu g$.
    \item $\frac{v^2}{2g\mu}}$.
    \item $\frac{1}{2}v^2+\mu g$.
    \item $\frac{2v^2}{g\mu}}$.
  \end{enumerate}
  \item L'energia cinetica di un corpo fermo che si trova a un'altezza $h$
  \begin{enumerate}[label=\Alph*.]
    \item è nulla.
    \item è $mgh$.
    \item dipende da $h$
    \item si conserva.
  \end{enumerate}
  \item Il lavoro della forza peso
  \begin{enumerate}[label=\Alph*.]
    \item è sempre negativo.
    \item è nullo se il percorso ha gli estremi posti alla stessa altitudine.
    \item è sempre positivo.
    \item è nullo solo se il percorso è chiuso.
  \end{enumerate}
  \item Se calcoliamo il lavoro di una forza lungo un percorso chiuso e troviamo zero, cosa possiamo concludere?
  \begin{enumerate}[label=\Alph*.]
    \item la forza è conservativa.
    \item la forza non è conservativa.
    \item niente.
    \item la forza è nulla.
  \end{enumerate}
  \item Un corpo sale lungo un piano inclinato di un angolo $\alpha$ e alto $h$, partendo da terra e arrivando fino alla cima del piano. Il lavoro della forza peso è:
  \begin{enumerate}[label=\Alph*.]
    \item $mgh\cos\alpha$
    \item $-mgh$
    \item $mgh$
    \item $mgh\sin\alpha$
  \end{enumerate}
  \item Sia $E_m$ l'energia meccanica, $K$ l'energia cinetica, $U$ quella potenziale e $W$ il lavoro di tutte le forze agenti. Quale delle seguenti relazioni è vera?
  \begin{enumerate}[label=\Alph*.]
    \item $\Delta U=W$.
    \item $\Delta U=W$.
    \item $\Delta E_m=W$.
    \item $\Delta K=W.$
  \end{enumerate}
  \item Sia \vec{F} una forza conservativa. Allora
  \begin{enumerate}[label=\Alph*.]
    \item il lavoro di \vec{F} non dipende dal percorso considerato.
    \item il lavoro di \vec{F} può essere nullo anche se il percorso non è chiuso.
    \item il lavoro di \vec{F} è nullo solo se il percorso è chiuso.
    \item il lavoro di \vec{F} è sempre nullo.
  \end{enumerate}
  \item Un corpo di massa $m_1$ e un corpo di massa $m_2$ sono lanciati in contemporanea da una torre alta $h$. Quali sono le velocità dei due corpi appena prima di toccare terra?
  \begin{enumerate}[label=\Alph*.]
    \item $v_1=\sqrt{2m_1gh}, v_2=\sqrt{2m_2gh}$.
    \item $v_1=\sqrt{2gh}, v_2=\sqrt{2gh}$.
    \item $v_1=\sqrt{\frac{2gh}{m_1}}, v_2=\sqrt{\frac{2gh}{m_2}}$.
    \item v_1=0, v_2=0
  \end{enumerate}
  \item Un corpo sale da terra fino a un'altezza $h$. Il lavoro della forza peso è:
  \begin{enumerate}[label=\Alph*.]
    \item negativo.
    \item $mgh$
    \item positivo.
    \item nullo
  \end{enumerate}
  \item Sia \vec{F} una forza non conservativa. Allora
  \begin{enumerate}[label=\Alph*.]
    \item il lavoro non dipende dal percorso.
    \item per ogni percorso aperto il lavoro è nullo.
    \item il lavoro non dipende dagli estremi del percorso.
    \item esiste un percorso chiuso il cui lavoro non è nullo.
  \end{enumerate}
  \item Il lavoro della forza d'attrito
  \begin{enumerate}[label=\Alph*.]
    \item è sempre positivo.
    \item non può essere calcolato perché la forza non è conservativa.
    \item è nullo se il percorso è rettilineo.
    \item è sempre negativo.
  \end{enumerate}
\end{enumerate}








\newpage \maketitle \centering \textbf{Griglia di valutazione}. Eventuali mezzi punti saranno arrotondati all'intero precedente. \begin{table}[h]     \centering \begin{tabular}{|c|c|c|c|c|c|c|c|c|c|c|c|c|c|c|c|c|c|c|c|} \hline Punti &  $\leq 4$ & 5 & 6 & 7 & 8 & 9 & 10 & 11 & \textbf{12} & 13 & 14 & 15 & 16 & 17 & 18 & 19 & 20 \\ \hline Voto & 2 & 2.5 & 3 & 3.5 & 4 & 4.5 & 5 & 5.5 & \textbf{6} & 6.5 & 7 & 7.5 & 8 & 8.5 & 9 & 9.5 & 10 \\ \hline \end{tabular} \end{table}
\textbf{Codice}: ptjwdvhbykofqxafirgfp


\begin{enumerate}
  \item Il lavoro della forza peso
  \begin{enumerate}[label=\Alph*.]
    \item è nullo se il percorso ha gli estremi posti alla stessa altitudine.
    \item è sempre negativo.
    \item è nullo solo se il percorso è chiuso.
    \item è sempre positivo.
  \end{enumerate}
  \item Un corpo, partendo da fermo, rotola giù dalla cima di un piano inclinato di un angolo $\alpha$, alto $h$. Quale è la sua velocità quando arriva in fondo?
  \begin{enumerate}[label=\Alph*.]
    \item $\sqrt{2gh\sin\alpha}$.
    \item $\sqrt{mgh\cos\alpha}$.
    \item $\sqrt{2gh}$.
    \item $\sqrt{2gh}\sin\alpha$.
  \end{enumerate}
  \item Un corpo impatta una molla con una velocità $v$. La molla di costante elastica $k$, opponendosi all'avanzare del corpo, lo ferma. Quale è il suo allungamento nel momento in cui si ferma, trascurando l'attrito?
  \begin{enumerate}[label=\Alph*.]
    \item $\sqrt{\frac{k}{m}}v$.
    \item $\sqrt{\frac{m}{k}}v$.
    \item $\sqrt{\frac{k}{m}}v$.
    \item $\sqrt{\frac{m}{k}v}$.
  \end{enumerate}
  \item Sia \vec{F} una forza non conservativa. Allora
  \begin{enumerate}[label=\Alph*.]
    \item la sua energia potenziale è tale che lungo un percorso $\Delta U=-W$.
    \item la sua energia potenziale non esiste.
    \item la sua energia potenziale è sempre negativa.
    \item la sua energia potenziale è tale che lungo un percorso $\Delta U=W$.
  \end{enumerate}
  \item Un corpo ha inizialmente una velocità $v$ e dopo un certo tempo si ferma. La variazione di energia cinetica è
  \begin{enumerate}[label=\Alph*.]
    \item dipende dal tempo in cui si ferma.
    \item dipende dallo spazio percorso.
    \item è positiva.
    \item è negativa
  \end{enumerate}
  \item Sia $E_m$ l'energia meccanica, $K$ l'energia cinetica, $U$ quella potenziale e $W$ il lavoro di tutte le forze agenti. Quale delle seguenti relazioni è vera?
  \begin{enumerate}[label=\Alph*.]
    \item $\Delta U=W$.
    \item $\Delta E_m=W$.
    \item $\Delta K=W.$
    \item $\Delta U=W$.
  \end{enumerate}
  \item Il lavoro della forza d'attrito
  \begin{enumerate}[label=\Alph*.]
    \item non può essere calcolato perché la forza non è conservativa.
    \item è nullo se il percorso è rettilineo.
    \item è sempre positivo.
    \item è sempre negativo.
  \end{enumerate}
  \item Sia \vec{F} una forza non conservativa. Allora
  \begin{enumerate}[label=\Alph*.]
    \item il lavoro non dipende dagli estremi del percorso.
    \item per ogni percorso aperto il lavoro è nullo.
    \item esiste un percorso chiuso il cui lavoro non è nullo.
    \item il lavoro non dipende dal percorso.
  \end{enumerate}
  \item Un corpo sale lungo un piano inclinato di un angolo $\alpha$ e alto $h$, partendo da terra e arrivando fino alla cima del piano. Il lavoro della forza peso è:
  \begin{enumerate}[label=\Alph*.]
    \item $mgh\sin\alpha$
    \item $mgh\cos\alpha$
    \item $-mgh$
    \item $mgh$
  \end{enumerate}
  \item Il lavoro della reazione vincolare
  \begin{enumerate}[label=\Alph*.]
    \item può assumere qualsiasi valore.
    \item è nullo solo se il percorso è chiuso.
    \item è sempre nullo.
    \item è sempre diverso da zero.
  \end{enumerate}
  \item Un corpo di massa $m$ scivola su un piano con attrito, partendo da una velocità v. Il coefficiente di attrito è \mu. Dopo quanto spazio si ferma?
  \begin{enumerate}[label=\Alph*.]
    \item $\frac{1}{2}v^2+\mu g$.
    \item $\frac{v^2}{2g\mu}}$.
    \item $\frac{2v^2}{g\mu}}$.
    \item $\frac{1}{2}v^2-\mu g$.
  \end{enumerate}
  \item Un corpo sale da terra fino a un'altezza $h$. Il lavoro della forza peso è:
  \begin{enumerate}[label=\Alph*.]
    \item $mgh$
    \item nullo
    \item negativo.
    \item positivo.
  \end{enumerate}
  \item Un corpo viene lasciato cadere da fermo da un altezza $h$. Quando raggiunge il suolo ha una velocità $v$. Quale relazione esprime correttamente la conservazione dell'energia?
  \begin{enumerate}[label=\Alph*.]
    \item $\frac{1}{2}mv^2+mgh=0.$
    \item $mgh+\frac{1}{2}mv^2=\frac{1}{2}mv^2.$
    \item $mgh=\frac{1}{2}mv^2$.
    \item $\frac{1}{2}mh^2=\frac{1}{2}mv^2$.
  \end{enumerate}
  \item Sia \vec{F} una forza conservativa. Allora
  \begin{enumerate}[label=\Alph*.]
    \item il lavoro di \vec{F} non dipende dal percorso considerato.
    \item il lavoro di \vec{F} può essere nullo anche se il percorso non è chiuso.
    \item il lavoro di \vec{F} è sempre nullo.
    \item il lavoro di \vec{F} è nullo solo se il percorso è chiuso.
  \end{enumerate}
  \item Sia $E_m$ l'energia meccanica, $K$ l'energia cinetica, $U$ quella potenziale e $W_{NC}$ il lavoro delle forze non conservative agenti. Quale delle seguenti relazioni è vera?
  \begin{enumerate}[label=\Alph*.]
    \item $\Delta U=W_{NC}$.
    \item $\Delta E_m=W_{NC}$.
    \item $\Delta K=W_{NC}.$
    \item $\Delta K = \Delta U$.
  \end{enumerate}
  \item Se calcoliamo il lavoro di una forza lungo un percorso chiuso e troviamo zero, cosa possiamo concludere?
  \begin{enumerate}[label=\Alph*.]
    \item niente.
    \item la forza è conservativa.
    \item la forza è nulla.
    \item la forza non è conservativa.
  \end{enumerate}
  \item Sia \vec{F} una forza generica, che agisce su un corpo che effettua uno spostamento $\Delta \vec{s}$. Allora
  \begin{enumerate}[label=\Alph*.]
    \item il suo lavoro non ha una direzione perché è scalare.
    \item il suo lavoro ha la stessa direzione dello spostamento.
    \item il suo lavoro ha una direzione data dalla regola del parallelogramma.
    \item il suo lavoro ha la stessa direzione della forza.
  \end{enumerate}
  \item L'energia cinetica di un corpo fermo che si trova a un'altezza $h$
  \begin{enumerate}[label=\Alph*.]
    \item è nulla.
    \item è $mgh$.
    \item dipende da $h$
    \item si conserva.
  \end{enumerate}
  \item L'energia meccanica di un corpo che si trova a un'altezza $h$
  \begin{enumerate}[label=\Alph*.]
    \item è negativa
    \item è $mgh$.
    \item dipende dalla velocità del corpo.
    \item è nulla.
  \end{enumerate}
  \item Un corpo sale da terra fino a un'altezza $h$. Il lavoro della forza peso è:
  \begin{enumerate}[label=\Alph*.]
    \item negativo.
    \item positivo.
    \item $mgh$
    \item nullo
  \end{enumerate}
  \item Un corpo di massa $m_1$ e un corpo di massa $m_2$ sono lanciati in contemporanea da una torre alta $h$. Quali sono le velocità dei due corpi appena prima di toccare terra?
  \begin{enumerate}[label=\Alph*.]
    \item $v_1=\sqrt{2m_1gh}, v_2=\sqrt{2m_2gh}$.
    \item $v_1=\sqrt{2gh}, v_2=\sqrt{2gh}$.
    \item $v_1=\sqrt{\frac{2gh}{m_1}}, v_2=\sqrt{\frac{2gh}{m_2}}$.
    \item v_1=0, v_2=0
  \end{enumerate}
\end{enumerate}








\newpage \maketitle \centering \textbf{Griglia di valutazione}. Eventuali mezzi punti saranno arrotondati all'intero precedente. \begin{table}[h]     \centering \begin{tabular}{|c|c|c|c|c|c|c|c|c|c|c|c|c|c|c|c|c|c|c|c|} \hline Punti &  $\leq 4$ & 5 & 6 & 7 & 8 & 9 & 10 & 11 & \textbf{12} & 13 & 14 & 15 & 16 & 17 & 18 & 19 & 20 \\ \hline Voto & 2 & 2.5 & 3 & 3.5 & 4 & 4.5 & 5 & 5.5 & \textbf{6} & 6.5 & 7 & 7.5 & 8 & 8.5 & 9 & 9.5 & 10 \\ \hline \end{tabular} \end{table}
\textbf{Codice}: qrlxaue zkoeoy gjrefp


\begin{enumerate}
  \item Sia \vec{F} una forza conservativa. Allora
  \begin{enumerate}[label=\Alph*.]
    \item il lavoro di \vec{F} è nullo solo se il percorso è chiuso.
    \item il lavoro di \vec{F} può essere nullo anche se il percorso non è chiuso.
    \item il lavoro di \vec{F} è sempre nullo.
    \item il lavoro di \vec{F} non dipende dal percorso considerato.
  \end{enumerate}
  \item Un corpo sale lungo un piano inclinato di un angolo $\alpha$ e alto $h$, partendo da terra e arrivando fino alla cima del piano. Il lavoro della forza peso è:
  \begin{enumerate}[label=\Alph*.]
    \item $-mgh$
    \item $mgh\sin\alpha$
    \item $mgh\cos\alpha$
    \item $mgh$
  \end{enumerate}
  \item Un corpo sale da terra fino a un'altezza $h$. Il lavoro della forza peso è:
  \begin{enumerate}[label=\Alph*.]
    \item positivo.
    \item nullo
    \item $mgh$
    \item negativo.
  \end{enumerate}
  \item Un corpo di massa $m_1$ e un corpo di massa $m_2$ sono lanciati in contemporanea da una torre alta $h$. Quali sono le velocità dei due corpi appena prima di toccare terra?
  \begin{enumerate}[label=\Alph*.]
    \item $v_1=\sqrt{2m_1gh}, v_2=\sqrt{2m_2gh}$.
    \item v_1=0, v_2=0
    \item $v_1=\sqrt{2gh}, v_2=\sqrt{2gh}$.
    \item $v_1=\sqrt{\frac{2gh}{m_1}}, v_2=\sqrt{\frac{2gh}{m_2}}$.
  \end{enumerate}
  \item Sia \vec{F} una forza generica, che agisce su un corpo che effettua uno spostamento $\Delta \vec{s}$. Allora
  \begin{enumerate}[label=\Alph*.]
    \item il suo lavoro non ha una direzione perché è scalare.
    \item il suo lavoro ha la stessa direzione della forza.
    \item il suo lavoro ha la stessa direzione dello spostamento.
    \item il suo lavoro ha una direzione data dalla regola del parallelogramma.
  \end{enumerate}
  \item Un corpo ha inizialmente una velocità $v$ e dopo un certo tempo si ferma. La variazione di energia cinetica è
  \begin{enumerate}[label=\Alph*.]
    \item è positiva.
    \item è negativa
    \item dipende dallo spazio percorso.
    \item dipende dal tempo in cui si ferma.
  \end{enumerate}
  \item Sia \vec{F} una forza non conservativa. Allora
  \begin{enumerate}[label=\Alph*.]
    \item la sua energia potenziale non esiste.
    \item la sua energia potenziale è tale che lungo un percorso $\Delta U=-W$.
    \item la sua energia potenziale è sempre negativa.
    \item la sua energia potenziale è tale che lungo un percorso $\Delta U=W$.
  \end{enumerate}
  \item Un corpo, partendo da fermo, rotola giù dalla cima di un piano inclinato di un angolo $\alpha$, alto $h$. Quale è la sua velocità quando arriva in fondo?
  \begin{enumerate}[label=\Alph*.]
    \item $\sqrt{2gh}$.
    \item $\sqrt{2gh\sin\alpha}$.
    \item $\sqrt{2gh}\sin\alpha$.
    \item $\sqrt{mgh\cos\alpha}$.
  \end{enumerate}
  \item Un corpo sale da terra fino a un'altezza $h$. Il lavoro della forza peso è:
  \begin{enumerate}[label=\Alph*.]
    \item positivo.
    \item nullo
    \item $mgh$
    \item negativo.
  \end{enumerate}
  \item L'energia meccanica di un corpo che si trova a un'altezza $h$
  \begin{enumerate}[label=\Alph*.]
    \item è $mgh$.
    \item è negativa
    \item dipende dalla velocità del corpo.
    \item è nulla.
  \end{enumerate}
  \item Il lavoro della forza d'attrito
  \begin{enumerate}[label=\Alph*.]
    \item non può essere calcolato perché la forza non è conservativa.
    \item è sempre negativo.
    \item è nullo se il percorso è rettilineo.
    \item è sempre positivo.
  \end{enumerate}
  \item L'energia cinetica di un corpo fermo che si trova a un'altezza $h$
  \begin{enumerate}[label=\Alph*.]
    \item è $mgh$.
    \item è nulla.
    \item si conserva.
    \item dipende da $h$
  \end{enumerate}
  \item Se calcoliamo il lavoro di una forza lungo un percorso chiuso e troviamo zero, cosa possiamo concludere?
  \begin{enumerate}[label=\Alph*.]
    \item niente.
    \item la forza è conservativa.
    \item la forza non è conservativa.
    \item la forza è nulla.
  \end{enumerate}
  \item Un corpo viene lasciato cadere da fermo da un altezza $h$. Quando raggiunge il suolo ha una velocità $v$. Quale relazione esprime correttamente la conservazione dell'energia?
  \begin{enumerate}[label=\Alph*.]
    \item $\frac{1}{2}mv^2+mgh=0.$
    \item $mgh+\frac{1}{2}mv^2=\frac{1}{2}mv^2.$
    \item $mgh=\frac{1}{2}mv^2$.
    \item $\frac{1}{2}mh^2=\frac{1}{2}mv^2$.
  \end{enumerate}
  \item Sia $E_m$ l'energia meccanica, $K$ l'energia cinetica, $U$ quella potenziale e $W$ il lavoro di tutte le forze agenti. Quale delle seguenti relazioni è vera?
  \begin{enumerate}[label=\Alph*.]
    \item $\Delta K=W.$
    \item $\Delta U=W$.
    \item $\Delta U=W$.
    \item $\Delta E_m=W$.
  \end{enumerate}
  \item Sia $E_m$ l'energia meccanica, $K$ l'energia cinetica, $U$ quella potenziale e $W_{NC}$ il lavoro delle forze non conservative agenti. Quale delle seguenti relazioni è vera?
  \begin{enumerate}[label=\Alph*.]
    \item $\Delta K=W_{NC}.$
    \item $\Delta E_m=W_{NC}$.
    \item $\Delta K = \Delta U$.
    \item $\Delta U=W_{NC}$.
  \end{enumerate}
  \item Il lavoro della forza peso
  \begin{enumerate}[label=\Alph*.]
    \item è sempre negativo.
    \item è nullo se il percorso ha gli estremi posti alla stessa altitudine.
    \item è sempre positivo.
    \item è nullo solo se il percorso è chiuso.
  \end{enumerate}
  \item Un corpo di massa $m$ scivola su un piano con attrito, partendo da una velocità v. Il coefficiente di attrito è \mu. Dopo quanto spazio si ferma?
  \begin{enumerate}[label=\Alph*.]
    \item $\frac{v^2}{2g\mu}}$.
    \item $\frac{1}{2}v^2+\mu g$.
    \item $\frac{2v^2}{g\mu}}$.
    \item $\frac{1}{2}v^2-\mu g$.
  \end{enumerate}
  \item Sia \vec{F} una forza non conservativa. Allora
  \begin{enumerate}[label=\Alph*.]
    \item esiste un percorso chiuso il cui lavoro non è nullo.
    \item il lavoro non dipende dal percorso.
    \item il lavoro non dipende dagli estremi del percorso.
    \item per ogni percorso aperto il lavoro è nullo.
  \end{enumerate}
  \item Il lavoro della reazione vincolare
  \begin{enumerate}[label=\Alph*.]
    \item è sempre nullo.
    \item è nullo solo se il percorso è chiuso.
    \item può assumere qualsiasi valore.
    \item è sempre diverso da zero.
  \end{enumerate}
  \item Un corpo impatta una molla con una velocità $v$. La molla di costante elastica $k$, opponendosi all'avanzare del corpo, lo ferma. Quale è il suo allungamento nel momento in cui si ferma, trascurando l'attrito?
  \begin{enumerate}[label=\Alph*.]
    \item $\sqrt{\frac{m}{k}v}$.
    \item $\sqrt{\frac{m}{k}}v$.
    \item $\sqrt{\frac{k}{m}}v$.
    \item $\sqrt{\frac{k}{m}}v$.
  \end{enumerate}
\end{enumerate}








\newpage \maketitle \centering \textbf{Griglia di valutazione}. Eventuali mezzi punti saranno arrotondati all'intero precedente. \begin{table}[h]     \centering \begin{tabular}{|c|c|c|c|c|c|c|c|c|c|c|c|c|c|c|c|c|c|c|c|} \hline Punti &  $\leq 4$ & 5 & 6 & 7 & 8 & 9 & 10 & 11 & \textbf{12} & 13 & 14 & 15 & 16 & 17 & 18 & 19 & 20 \\ \hline Voto & 2 & 2.5 & 3 & 3.5 & 4 & 4.5 & 5 & 5.5 & \textbf{6} & 6.5 & 7 & 7.5 & 8 & 8.5 & 9 & 9.5 & 10 \\ \hline \end{tabular} \end{table}
\textbf{Codice}: stkxawhbylnerw hjrgir


\begin{enumerate}
  \item Un corpo viene lasciato cadere da fermo da un altezza $h$. Quando raggiunge il suolo ha una velocità $v$. Quale relazione esprime correttamente la conservazione dell'energia?
  \begin{enumerate}[label=\Alph*.]
    \item $\frac{1}{2}mv^2+mgh=0.$
    \item $mgh+\frac{1}{2}mv^2=\frac{1}{2}mv^2.$
    \item $\frac{1}{2}mh^2=\frac{1}{2}mv^2$.
    \item $mgh=\frac{1}{2}mv^2$.
  \end{enumerate}
  \item Il lavoro della forza peso
  \begin{enumerate}[label=\Alph*.]
    \item è sempre negativo.
    \item è nullo solo se il percorso è chiuso.
    \item è nullo se il percorso ha gli estremi posti alla stessa altitudine.
    \item è sempre positivo.
  \end{enumerate}
  \item Il lavoro della reazione vincolare
  \begin{enumerate}[label=\Alph*.]
    \item è nullo solo se il percorso è chiuso.
    \item è sempre diverso da zero.
    \item è sempre nullo.
    \item può assumere qualsiasi valore.
  \end{enumerate}
  \item Il lavoro della forza d'attrito
  \begin{enumerate}[label=\Alph*.]
    \item non può essere calcolato perché la forza non è conservativa.
    \item è sempre positivo.
    \item è sempre negativo.
    \item è nullo se il percorso è rettilineo.
  \end{enumerate}
  \item Sia \vec{F} una forza generica, che agisce su un corpo che effettua uno spostamento $\Delta \vec{s}$. Allora
  \begin{enumerate}[label=\Alph*.]
    \item il suo lavoro non ha una direzione perché è scalare.
    \item il suo lavoro ha una direzione data dalla regola del parallelogramma.
    \item il suo lavoro ha la stessa direzione della forza.
    \item il suo lavoro ha la stessa direzione dello spostamento.
  \end{enumerate}
  \item Un corpo sale da terra fino a un'altezza $h$. Il lavoro della forza peso è:
  \begin{enumerate}[label=\Alph*.]
    \item nullo
    \item $mgh$
    \item positivo.
    \item negativo.
  \end{enumerate}
  \item L'energia meccanica di un corpo che si trova a un'altezza $h$
  \begin{enumerate}[label=\Alph*.]
    \item è negativa
    \item è nulla.
    \item è $mgh$.
    \item dipende dalla velocità del corpo.
  \end{enumerate}
  \item Un corpo ha inizialmente una velocità $v$ e dopo un certo tempo si ferma. La variazione di energia cinetica è
  \begin{enumerate}[label=\Alph*.]
    \item è positiva.
    \item dipende dallo spazio percorso.
    \item è negativa
    \item dipende dal tempo in cui si ferma.
  \end{enumerate}
  \item Sia \vec{F} una forza conservativa. Allora
  \begin{enumerate}[label=\Alph*.]
    \item il lavoro di \vec{F} è nullo solo se il percorso è chiuso.
    \item il lavoro di \vec{F} è sempre nullo.
    \item il lavoro di \vec{F} può essere nullo anche se il percorso non è chiuso.
    \item il lavoro di \vec{F} non dipende dal percorso considerato.
  \end{enumerate}
  \item Sia \vec{F} una forza non conservativa. Allora
  \begin{enumerate}[label=\Alph*.]
    \item il lavoro non dipende dal percorso.
    \item il lavoro non dipende dagli estremi del percorso.
    \item per ogni percorso aperto il lavoro è nullo.
    \item esiste un percorso chiuso il cui lavoro non è nullo.
  \end{enumerate}
  \item Sia $E_m$ l'energia meccanica, $K$ l'energia cinetica, $U$ quella potenziale e $W$ il lavoro di tutte le forze agenti. Quale delle seguenti relazioni è vera?
  \begin{enumerate}[label=\Alph*.]
    \item $\Delta K=W.$
    \item $\Delta U=W$.
    \item $\Delta E_m=W$.
    \item $\Delta U=W$.
  \end{enumerate}
  \item Un corpo sale da terra fino a un'altezza $h$. Il lavoro della forza peso è:
  \begin{enumerate}[label=\Alph*.]
    \item nullo
    \item negativo.
    \item $mgh$
    \item positivo.
  \end{enumerate}
  \item Un corpo, partendo da fermo, rotola giù dalla cima di un piano inclinato di un angolo $\alpha$, alto $h$. Quale è la sua velocità quando arriva in fondo?
  \begin{enumerate}[label=\Alph*.]
    \item $\sqrt{2gh}\sin\alpha$.
    \item $\sqrt{2gh\sin\alpha}$.
    \item $\sqrt{mgh\cos\alpha}$.
    \item $\sqrt{2gh}$.
  \end{enumerate}
  \item Un corpo impatta una molla con una velocità $v$. La molla di costante elastica $k$, opponendosi all'avanzare del corpo, lo ferma. Quale è il suo allungamento nel momento in cui si ferma, trascurando l'attrito?
  \begin{enumerate}[label=\Alph*.]
    \item $\sqrt{\frac{m}{k}}v$.
    \item $\sqrt{\frac{k}{m}}v$.
    \item $\sqrt{\frac{m}{k}v}$.
    \item $\sqrt{\frac{k}{m}}v$.
  \end{enumerate}
  \item Se calcoliamo il lavoro di una forza lungo un percorso chiuso e troviamo zero, cosa possiamo concludere?
  \begin{enumerate}[label=\Alph*.]
    \item niente.
    \item la forza è nulla.
    \item la forza non è conservativa.
    \item la forza è conservativa.
  \end{enumerate}
  \item Un corpo sale lungo un piano inclinato di un angolo $\alpha$ e alto $h$, partendo da terra e arrivando fino alla cima del piano. Il lavoro della forza peso è:
  \begin{enumerate}[label=\Alph*.]
    \item $mgh\sin\alpha$
    \item $mgh$
    \item $-mgh$
    \item $mgh\cos\alpha$
  \end{enumerate}
  \item Sia \vec{F} una forza non conservativa. Allora
  \begin{enumerate}[label=\Alph*.]
    \item la sua energia potenziale è sempre negativa.
    \item la sua energia potenziale non esiste.
    \item la sua energia potenziale è tale che lungo un percorso $\Delta U=W$.
    \item la sua energia potenziale è tale che lungo un percorso $\Delta U=-W$.
  \end{enumerate}
  \item L'energia cinetica di un corpo fermo che si trova a un'altezza $h$
  \begin{enumerate}[label=\Alph*.]
    \item è nulla.
    \item è $mgh$.
    \item si conserva.
    \item dipende da $h$
  \end{enumerate}
  \item Sia $E_m$ l'energia meccanica, $K$ l'energia cinetica, $U$ quella potenziale e $W_{NC}$ il lavoro delle forze non conservative agenti. Quale delle seguenti relazioni è vera?
  \begin{enumerate}[label=\Alph*.]
    \item $\Delta U=W_{NC}$.
    \item $\Delta K = \Delta U$.
    \item $\Delta E_m=W_{NC}$.
    \item $\Delta K=W_{NC}.$
  \end{enumerate}
  \item Un corpo di massa $m_1$ e un corpo di massa $m_2$ sono lanciati in contemporanea da una torre alta $h$. Quali sono le velocità dei due corpi appena prima di toccare terra?
  \begin{enumerate}[label=\Alph*.]
    \item $v_1=\sqrt{\frac{2gh}{m_1}}, v_2=\sqrt{\frac{2gh}{m_2}}$.
    \item v_1=0, v_2=0
    \item $v_1=\sqrt{2m_1gh}, v_2=\sqrt{2m_2gh}$.
    \item $v_1=\sqrt{2gh}, v_2=\sqrt{2gh}$.
  \end{enumerate}
  \item Un corpo di massa $m$ scivola su un piano con attrito, partendo da una velocità v. Il coefficiente di attrito è \mu. Dopo quanto spazio si ferma?
  \begin{enumerate}[label=\Alph*.]
    \item $\frac{1}{2}v^2+\mu g$.
    \item $\frac{2v^2}{g\mu}}$.
    \item $\frac{1}{2}v^2-\mu g$.
    \item $\frac{v^2}{2g\mu}}$.
  \end{enumerate}
\end{enumerate}








\newpage \maketitle \centering \textbf{Griglia di valutazione}. Eventuali mezzi punti saranno arrotondati all'intero precedente. \begin{table}[h]     \centering \begin{tabular}{|c|c|c|c|c|c|c|c|c|c|c|c|c|c|c|c|c|c|c|c|} \hline Punti &  $\leq 4$ & 5 & 6 & 7 & 8 & 9 & 10 & 11 & \textbf{12} & 13 & 14 & 15 & 16 & 17 & 18 & 19 & 20 \\ \hline Voto & 2 & 2.5 & 3 & 3.5 & 4 & 4.5 & 5 & 5.5 & \textbf{6} & 6.5 & 7 & 7.5 & 8 & 8.5 & 9 & 9.5 & 10 \\ \hline \end{tabular} \end{table}
\textbf{Codice}: rrjwdwgcyiqdqwchjtgfo


\begin{enumerate}
  \item Sia \vec{F} una forza conservativa. Allora
  \begin{enumerate}[label=\Alph*.]
    \item il lavoro di \vec{F} è sempre nullo.
    \item il lavoro di \vec{F} non dipende dal percorso considerato.
    \item il lavoro di \vec{F} può essere nullo anche se il percorso non è chiuso.
    \item il lavoro di \vec{F} è nullo solo se il percorso è chiuso.
  \end{enumerate}
  \item Un corpo impatta una molla con una velocità $v$. La molla di costante elastica $k$, opponendosi all'avanzare del corpo, lo ferma. Quale è il suo allungamento nel momento in cui si ferma, trascurando l'attrito?
  \begin{enumerate}[label=\Alph*.]
    \item $\sqrt{\frac{m}{k}}v$.
    \item $\sqrt{\frac{k}{m}}v$.
    \item $\sqrt{\frac{m}{k}v}$.
    \item $\sqrt{\frac{k}{m}}v$.
  \end{enumerate}
  \item Sia \vec{F} una forza non conservativa. Allora
  \begin{enumerate}[label=\Alph*.]
    \item la sua energia potenziale è sempre negativa.
    \item la sua energia potenziale non esiste.
    \item la sua energia potenziale è tale che lungo un percorso $\Delta U=W$.
    \item la sua energia potenziale è tale che lungo un percorso $\Delta U=-W$.
  \end{enumerate}
  \item Un corpo sale lungo un piano inclinato di un angolo $\alpha$ e alto $h$, partendo da terra e arrivando fino alla cima del piano. Il lavoro della forza peso è:
  \begin{enumerate}[label=\Alph*.]
    \item $mgh\sin\alpha$
    \item $-mgh$
    \item $mgh$
    \item $mgh\cos\alpha$
  \end{enumerate}
  \item Un corpo, partendo da fermo, rotola giù dalla cima di un piano inclinato di un angolo $\alpha$, alto $h$. Quale è la sua velocità quando arriva in fondo?
  \begin{enumerate}[label=\Alph*.]
    \item $\sqrt{2gh\sin\alpha}$.
    \item $\sqrt{2gh}\sin\alpha$.
    \item $\sqrt{mgh\cos\alpha}$.
    \item $\sqrt{2gh}$.
  \end{enumerate}
  \item Un corpo viene lasciato cadere da fermo da un altezza $h$. Quando raggiunge il suolo ha una velocità $v$. Quale relazione esprime correttamente la conservazione dell'energia?
  \begin{enumerate}[label=\Alph*.]
    \item $\frac{1}{2}mh^2=\frac{1}{2}mv^2$.
    \item $mgh+\frac{1}{2}mv^2=\frac{1}{2}mv^2.$
    \item $\frac{1}{2}mv^2+mgh=0.$
    \item $mgh=\frac{1}{2}mv^2$.
  \end{enumerate}
  \item Se calcoliamo il lavoro di una forza lungo un percorso chiuso e troviamo zero, cosa possiamo concludere?
  \begin{enumerate}[label=\Alph*.]
    \item la forza è conservativa.
    \item la forza non è conservativa.
    \item niente.
    \item la forza è nulla.
  \end{enumerate}
  \item Un corpo di massa $m$ scivola su un piano con attrito, partendo da una velocità v. Il coefficiente di attrito è \mu. Dopo quanto spazio si ferma?
  \begin{enumerate}[label=\Alph*.]
    \item $\frac{1}{2}v^2+\mu g$.
    \item $\frac{2v^2}{g\mu}}$.
    \item $\frac{1}{2}v^2-\mu g$.
    \item $\frac{v^2}{2g\mu}}$.
  \end{enumerate}
  \item Un corpo ha inizialmente una velocità $v$ e dopo un certo tempo si ferma. La variazione di energia cinetica è
  \begin{enumerate}[label=\Alph*.]
    \item è positiva.
    \item dipende dal tempo in cui si ferma.
    \item è negativa
    \item dipende dallo spazio percorso.
  \end{enumerate}
  \item Il lavoro della forza peso
  \begin{enumerate}[label=\Alph*.]
    \item è nullo se il percorso ha gli estremi posti alla stessa altitudine.
    \item è sempre negativo.
    \item è sempre positivo.
    \item è nullo solo se il percorso è chiuso.
  \end{enumerate}
  \item Il lavoro della reazione vincolare
  \begin{enumerate}[label=\Alph*.]
    \item è sempre diverso da zero.
    \item può assumere qualsiasi valore.
    \item è nullo solo se il percorso è chiuso.
    \item è sempre nullo.
  \end{enumerate}
  \item Sia $E_m$ l'energia meccanica, $K$ l'energia cinetica, $U$ quella potenziale e $W_{NC}$ il lavoro delle forze non conservative agenti. Quale delle seguenti relazioni è vera?
  \begin{enumerate}[label=\Alph*.]
    \item $\Delta E_m=W_{NC}$.
    \item $\Delta K = \Delta U$.
    \item $\Delta K=W_{NC}.$
    \item $\Delta U=W_{NC}$.
  \end{enumerate}
  \item Un corpo di massa $m_1$ e un corpo di massa $m_2$ sono lanciati in contemporanea da una torre alta $h$. Quali sono le velocità dei due corpi appena prima di toccare terra?
  \begin{enumerate}[label=\Alph*.]
    \item $v_1=\sqrt{\frac{2gh}{m_1}}, v_2=\sqrt{\frac{2gh}{m_2}}$.
    \item $v_1=\sqrt{2m_1gh}, v_2=\sqrt{2m_2gh}$.
    \item $v_1=\sqrt{2gh}, v_2=\sqrt{2gh}$.
    \item v_1=0, v_2=0
  \end{enumerate}
  \item Sia \vec{F} una forza generica, che agisce su un corpo che effettua uno spostamento $\Delta \vec{s}$. Allora
  \begin{enumerate}[label=\Alph*.]
    \item il suo lavoro non ha una direzione perché è scalare.
    \item il suo lavoro ha la stessa direzione della forza.
    \item il suo lavoro ha una direzione data dalla regola del parallelogramma.
    \item il suo lavoro ha la stessa direzione dello spostamento.
  \end{enumerate}
  \item Un corpo sale da terra fino a un'altezza $h$. Il lavoro della forza peso è:
  \begin{enumerate}[label=\Alph*.]
    \item positivo.
    \item $mgh$
    \item nullo
    \item negativo.
  \end{enumerate}
  \item Sia \vec{F} una forza non conservativa. Allora
  \begin{enumerate}[label=\Alph*.]
    \item il lavoro non dipende dal percorso.
    \item il lavoro non dipende dagli estremi del percorso.
    \item esiste un percorso chiuso il cui lavoro non è nullo.
    \item per ogni percorso aperto il lavoro è nullo.
  \end{enumerate}
  \item L'energia meccanica di un corpo che si trova a un'altezza $h$
  \begin{enumerate}[label=\Alph*.]
    \item è nulla.
    \item dipende dalla velocità del corpo.
    \item è negativa
    \item è $mgh$.
  \end{enumerate}
  \item L'energia cinetica di un corpo fermo che si trova a un'altezza $h$
  \begin{enumerate}[label=\Alph*.]
    \item si conserva.
    \item è $mgh$.
    \item è nulla.
    \item dipende da $h$
  \end{enumerate}
  \item Un corpo sale da terra fino a un'altezza $h$. Il lavoro della forza peso è:
  \begin{enumerate}[label=\Alph*.]
    \item $mgh$
    \item nullo
    \item negativo.
    \item positivo.
  \end{enumerate}
  \item Il lavoro della forza d'attrito
  \begin{enumerate}[label=\Alph*.]
    \item è sempre negativo.
    \item non può essere calcolato perché la forza non è conservativa.
    \item è sempre positivo.
    \item è nullo se il percorso è rettilineo.
  \end{enumerate}
  \item Sia $E_m$ l'energia meccanica, $K$ l'energia cinetica, $U$ quella potenziale e $W$ il lavoro di tutte le forze agenti. Quale delle seguenti relazioni è vera?
  \begin{enumerate}[label=\Alph*.]
    \item $\Delta K=W.$
    \item $\Delta E_m=W$.
    \item $\Delta U=W$.
    \item $\Delta U=W$.
  \end{enumerate}
\end{enumerate}








\newpage \maketitle \centering \textbf{Griglia di valutazione}. Eventuali mezzi punti saranno arrotondati all'intero precedente. \begin{table}[h]     \centering \begin{tabular}{|c|c|c|c|c|c|c|c|c|c|c|c|c|c|c|c|c|c|c|c|} \hline Punti &  $\leq 4$ & 5 & 6 & 7 & 8 & 9 & 10 & 11 & \textbf{12} & 13 & 14 & 15 & 16 & 17 & 18 & 19 & 20 \\ \hline Voto & 2 & 2.5 & 3 & 3.5 & 4 & 4.5 & 5 & 5.5 & \textbf{6} & 6.5 & 7 & 7.5 & 8 & 8.5 & 9 & 9.5 & 10 \\ \hline \end{tabular} \end{table}
\textbf{Codice}: sskxbwhbzlndowchisgip


\begin{enumerate}
  \item Un corpo impatta una molla con una velocità $v$. La molla di costante elastica $k$, opponendosi all'avanzare del corpo, lo ferma. Quale è il suo allungamento nel momento in cui si ferma, trascurando l'attrito?
  \begin{enumerate}[label=\Alph*.]
    \item $\sqrt{\frac{k}{m}}v$.
    \item $\sqrt{\frac{m}{k}v}$.
    \item $\sqrt{\frac{k}{m}}v$.
    \item $\sqrt{\frac{m}{k}}v$.
  \end{enumerate}
  \item Sia $E_m$ l'energia meccanica, $K$ l'energia cinetica, $U$ quella potenziale e $W_{NC}$ il lavoro delle forze non conservative agenti. Quale delle seguenti relazioni è vera?
  \begin{enumerate}[label=\Alph*.]
    \item $\Delta U=W_{NC}$.
    \item $\Delta E_m=W_{NC}$.
    \item $\Delta K = \Delta U$.
    \item $\Delta K=W_{NC}.$
  \end{enumerate}
  \item Un corpo di massa $m$ scivola su un piano con attrito, partendo da una velocità v. Il coefficiente di attrito è \mu. Dopo quanto spazio si ferma?
  \begin{enumerate}[label=\Alph*.]
    \item $\frac{1}{2}v^2+\mu g$.
    \item $\frac{2v^2}{g\mu}}$.
    \item $\frac{v^2}{2g\mu}}$.
    \item $\frac{1}{2}v^2-\mu g$.
  \end{enumerate}
  \item Sia \vec{F} una forza generica, che agisce su un corpo che effettua uno spostamento $\Delta \vec{s}$. Allora
  \begin{enumerate}[label=\Alph*.]
    \item il suo lavoro ha la stessa direzione della forza.
    \item il suo lavoro ha una direzione data dalla regola del parallelogramma.
    \item il suo lavoro non ha una direzione perché è scalare.
    \item il suo lavoro ha la stessa direzione dello spostamento.
  \end{enumerate}
  \item Un corpo ha inizialmente una velocità $v$ e dopo un certo tempo si ferma. La variazione di energia cinetica è
  \begin{enumerate}[label=\Alph*.]
    \item dipende dallo spazio percorso.
    \item è negativa
    \item dipende dal tempo in cui si ferma.
    \item è positiva.
  \end{enumerate}
  \item Un corpo sale da terra fino a un'altezza $h$. Il lavoro della forza peso è:
  \begin{enumerate}[label=\Alph*.]
    \item nullo
    \item $mgh$
    \item positivo.
    \item negativo.
  \end{enumerate}
  \item L'energia meccanica di un corpo che si trova a un'altezza $h$
  \begin{enumerate}[label=\Alph*.]
    \item è negativa
    \item è nulla.
    \item è $mgh$.
    \item dipende dalla velocità del corpo.
  \end{enumerate}
  \item Se calcoliamo il lavoro di una forza lungo un percorso chiuso e troviamo zero, cosa possiamo concludere?
  \begin{enumerate}[label=\Alph*.]
    \item la forza è conservativa.
    \item la forza è nulla.
    \item niente.
    \item la forza non è conservativa.
  \end{enumerate}
  \item Un corpo, partendo da fermo, rotola giù dalla cima di un piano inclinato di un angolo $\alpha$, alto $h$. Quale è la sua velocità quando arriva in fondo?
  \begin{enumerate}[label=\Alph*.]
    \item $\sqrt{2gh}\sin\alpha$.
    \item $\sqrt{2gh\sin\alpha}$.
    \item $\sqrt{mgh\cos\alpha}$.
    \item $\sqrt{2gh}$.
  \end{enumerate}
  \item Un corpo di massa $m_1$ e un corpo di massa $m_2$ sono lanciati in contemporanea da una torre alta $h$. Quali sono le velocità dei due corpi appena prima di toccare terra?
  \begin{enumerate}[label=\Alph*.]
    \item $v_1=\sqrt{\frac{2gh}{m_1}}, v_2=\sqrt{\frac{2gh}{m_2}}$.
    \item $v_1=\sqrt{2m_1gh}, v_2=\sqrt{2m_2gh}$.
    \item v_1=0, v_2=0
    \item $v_1=\sqrt{2gh}, v_2=\sqrt{2gh}$.
  \end{enumerate}
  \item L'energia cinetica di un corpo fermo che si trova a un'altezza $h$
  \begin{enumerate}[label=\Alph*.]
    \item è nulla.
    \item si conserva.
    \item è $mgh$.
    \item dipende da $h$
  \end{enumerate}
  \item Il lavoro della forza peso
  \begin{enumerate}[label=\Alph*.]
    \item è nullo se il percorso ha gli estremi posti alla stessa altitudine.
    \item è sempre negativo.
    \item è nullo solo se il percorso è chiuso.
    \item è sempre positivo.
  \end{enumerate}
  \item Un corpo sale lungo un piano inclinato di un angolo $\alpha$ e alto $h$, partendo da terra e arrivando fino alla cima del piano. Il lavoro della forza peso è:
  \begin{enumerate}[label=\Alph*.]
    \item $-mgh$
    \item $mgh\cos\alpha$
    \item $mgh$
    \item $mgh\sin\alpha$
  \end{enumerate}
  \item Un corpo sale da terra fino a un'altezza $h$. Il lavoro della forza peso è:
  \begin{enumerate}[label=\Alph*.]
    \item negativo.
    \item $mgh$
    \item nullo
    \item positivo.
  \end{enumerate}
  \item Sia $E_m$ l'energia meccanica, $K$ l'energia cinetica, $U$ quella potenziale e $W$ il lavoro di tutte le forze agenti. Quale delle seguenti relazioni è vera?
  \begin{enumerate}[label=\Alph*.]
    \item $\Delta U=W$.
    \item $\Delta E_m=W$.
    \item $\Delta U=W$.
    \item $\Delta K=W.$
  \end{enumerate}
  \item Il lavoro della reazione vincolare
  \begin{enumerate}[label=\Alph*.]
    \item può assumere qualsiasi valore.
    \item è sempre diverso da zero.
    \item è sempre nullo.
    \item è nullo solo se il percorso è chiuso.
  \end{enumerate}
  \item Un corpo viene lasciato cadere da fermo da un altezza $h$. Quando raggiunge il suolo ha una velocità $v$. Quale relazione esprime correttamente la conservazione dell'energia?
  \begin{enumerate}[label=\Alph*.]
    \item $mgh=\frac{1}{2}mv^2$.
    \item $\frac{1}{2}mv^2+mgh=0.$
    \item $mgh+\frac{1}{2}mv^2=\frac{1}{2}mv^2.$
    \item $\frac{1}{2}mh^2=\frac{1}{2}mv^2$.
  \end{enumerate}
  \item Sia \vec{F} una forza non conservativa. Allora
  \begin{enumerate}[label=\Alph*.]
    \item il lavoro non dipende dal percorso.
    \item esiste un percorso chiuso il cui lavoro non è nullo.
    \item il lavoro non dipende dagli estremi del percorso.
    \item per ogni percorso aperto il lavoro è nullo.
  \end{enumerate}
  \item Sia \vec{F} una forza conservativa. Allora
  \begin{enumerate}[label=\Alph*.]
    \item il lavoro di \vec{F} è sempre nullo.
    \item il lavoro di \vec{F} non dipende dal percorso considerato.
    \item il lavoro di \vec{F} può essere nullo anche se il percorso non è chiuso.
    \item il lavoro di \vec{F} è nullo solo se il percorso è chiuso.
  \end{enumerate}
  \item Sia \vec{F} una forza non conservativa. Allora
  \begin{enumerate}[label=\Alph*.]
    \item la sua energia potenziale è tale che lungo un percorso $\Delta U=W$.
    \item la sua energia potenziale è tale che lungo un percorso $\Delta U=-W$.
    \item la sua energia potenziale è sempre negativa.
    \item la sua energia potenziale non esiste.
  \end{enumerate}
  \item Il lavoro della forza d'attrito
  \begin{enumerate}[label=\Alph*.]
    \item è nullo se il percorso è rettilineo.
    \item è sempre negativo.
    \item è sempre positivo.
    \item non può essere calcolato perché la forza non è conservativa.
  \end{enumerate}
\end{enumerate}








\newpage \maketitle \centering \textbf{Griglia di valutazione}. Eventuali mezzi punti saranno arrotondati all'intero precedente. \begin{table}[h]     \centering \begin{tabular}{|c|c|c|c|c|c|c|c|c|c|c|c|c|c|c|c|c|c|c|c|} \hline Punti &  $\leq 4$ & 5 & 6 & 7 & 8 & 9 & 10 & 11 & \textbf{12} & 13 & 14 & 15 & 16 & 17 & 18 & 19 & 20 \\ \hline Voto & 2 & 2.5 & 3 & 3.5 & 4 & 4.5 & 5 & 5.5 & \textbf{6} & 6.5 & 7 & 7.5 & 8 & 8.5 & 9 & 9.5 & 10 \\ \hline \end{tabular} \end{table}
\textbf{Codice}: prkvavecykogpzcfisehq


\begin{enumerate}
  \item Un corpo sale da terra fino a un'altezza $h$. Il lavoro della forza peso è:
  \begin{enumerate}[label=\Alph*.]
    \item negativo.
    \item positivo.
    \item nullo
    \item $mgh$
  \end{enumerate}
  \item Il lavoro della reazione vincolare
  \begin{enumerate}[label=\Alph*.]
    \item è sempre nullo.
    \item può assumere qualsiasi valore.
    \item è nullo solo se il percorso è chiuso.
    \item è sempre diverso da zero.
  \end{enumerate}
  \item Un corpo, partendo da fermo, rotola giù dalla cima di un piano inclinato di un angolo $\alpha$, alto $h$. Quale è la sua velocità quando arriva in fondo?
  \begin{enumerate}[label=\Alph*.]
    \item $\sqrt{mgh\cos\alpha}$.
    \item $\sqrt{2gh}\sin\alpha$.
    \item $\sqrt{2gh}$.
    \item $\sqrt{2gh\sin\alpha}$.
  \end{enumerate}
  \item L'energia cinetica di un corpo fermo che si trova a un'altezza $h$
  \begin{enumerate}[label=\Alph*.]
    \item è nulla.
    \item è $mgh$.
    \item si conserva.
    \item dipende da $h$
  \end{enumerate}
  \item Sia $E_m$ l'energia meccanica, $K$ l'energia cinetica, $U$ quella potenziale e $W$ il lavoro di tutte le forze agenti. Quale delle seguenti relazioni è vera?
  \begin{enumerate}[label=\Alph*.]
    \item $\Delta K=W.$
    \item $\Delta U=W$.
    \item $\Delta E_m=W$.
    \item $\Delta U=W$.
  \end{enumerate}
  \item Se calcoliamo il lavoro di una forza lungo un percorso chiuso e troviamo zero, cosa possiamo concludere?
  \begin{enumerate}[label=\Alph*.]
    \item la forza non è conservativa.
    \item la forza è conservativa.
    \item niente.
    \item la forza è nulla.
  \end{enumerate}
  \item Il lavoro della forza d'attrito
  \begin{enumerate}[label=\Alph*.]
    \item è sempre negativo.
    \item è nullo se il percorso è rettilineo.
    \item non può essere calcolato perché la forza non è conservativa.
    \item è sempre positivo.
  \end{enumerate}
  \item Un corpo viene lasciato cadere da fermo da un altezza $h$. Quando raggiunge il suolo ha una velocità $v$. Quale relazione esprime correttamente la conservazione dell'energia?
  \begin{enumerate}[label=\Alph*.]
    \item $\frac{1}{2}mv^2+mgh=0.$
    \item $mgh+\frac{1}{2}mv^2=\frac{1}{2}mv^2.$
    \item $\frac{1}{2}mh^2=\frac{1}{2}mv^2$.
    \item $mgh=\frac{1}{2}mv^2$.
  \end{enumerate}
  \item Sia $E_m$ l'energia meccanica, $K$ l'energia cinetica, $U$ quella potenziale e $W_{NC}$ il lavoro delle forze non conservative agenti. Quale delle seguenti relazioni è vera?
  \begin{enumerate}[label=\Alph*.]
    \item $\Delta K=W_{NC}.$
    \item $\Delta K = \Delta U$.
    \item $\Delta E_m=W_{NC}$.
    \item $\Delta U=W_{NC}$.
  \end{enumerate}
  \item Un corpo ha inizialmente una velocità $v$ e dopo un certo tempo si ferma. La variazione di energia cinetica è
  \begin{enumerate}[label=\Alph*.]
    \item dipende dal tempo in cui si ferma.
    \item è positiva.
    \item è negativa
    \item dipende dallo spazio percorso.
  \end{enumerate}
  \item Sia \vec{F} una forza generica, che agisce su un corpo che effettua uno spostamento $\Delta \vec{s}$. Allora
  \begin{enumerate}[label=\Alph*.]
    \item il suo lavoro ha una direzione data dalla regola del parallelogramma.
    \item il suo lavoro non ha una direzione perché è scalare.
    \item il suo lavoro ha la stessa direzione dello spostamento.
    \item il suo lavoro ha la stessa direzione della forza.
  \end{enumerate}
  \item Sia \vec{F} una forza non conservativa. Allora
  \begin{enumerate}[label=\Alph*.]
    \item il lavoro non dipende dagli estremi del percorso.
    \item il lavoro non dipende dal percorso.
    \item per ogni percorso aperto il lavoro è nullo.
    \item esiste un percorso chiuso il cui lavoro non è nullo.
  \end{enumerate}
  \item Un corpo di massa $m$ scivola su un piano con attrito, partendo da una velocità v. Il coefficiente di attrito è \mu. Dopo quanto spazio si ferma?
  \begin{enumerate}[label=\Alph*.]
    \item $\frac{1}{2}v^2-\mu g$.
    \item $\frac{v^2}{2g\mu}}$.
    \item $\frac{2v^2}{g\mu}}$.
    \item $\frac{1}{2}v^2+\mu g$.
  \end{enumerate}
  \item Un corpo impatta una molla con una velocità $v$. La molla di costante elastica $k$, opponendosi all'avanzare del corpo, lo ferma. Quale è il suo allungamento nel momento in cui si ferma, trascurando l'attrito?
  \begin{enumerate}[label=\Alph*.]
    \item $\sqrt{\frac{k}{m}}v$.
    \item $\sqrt{\frac{m}{k}v}$.
    \item $\sqrt{\frac{k}{m}}v$.
    \item $\sqrt{\frac{m}{k}}v$.
  \end{enumerate}
  \item Sia \vec{F} una forza conservativa. Allora
  \begin{enumerate}[label=\Alph*.]
    \item il lavoro di \vec{F} è sempre nullo.
    \item il lavoro di \vec{F} non dipende dal percorso considerato.
    \item il lavoro di \vec{F} è nullo solo se il percorso è chiuso.
    \item il lavoro di \vec{F} può essere nullo anche se il percorso non è chiuso.
  \end{enumerate}
  \item Un corpo di massa $m_1$ e un corpo di massa $m_2$ sono lanciati in contemporanea da una torre alta $h$. Quali sono le velocità dei due corpi appena prima di toccare terra?
  \begin{enumerate}[label=\Alph*.]
    \item $v_1=\sqrt{2gh}, v_2=\sqrt{2gh}$.
    \item $v_1=\sqrt{2m_1gh}, v_2=\sqrt{2m_2gh}$.
    \item $v_1=\sqrt{\frac{2gh}{m_1}}, v_2=\sqrt{\frac{2gh}{m_2}}$.
    \item v_1=0, v_2=0
  \end{enumerate}
  \item Un corpo sale lungo un piano inclinato di un angolo $\alpha$ e alto $h$, partendo da terra e arrivando fino alla cima del piano. Il lavoro della forza peso è:
  \begin{enumerate}[label=\Alph*.]
    \item $-mgh$
    \item $mgh$
    \item $mgh\cos\alpha$
    \item $mgh\sin\alpha$
  \end{enumerate}
  \item Un corpo sale da terra fino a un'altezza $h$. Il lavoro della forza peso è:
  \begin{enumerate}[label=\Alph*.]
    \item nullo
    \item negativo.
    \item positivo.
    \item $mgh$
  \end{enumerate}
  \item Sia \vec{F} una forza non conservativa. Allora
  \begin{enumerate}[label=\Alph*.]
    \item la sua energia potenziale non esiste.
    \item la sua energia potenziale è tale che lungo un percorso $\Delta U=-W$.
    \item la sua energia potenziale è sempre negativa.
    \item la sua energia potenziale è tale che lungo un percorso $\Delta U=W$.
  \end{enumerate}
  \item L'energia meccanica di un corpo che si trova a un'altezza $h$
  \begin{enumerate}[label=\Alph*.]
    \item è negativa
    \item è $mgh$.
    \item dipende dalla velocità del corpo.
    \item è nulla.
  \end{enumerate}
  \item Il lavoro della forza peso
  \begin{enumerate}[label=\Alph*.]
    \item è nullo solo se il percorso è chiuso.
    \item è sempre negativo.
    \item è nullo se il percorso ha gli estremi posti alla stessa altitudine.
    \item è sempre positivo.
  \end{enumerate}
\end{enumerate}








\newpage \maketitle \centering \textbf{Griglia di valutazione}. Eventuali mezzi punti saranno arrotondati all'intero precedente. \begin{table}[h]     \centering \begin{tabular}{|c|c|c|c|c|c|c|c|c|c|c|c|c|c|c|c|c|c|c|c|} \hline Punti &  $\leq 4$ & 5 & 6 & 7 & 8 & 9 & 10 & 11 & \textbf{12} & 13 & 14 & 15 & 16 & 17 & 18 & 19 & 20 \\ \hline Voto & 2 & 2.5 & 3 & 3.5 & 4 & 4.5 & 5 & 5.5 & \textbf{6} & 6.5 & 7 & 7.5 & 8 & 8.5 & 9 & 9.5 & 10 \\ \hline \end{tabular} \end{table}
\textbf{Codice}: qsixdvh xiqeqy ilrhgq


\begin{enumerate}
  \item Un corpo di massa $m$ scivola su un piano con attrito, partendo da una velocità v. Il coefficiente di attrito è \mu. Dopo quanto spazio si ferma?
  \begin{enumerate}[label=\Alph*.]
    \item $\frac{2v^2}{g\mu}}$.
    \item $\frac{v^2}{2g\mu}}$.
    \item $\frac{1}{2}v^2-\mu g$.
    \item $\frac{1}{2}v^2+\mu g$.
  \end{enumerate}
  \item Un corpo sale lungo un piano inclinato di un angolo $\alpha$ e alto $h$, partendo da terra e arrivando fino alla cima del piano. Il lavoro della forza peso è:
  \begin{enumerate}[label=\Alph*.]
    \item $mgh\sin\alpha$
    \item $-mgh$
    \item $mgh$
    \item $mgh\cos\alpha$
  \end{enumerate}
  \item Un corpo viene lasciato cadere da fermo da un altezza $h$. Quando raggiunge il suolo ha una velocità $v$. Quale relazione esprime correttamente la conservazione dell'energia?
  \begin{enumerate}[label=\Alph*.]
    \item $mgh=\frac{1}{2}mv^2$.
    \item $\frac{1}{2}mv^2+mgh=0.$
    \item $mgh+\frac{1}{2}mv^2=\frac{1}{2}mv^2.$
    \item $\frac{1}{2}mh^2=\frac{1}{2}mv^2$.
  \end{enumerate}
  \item Sia \vec{F} una forza non conservativa. Allora
  \begin{enumerate}[label=\Alph*.]
    \item la sua energia potenziale è sempre negativa.
    \item la sua energia potenziale è tale che lungo un percorso $\Delta U=W$.
    \item la sua energia potenziale non esiste.
    \item la sua energia potenziale è tale che lungo un percorso $\Delta U=-W$.
  \end{enumerate}
  \item Se calcoliamo il lavoro di una forza lungo un percorso chiuso e troviamo zero, cosa possiamo concludere?
  \begin{enumerate}[label=\Alph*.]
    \item la forza è nulla.
    \item la forza è conservativa.
    \item la forza non è conservativa.
    \item niente.
  \end{enumerate}
  \item Un corpo ha inizialmente una velocità $v$ e dopo un certo tempo si ferma. La variazione di energia cinetica è
  \begin{enumerate}[label=\Alph*.]
    \item dipende dallo spazio percorso.
    \item è positiva.
    \item è negativa
    \item dipende dal tempo in cui si ferma.
  \end{enumerate}
  \item Un corpo di massa $m_1$ e un corpo di massa $m_2$ sono lanciati in contemporanea da una torre alta $h$. Quali sono le velocità dei due corpi appena prima di toccare terra?
  \begin{enumerate}[label=\Alph*.]
    \item v_1=0, v_2=0
    \item $v_1=\sqrt{2m_1gh}, v_2=\sqrt{2m_2gh}$.
    \item $v_1=\sqrt{\frac{2gh}{m_1}}, v_2=\sqrt{\frac{2gh}{m_2}}$.
    \item $v_1=\sqrt{2gh}, v_2=\sqrt{2gh}$.
  \end{enumerate}
  \item Sia \vec{F} una forza conservativa. Allora
  \begin{enumerate}[label=\Alph*.]
    \item il lavoro di \vec{F} può essere nullo anche se il percorso non è chiuso.
    \item il lavoro di \vec{F} non dipende dal percorso considerato.
    \item il lavoro di \vec{F} è sempre nullo.
    \item il lavoro di \vec{F} è nullo solo se il percorso è chiuso.
  \end{enumerate}
  \item Il lavoro della forza peso
  \begin{enumerate}[label=\Alph*.]
    \item è sempre positivo.
    \item è nullo se il percorso ha gli estremi posti alla stessa altitudine.
    \item è sempre negativo.
    \item è nullo solo se il percorso è chiuso.
  \end{enumerate}
  \item Sia $E_m$ l'energia meccanica, $K$ l'energia cinetica, $U$ quella potenziale e $W_{NC}$ il lavoro delle forze non conservative agenti. Quale delle seguenti relazioni è vera?
  \begin{enumerate}[label=\Alph*.]
    \item $\Delta E_m=W_{NC}$.
    \item $\Delta K=W_{NC}.$
    \item $\Delta U=W_{NC}$.
    \item $\Delta K = \Delta U$.
  \end{enumerate}
  \item Sia \vec{F} una forza non conservativa. Allora
  \begin{enumerate}[label=\Alph*.]
    \item per ogni percorso aperto il lavoro è nullo.
    \item il lavoro non dipende dal percorso.
    \item il lavoro non dipende dagli estremi del percorso.
    \item esiste un percorso chiuso il cui lavoro non è nullo.
  \end{enumerate}
  \item Un corpo sale da terra fino a un'altezza $h$. Il lavoro della forza peso è:
  \begin{enumerate}[label=\Alph*.]
    \item $mgh$
    \item negativo.
    \item nullo
    \item positivo.
  \end{enumerate}
  \item Un corpo, partendo da fermo, rotola giù dalla cima di un piano inclinato di un angolo $\alpha$, alto $h$. Quale è la sua velocità quando arriva in fondo?
  \begin{enumerate}[label=\Alph*.]
    \item $\sqrt{2gh}\sin\alpha$.
    \item $\sqrt{2gh\sin\alpha}$.
    \item $\sqrt{2gh}$.
    \item $\sqrt{mgh\cos\alpha}$.
  \end{enumerate}
  \item Sia $E_m$ l'energia meccanica, $K$ l'energia cinetica, $U$ quella potenziale e $W$ il lavoro di tutte le forze agenti. Quale delle seguenti relazioni è vera?
  \begin{enumerate}[label=\Alph*.]
    \item $\Delta U=W$.
    \item $\Delta E_m=W$.
    \item $\Delta K=W.$
    \item $\Delta U=W$.
  \end{enumerate}
  \item Il lavoro della reazione vincolare
  \begin{enumerate}[label=\Alph*.]
    \item è sempre nullo.
    \item è nullo solo se il percorso è chiuso.
    \item può assumere qualsiasi valore.
    \item è sempre diverso da zero.
  \end{enumerate}
  \item L'energia meccanica di un corpo che si trova a un'altezza $h$
  \begin{enumerate}[label=\Alph*.]
    \item è $mgh$.
    \item è nulla.
    \item è negativa
    \item dipende dalla velocità del corpo.
  \end{enumerate}
  \item L'energia cinetica di un corpo fermo che si trova a un'altezza $h$
  \begin{enumerate}[label=\Alph*.]
    \item si conserva.
    \item è $mgh$.
    \item dipende da $h$
    \item è nulla.
  \end{enumerate}
  \item Un corpo sale da terra fino a un'altezza $h$. Il lavoro della forza peso è:
  \begin{enumerate}[label=\Alph*.]
    \item negativo.
    \item positivo.
    \item nullo
    \item $mgh$
  \end{enumerate}
  \item Sia \vec{F} una forza generica, che agisce su un corpo che effettua uno spostamento $\Delta \vec{s}$. Allora
  \begin{enumerate}[label=\Alph*.]
    \item il suo lavoro ha la stessa direzione della forza.
    \item il suo lavoro ha una direzione data dalla regola del parallelogramma.
    \item il suo lavoro ha la stessa direzione dello spostamento.
    \item il suo lavoro non ha una direzione perché è scalare.
  \end{enumerate}
  \item Il lavoro della forza d'attrito
  \begin{enumerate}[label=\Alph*.]
    \item non può essere calcolato perché la forza non è conservativa.
    \item è sempre negativo.
    \item è nullo se il percorso è rettilineo.
    \item è sempre positivo.
  \end{enumerate}
  \item Un corpo impatta una molla con una velocità $v$. La molla di costante elastica $k$, opponendosi all'avanzare del corpo, lo ferma. Quale è il suo allungamento nel momento in cui si ferma, trascurando l'attrito?
  \begin{enumerate}[label=\Alph*.]
    \item $\sqrt{\frac{m}{k}v}$.
    \item $\sqrt{\frac{k}{m}}v$.
    \item $\sqrt{\frac{m}{k}}v$.
    \item $\sqrt{\frac{k}{m}}v$.
  \end{enumerate}
\end{enumerate}








\newpage \maketitle \centering \textbf{Griglia di valutazione}. Eventuali mezzi punti saranno arrotondati all'intero precedente. \begin{table}[h]     \centering \begin{tabular}{|c|c|c|c|c|c|c|c|c|c|c|c|c|c|c|c|c|c|c|c|} \hline Punti &  $\leq 4$ & 5 & 6 & 7 & 8 & 9 & 10 & 11 & \textbf{12} & 13 & 14 & 15 & 16 & 17 & 18 & 19 & 20 \\ \hline Voto & 2 & 2.5 & 3 & 3.5 & 4 & 4.5 & 5 & 5.5 & \textbf{6} & 6.5 & 7 & 7.5 & 8 & 8.5 & 9 & 9.5 & 10 \\ \hline \end{tabular} \end{table}
\textbf{Codice}: pukwcvgbxlqgoz flrfip


\begin{enumerate}
  \item Sia \vec{F} una forza non conservativa. Allora
  \begin{enumerate}[label=\Alph*.]
    \item la sua energia potenziale non esiste.
    \item la sua energia potenziale è tale che lungo un percorso $\Delta U=-W$.
    \item la sua energia potenziale è tale che lungo un percorso $\Delta U=W$.
    \item la sua energia potenziale è sempre negativa.
  \end{enumerate}
  \item Un corpo sale lungo un piano inclinato di un angolo $\alpha$ e alto $h$, partendo da terra e arrivando fino alla cima del piano. Il lavoro della forza peso è:
  \begin{enumerate}[label=\Alph*.]
    \item $mgh\cos\alpha$
    \item $mgh$
    \item $mgh\sin\alpha$
    \item $-mgh$
  \end{enumerate}
  \item Un corpo, partendo da fermo, rotola giù dalla cima di un piano inclinato di un angolo $\alpha$, alto $h$. Quale è la sua velocità quando arriva in fondo?
  \begin{enumerate}[label=\Alph*.]
    \item $\sqrt{mgh\cos\alpha}$.
    \item $\sqrt{2gh\sin\alpha}$.
    \item $\sqrt{2gh}$.
    \item $\sqrt{2gh}\sin\alpha$.
  \end{enumerate}
  \item Il lavoro della forza peso
  \begin{enumerate}[label=\Alph*.]
    \item è sempre positivo.
    \item è nullo se il percorso ha gli estremi posti alla stessa altitudine.
    \item è sempre negativo.
    \item è nullo solo se il percorso è chiuso.
  \end{enumerate}
  \item Sia \vec{F} una forza generica, che agisce su un corpo che effettua uno spostamento $\Delta \vec{s}$. Allora
  \begin{enumerate}[label=\Alph*.]
    \item il suo lavoro ha la stessa direzione dello spostamento.
    \item il suo lavoro ha la stessa direzione della forza.
    \item il suo lavoro non ha una direzione perché è scalare.
    \item il suo lavoro ha una direzione data dalla regola del parallelogramma.
  \end{enumerate}
  \item Sia $E_m$ l'energia meccanica, $K$ l'energia cinetica, $U$ quella potenziale e $W_{NC}$ il lavoro delle forze non conservative agenti. Quale delle seguenti relazioni è vera?
  \begin{enumerate}[label=\Alph*.]
    \item $\Delta K=W_{NC}.$
    \item $\Delta K = \Delta U$.
    \item $\Delta E_m=W_{NC}$.
    \item $\Delta U=W_{NC}$.
  \end{enumerate}
  \item Se calcoliamo il lavoro di una forza lungo un percorso chiuso e troviamo zero, cosa possiamo concludere?
  \begin{enumerate}[label=\Alph*.]
    \item la forza è conservativa.
    \item la forza è nulla.
    \item niente.
    \item la forza non è conservativa.
  \end{enumerate}
  \item Un corpo sale da terra fino a un'altezza $h$. Il lavoro della forza peso è:
  \begin{enumerate}[label=\Alph*.]
    \item positivo.
    \item nullo
    \item negativo.
    \item $mgh$
  \end{enumerate}
  \item Il lavoro della reazione vincolare
  \begin{enumerate}[label=\Alph*.]
    \item è sempre diverso da zero.
    \item è sempre nullo.
    \item è nullo solo se il percorso è chiuso.
    \item può assumere qualsiasi valore.
  \end{enumerate}
  \item Un corpo impatta una molla con una velocità $v$. La molla di costante elastica $k$, opponendosi all'avanzare del corpo, lo ferma. Quale è il suo allungamento nel momento in cui si ferma, trascurando l'attrito?
  \begin{enumerate}[label=\Alph*.]
    \item $\sqrt{\frac{k}{m}}v$.
    \item $\sqrt{\frac{k}{m}}v$.
    \item $\sqrt{\frac{m}{k}v}$.
    \item $\sqrt{\frac{m}{k}}v$.
  \end{enumerate}
  \item Sia \vec{F} una forza conservativa. Allora
  \begin{enumerate}[label=\Alph*.]
    \item il lavoro di \vec{F} è sempre nullo.
    \item il lavoro di \vec{F} non dipende dal percorso considerato.
    \item il lavoro di \vec{F} è nullo solo se il percorso è chiuso.
    \item il lavoro di \vec{F} può essere nullo anche se il percorso non è chiuso.
  \end{enumerate}
  \item Sia $E_m$ l'energia meccanica, $K$ l'energia cinetica, $U$ quella potenziale e $W$ il lavoro di tutte le forze agenti. Quale delle seguenti relazioni è vera?
  \begin{enumerate}[label=\Alph*.]
    \item $\Delta U=W$.
    \item $\Delta U=W$.
    \item $\Delta E_m=W$.
    \item $\Delta K=W.$
  \end{enumerate}
  \item L'energia meccanica di un corpo che si trova a un'altezza $h$
  \begin{enumerate}[label=\Alph*.]
    \item dipende dalla velocità del corpo.
    \item è nulla.
    \item è $mgh$.
    \item è negativa
  \end{enumerate}
  \item Il lavoro della forza d'attrito
  \begin{enumerate}[label=\Alph*.]
    \item è nullo se il percorso è rettilineo.
    \item è sempre positivo.
    \item non può essere calcolato perché la forza non è conservativa.
    \item è sempre negativo.
  \end{enumerate}
  \item Un corpo sale da terra fino a un'altezza $h$. Il lavoro della forza peso è:
  \begin{enumerate}[label=\Alph*.]
    \item negativo.
    \item nullo
    \item $mgh$
    \item positivo.
  \end{enumerate}
  \item Un corpo di massa $m_1$ e un corpo di massa $m_2$ sono lanciati in contemporanea da una torre alta $h$. Quali sono le velocità dei due corpi appena prima di toccare terra?
  \begin{enumerate}[label=\Alph*.]
    \item $v_1=\sqrt{2gh}, v_2=\sqrt{2gh}$.
    \item $v_1=\sqrt{2m_1gh}, v_2=\sqrt{2m_2gh}$.
    \item v_1=0, v_2=0
    \item $v_1=\sqrt{\frac{2gh}{m_1}}, v_2=\sqrt{\frac{2gh}{m_2}}$.
  \end{enumerate}
  \item Un corpo viene lasciato cadere da fermo da un altezza $h$. Quando raggiunge il suolo ha una velocità $v$. Quale relazione esprime correttamente la conservazione dell'energia?
  \begin{enumerate}[label=\Alph*.]
    \item $\frac{1}{2}mv^2+mgh=0.$
    \item $mgh+\frac{1}{2}mv^2=\frac{1}{2}mv^2.$
    \item $\frac{1}{2}mh^2=\frac{1}{2}mv^2$.
    \item $mgh=\frac{1}{2}mv^2$.
  \end{enumerate}
  \item Sia \vec{F} una forza non conservativa. Allora
  \begin{enumerate}[label=\Alph*.]
    \item esiste un percorso chiuso il cui lavoro non è nullo.
    \item il lavoro non dipende dagli estremi del percorso.
    \item il lavoro non dipende dal percorso.
    \item per ogni percorso aperto il lavoro è nullo.
  \end{enumerate}
  \item L'energia cinetica di un corpo fermo che si trova a un'altezza $h$
  \begin{enumerate}[label=\Alph*.]
    \item dipende da $h$
    \item è nulla.
    \item è $mgh$.
    \item si conserva.
  \end{enumerate}
  \item Un corpo ha inizialmente una velocità $v$ e dopo un certo tempo si ferma. La variazione di energia cinetica è
  \begin{enumerate}[label=\Alph*.]
    \item dipende dal tempo in cui si ferma.
    \item dipende dallo spazio percorso.
    \item è positiva.
    \item è negativa
  \end{enumerate}
  \item Un corpo di massa $m$ scivola su un piano con attrito, partendo da una velocità v. Il coefficiente di attrito è \mu. Dopo quanto spazio si ferma?
  \begin{enumerate}[label=\Alph*.]
    \item $\frac{2v^2}{g\mu}}$.
    \item $\frac{v^2}{2g\mu}}$.
    \item $\frac{1}{2}v^2-\mu g$.
    \item $\frac{1}{2}v^2+\mu g$.
  \end{enumerate}
\end{enumerate}








\newpage \maketitle \centering \textbf{Griglia di valutazione}. Eventuali mezzi punti saranno arrotondati all'intero precedente. \begin{table}[h]     \centering \begin{tabular}{|c|c|c|c|c|c|c|c|c|c|c|c|c|c|c|c|c|c|c|c|} \hline Punti &  $\leq 4$ & 5 & 6 & 7 & 8 & 9 & 10 & 11 & \textbf{12} & 13 & 14 & 15 & 16 & 17 & 18 & 19 & 20 \\ \hline Voto & 2 & 2.5 & 3 & 3.5 & 4 & 4.5 & 5 & 5.5 & \textbf{6} & 6.5 & 7 & 7.5 & 8 & 8.5 & 9 & 9.5 & 10 \\ \hline \end{tabular} \end{table}
\textbf{Codice}: prlvbthcykqgpyagitehp


\begin{enumerate}
  \item Un corpo, partendo da fermo, rotola giù dalla cima di un piano inclinato di un angolo $\alpha$, alto $h$. Quale è la sua velocità quando arriva in fondo?
  \begin{enumerate}[label=\Alph*.]
    \item $\sqrt{2gh}$.
    \item $\sqrt{mgh\cos\alpha}$.
    \item $\sqrt{2gh}\sin\alpha$.
    \item $\sqrt{2gh\sin\alpha}$.
  \end{enumerate}
  \item Sia \vec{F} una forza conservativa. Allora
  \begin{enumerate}[label=\Alph*.]
    \item il lavoro di \vec{F} può essere nullo anche se il percorso non è chiuso.
    \item il lavoro di \vec{F} è sempre nullo.
    \item il lavoro di \vec{F} non dipende dal percorso considerato.
    \item il lavoro di \vec{F} è nullo solo se il percorso è chiuso.
  \end{enumerate}
  \item Un corpo ha inizialmente una velocità $v$ e dopo un certo tempo si ferma. La variazione di energia cinetica è
  \begin{enumerate}[label=\Alph*.]
    \item dipende dallo spazio percorso.
    \item è positiva.
    \item dipende dal tempo in cui si ferma.
    \item è negativa
  \end{enumerate}
  \item Sia $E_m$ l'energia meccanica, $K$ l'energia cinetica, $U$ quella potenziale e $W$ il lavoro di tutte le forze agenti. Quale delle seguenti relazioni è vera?
  \begin{enumerate}[label=\Alph*.]
    \item $\Delta K=W.$
    \item $\Delta U=W$.
    \item $\Delta E_m=W$.
    \item $\Delta U=W$.
  \end{enumerate}
  \item Il lavoro della reazione vincolare
  \begin{enumerate}[label=\Alph*.]
    \item è sempre diverso da zero.
    \item è sempre nullo.
    \item può assumere qualsiasi valore.
    \item è nullo solo se il percorso è chiuso.
  \end{enumerate}
  \item Un corpo sale da terra fino a un'altezza $h$. Il lavoro della forza peso è:
  \begin{enumerate}[label=\Alph*.]
    \item negativo.
    \item positivo.
    \item nullo
    \item $mgh$
  \end{enumerate}
  \item Sia $E_m$ l'energia meccanica, $K$ l'energia cinetica, $U$ quella potenziale e $W_{NC}$ il lavoro delle forze non conservative agenti. Quale delle seguenti relazioni è vera?
  \begin{enumerate}[label=\Alph*.]
    \item $\Delta U=W_{NC}$.
    \item $\Delta K=W_{NC}.$
    \item $\Delta K = \Delta U$.
    \item $\Delta E_m=W_{NC}$.
  \end{enumerate}
  \item Sia \vec{F} una forza non conservativa. Allora
  \begin{enumerate}[label=\Alph*.]
    \item il lavoro non dipende dagli estremi del percorso.
    \item per ogni percorso aperto il lavoro è nullo.
    \item il lavoro non dipende dal percorso.
    \item esiste un percorso chiuso il cui lavoro non è nullo.
  \end{enumerate}
  \item Un corpo sale lungo un piano inclinato di un angolo $\alpha$ e alto $h$, partendo da terra e arrivando fino alla cima del piano. Il lavoro della forza peso è:
  \begin{enumerate}[label=\Alph*.]
    \item $mgh\sin\alpha$
    \item $mgh\cos\alpha$
    \item $-mgh$
    \item $mgh$
  \end{enumerate}
  \item L'energia meccanica di un corpo che si trova a un'altezza $h$
  \begin{enumerate}[label=\Alph*.]
    \item è nulla.
    \item è $mgh$.
    \item dipende dalla velocità del corpo.
    \item è negativa
  \end{enumerate}
  \item Sia \vec{F} una forza generica, che agisce su un corpo che effettua uno spostamento $\Delta \vec{s}$. Allora
  \begin{enumerate}[label=\Alph*.]
    \item il suo lavoro ha una direzione data dalla regola del parallelogramma.
    \item il suo lavoro ha la stessa direzione dello spostamento.
    \item il suo lavoro ha la stessa direzione della forza.
    \item il suo lavoro non ha una direzione perché è scalare.
  \end{enumerate}
  \item Se calcoliamo il lavoro di una forza lungo un percorso chiuso e troviamo zero, cosa possiamo concludere?
  \begin{enumerate}[label=\Alph*.]
    \item la forza è conservativa.
    \item la forza è nulla.
    \item la forza non è conservativa.
    \item niente.
  \end{enumerate}
  \item Un corpo viene lasciato cadere da fermo da un altezza $h$. Quando raggiunge il suolo ha una velocità $v$. Quale relazione esprime correttamente la conservazione dell'energia?
  \begin{enumerate}[label=\Alph*.]
    \item $mgh+\frac{1}{2}mv^2=\frac{1}{2}mv^2.$
    \item $mgh=\frac{1}{2}mv^2$.
    \item $\frac{1}{2}mh^2=\frac{1}{2}mv^2$.
    \item $\frac{1}{2}mv^2+mgh=0.$
  \end{enumerate}
  \item Il lavoro della forza peso
  \begin{enumerate}[label=\Alph*.]
    \item è sempre positivo.
    \item è nullo solo se il percorso è chiuso.
    \item è nullo se il percorso ha gli estremi posti alla stessa altitudine.
    \item è sempre negativo.
  \end{enumerate}
  \item Il lavoro della forza d'attrito
  \begin{enumerate}[label=\Alph*.]
    \item è nullo se il percorso è rettilineo.
    \item è sempre negativo.
    \item non può essere calcolato perché la forza non è conservativa.
    \item è sempre positivo.
  \end{enumerate}
  \item Un corpo sale da terra fino a un'altezza $h$. Il lavoro della forza peso è:
  \begin{enumerate}[label=\Alph*.]
    \item positivo.
    \item negativo.
    \item nullo
    \item $mgh$
  \end{enumerate}
  \item Un corpo di massa $m_1$ e un corpo di massa $m_2$ sono lanciati in contemporanea da una torre alta $h$. Quali sono le velocità dei due corpi appena prima di toccare terra?
  \begin{enumerate}[label=\Alph*.]
    \item $v_1=\sqrt{2gh}, v_2=\sqrt{2gh}$.
    \item $v_1=\sqrt{\frac{2gh}{m_1}}, v_2=\sqrt{\frac{2gh}{m_2}}$.
    \item $v_1=\sqrt{2m_1gh}, v_2=\sqrt{2m_2gh}$.
    \item v_1=0, v_2=0
  \end{enumerate}
  \item Sia \vec{F} una forza non conservativa. Allora
  \begin{enumerate}[label=\Alph*.]
    \item la sua energia potenziale è sempre negativa.
    \item la sua energia potenziale è tale che lungo un percorso $\Delta U=W$.
    \item la sua energia potenziale non esiste.
    \item la sua energia potenziale è tale che lungo un percorso $\Delta U=-W$.
  \end{enumerate}
  \item Un corpo impatta una molla con una velocità $v$. La molla di costante elastica $k$, opponendosi all'avanzare del corpo, lo ferma. Quale è il suo allungamento nel momento in cui si ferma, trascurando l'attrito?
  \begin{enumerate}[label=\Alph*.]
    \item $\sqrt{\frac{m}{k}}v$.
    \item $\sqrt{\frac{k}{m}}v$.
    \item $\sqrt{\frac{m}{k}v}$.
    \item $\sqrt{\frac{k}{m}}v$.
  \end{enumerate}
  \item Un corpo di massa $m$ scivola su un piano con attrito, partendo da una velocità v. Il coefficiente di attrito è \mu. Dopo quanto spazio si ferma?
  \begin{enumerate}[label=\Alph*.]
    \item $\frac{2v^2}{g\mu}}$.
    \item $\frac{1}{2}v^2+\mu g$.
    \item $\frac{v^2}{2g\mu}}$.
    \item $\frac{1}{2}v^2-\mu g$.
  \end{enumerate}
  \item L'energia cinetica di un corpo fermo che si trova a un'altezza $h$
  \begin{enumerate}[label=\Alph*.]
    \item si conserva.
    \item è nulla.
    \item dipende da $h$
    \item è $mgh$.
  \end{enumerate}
\end{enumerate}








\newpage \maketitle \centering \textbf{Griglia di valutazione}. Eventuali mezzi punti saranno arrotondati all'intero precedente. \begin{table}[h]     \centering \begin{tabular}{|c|c|c|c|c|c|c|c|c|c|c|c|c|c|c|c|c|c|c|c|} \hline Punti &  $\leq 4$ & 5 & 6 & 7 & 8 & 9 & 10 & 11 & \textbf{12} & 13 & 14 & 15 & 16 & 17 & 18 & 19 & 20 \\ \hline Voto & 2 & 2.5 & 3 & 3.5 & 4 & 4.5 & 5 & 5.5 & \textbf{6} & 6.5 & 7 & 7.5 & 8 & 8.5 & 9 & 9.5 & 10 \\ \hline \end{tabular} \end{table}
\textbf{Codice}: qskxbvfbzkofpxciktefo


\begin{enumerate}
  \item Un corpo impatta una molla con una velocità $v$. La molla di costante elastica $k$, opponendosi all'avanzare del corpo, lo ferma. Quale è il suo allungamento nel momento in cui si ferma, trascurando l'attrito?
  \begin{enumerate}[label=\Alph*.]
    \item $\sqrt{\frac{k}{m}}v$.
    \item $\sqrt{\frac{m}{k}}v$.
    \item $\sqrt{\frac{k}{m}}v$.
    \item $\sqrt{\frac{m}{k}v}$.
  \end{enumerate}
  \item Se calcoliamo il lavoro di una forza lungo un percorso chiuso e troviamo zero, cosa possiamo concludere?
  \begin{enumerate}[label=\Alph*.]
    \item la forza è conservativa.
    \item niente.
    \item la forza è nulla.
    \item la forza non è conservativa.
  \end{enumerate}
  \item Sia \vec{F} una forza non conservativa. Allora
  \begin{enumerate}[label=\Alph*.]
    \item la sua energia potenziale è tale che lungo un percorso $\Delta U=-W$.
    \item la sua energia potenziale è sempre negativa.
    \item la sua energia potenziale non esiste.
    \item la sua energia potenziale è tale che lungo un percorso $\Delta U=W$.
  \end{enumerate}
  \item Sia \vec{F} una forza generica, che agisce su un corpo che effettua uno spostamento $\Delta \vec{s}$. Allora
  \begin{enumerate}[label=\Alph*.]
    \item il suo lavoro ha una direzione data dalla regola del parallelogramma.
    \item il suo lavoro ha la stessa direzione della forza.
    \item il suo lavoro non ha una direzione perché è scalare.
    \item il suo lavoro ha la stessa direzione dello spostamento.
  \end{enumerate}
  \item Il lavoro della forza peso
  \begin{enumerate}[label=\Alph*.]
    \item è sempre negativo.
    \item è nullo se il percorso ha gli estremi posti alla stessa altitudine.
    \item è sempre positivo.
    \item è nullo solo se il percorso è chiuso.
  \end{enumerate}
  \item Un corpo sale da terra fino a un'altezza $h$. Il lavoro della forza peso è:
  \begin{enumerate}[label=\Alph*.]
    \item positivo.
    \item $mgh$
    \item negativo.
    \item nullo
  \end{enumerate}
  \item Un corpo di massa $m$ scivola su un piano con attrito, partendo da una velocità v. Il coefficiente di attrito è \mu. Dopo quanto spazio si ferma?
  \begin{enumerate}[label=\Alph*.]
    \item $\frac{1}{2}v^2+\mu g$.
    \item $\frac{v^2}{2g\mu}}$.
    \item $\frac{1}{2}v^2-\mu g$.
    \item $\frac{2v^2}{g\mu}}$.
  \end{enumerate}
  \item L'energia cinetica di un corpo fermo che si trova a un'altezza $h$
  \begin{enumerate}[label=\Alph*.]
    \item è $mgh$.
    \item dipende da $h$
    \item è nulla.
    \item si conserva.
  \end{enumerate}
  \item Un corpo sale da terra fino a un'altezza $h$. Il lavoro della forza peso è:
  \begin{enumerate}[label=\Alph*.]
    \item positivo.
    \item $mgh$
    \item nullo
    \item negativo.
  \end{enumerate}
  \item Sia \vec{F} una forza non conservativa. Allora
  \begin{enumerate}[label=\Alph*.]
    \item per ogni percorso aperto il lavoro è nullo.
    \item il lavoro non dipende dal percorso.
    \item esiste un percorso chiuso il cui lavoro non è nullo.
    \item il lavoro non dipende dagli estremi del percorso.
  \end{enumerate}
  \item Un corpo, partendo da fermo, rotola giù dalla cima di un piano inclinato di un angolo $\alpha$, alto $h$. Quale è la sua velocità quando arriva in fondo?
  \begin{enumerate}[label=\Alph*.]
    \item $\sqrt{2gh\sin\alpha}$.
    \item $\sqrt{2gh}$.
    \item $\sqrt{mgh\cos\alpha}$.
    \item $\sqrt{2gh}\sin\alpha$.
  \end{enumerate}
  \item L'energia meccanica di un corpo che si trova a un'altezza $h$
  \begin{enumerate}[label=\Alph*.]
    \item è nulla.
    \item è $mgh$.
    \item dipende dalla velocità del corpo.
    \item è negativa
  \end{enumerate}
  \item Un corpo di massa $m_1$ e un corpo di massa $m_2$ sono lanciati in contemporanea da una torre alta $h$. Quali sono le velocità dei due corpi appena prima di toccare terra?
  \begin{enumerate}[label=\Alph*.]
    \item $v_1=\sqrt{2m_1gh}, v_2=\sqrt{2m_2gh}$.
    \item $v_1=\sqrt{2gh}, v_2=\sqrt{2gh}$.
    \item v_1=0, v_2=0
    \item $v_1=\sqrt{\frac{2gh}{m_1}}, v_2=\sqrt{\frac{2gh}{m_2}}$.
  \end{enumerate}
  \item Un corpo viene lasciato cadere da fermo da un altezza $h$. Quando raggiunge il suolo ha una velocità $v$. Quale relazione esprime correttamente la conservazione dell'energia?
  \begin{enumerate}[label=\Alph*.]
    \item $\frac{1}{2}mv^2+mgh=0.$
    \item $mgh=\frac{1}{2}mv^2$.
    \item $\frac{1}{2}mh^2=\frac{1}{2}mv^2$.
    \item $mgh+\frac{1}{2}mv^2=\frac{1}{2}mv^2.$
  \end{enumerate}
  \item Sia \vec{F} una forza conservativa. Allora
  \begin{enumerate}[label=\Alph*.]
    \item il lavoro di \vec{F} non dipende dal percorso considerato.
    \item il lavoro di \vec{F} è sempre nullo.
    \item il lavoro di \vec{F} è nullo solo se il percorso è chiuso.
    \item il lavoro di \vec{F} può essere nullo anche se il percorso non è chiuso.
  \end{enumerate}
  \item Sia $E_m$ l'energia meccanica, $K$ l'energia cinetica, $U$ quella potenziale e $W_{NC}$ il lavoro delle forze non conservative agenti. Quale delle seguenti relazioni è vera?
  \begin{enumerate}[label=\Alph*.]
    \item $\Delta U=W_{NC}$.
    \item $\Delta K = \Delta U$.
    \item $\Delta K=W_{NC}.$
    \item $\Delta E_m=W_{NC}$.
  \end{enumerate}
  \item Il lavoro della forza d'attrito
  \begin{enumerate}[label=\Alph*.]
    \item non può essere calcolato perché la forza non è conservativa.
    \item è nullo se il percorso è rettilineo.
    \item è sempre negativo.
    \item è sempre positivo.
  \end{enumerate}
  \item Un corpo sale lungo un piano inclinato di un angolo $\alpha$ e alto $h$, partendo da terra e arrivando fino alla cima del piano. Il lavoro della forza peso è:
  \begin{enumerate}[label=\Alph*.]
    \item $mgh\sin\alpha$
    \item $mgh$
    \item $-mgh$
    \item $mgh\cos\alpha$
  \end{enumerate}
  \item Il lavoro della reazione vincolare
  \begin{enumerate}[label=\Alph*.]
    \item è sempre nullo.
    \item è nullo solo se il percorso è chiuso.
    \item può assumere qualsiasi valore.
    \item è sempre diverso da zero.
  \end{enumerate}
  \item Un corpo ha inizialmente una velocità $v$ e dopo un certo tempo si ferma. La variazione di energia cinetica è
  \begin{enumerate}[label=\Alph*.]
    \item è negativa
    \item dipende dal tempo in cui si ferma.
    \item dipende dallo spazio percorso.
    \item è positiva.
  \end{enumerate}
  \item Sia $E_m$ l'energia meccanica, $K$ l'energia cinetica, $U$ quella potenziale e $W$ il lavoro di tutte le forze agenti. Quale delle seguenti relazioni è vera?
  \begin{enumerate}[label=\Alph*.]
    \item $\Delta K=W.$
    \item $\Delta U=W$.
    \item $\Delta U=W$.
    \item $\Delta E_m=W$.
  \end{enumerate}
\end{enumerate}








\newpage \maketitle \centering \textbf{Griglia di valutazione}. Eventuali mezzi punti saranno arrotondati all'intero precedente. \begin{table}[h]     \centering \begin{tabular}{|c|c|c|c|c|c|c|c|c|c|c|c|c|c|c|c|c|c|c|c|} \hline Punti &  $\leq 4$ & 5 & 6 & 7 & 8 & 9 & 10 & 11 & \textbf{12} & 13 & 14 & 15 & 16 & 17 & 18 & 19 & 20 \\ \hline Voto & 2 & 2.5 & 3 & 3.5 & 4 & 4.5 & 5 & 5.5 & \textbf{6} & 6.5 & 7 & 7.5 & 8 & 8.5 & 9 & 9.5 & 10 \\ \hline \end{tabular} \end{table}
\textbf{Codice}: rtixbwh zlodpxbglsegq


\begin{enumerate}
  \item Un corpo viene lasciato cadere da fermo da un altezza $h$. Quando raggiunge il suolo ha una velocità $v$. Quale relazione esprime correttamente la conservazione dell'energia?
  \begin{enumerate}[label=\Alph*.]
    \item $\frac{1}{2}mh^2=\frac{1}{2}mv^2$.
    \item $\frac{1}{2}mv^2+mgh=0.$
    \item $mgh=\frac{1}{2}mv^2$.
    \item $mgh+\frac{1}{2}mv^2=\frac{1}{2}mv^2.$
  \end{enumerate}
  \item Sia $E_m$ l'energia meccanica, $K$ l'energia cinetica, $U$ quella potenziale e $W$ il lavoro di tutte le forze agenti. Quale delle seguenti relazioni è vera?
  \begin{enumerate}[label=\Alph*.]
    \item $\Delta U=W$.
    \item $\Delta E_m=W$.
    \item $\Delta K=W.$
    \item $\Delta U=W$.
  \end{enumerate}
  \item Un corpo sale lungo un piano inclinato di un angolo $\alpha$ e alto $h$, partendo da terra e arrivando fino alla cima del piano. Il lavoro della forza peso è:
  \begin{enumerate}[label=\Alph*.]
    \item $-mgh$
    \item $mgh$
    \item $mgh\cos\alpha$
    \item $mgh\sin\alpha$
  \end{enumerate}
  \item Se calcoliamo il lavoro di una forza lungo un percorso chiuso e troviamo zero, cosa possiamo concludere?
  \begin{enumerate}[label=\Alph*.]
    \item la forza non è conservativa.
    \item la forza è nulla.
    \item niente.
    \item la forza è conservativa.
  \end{enumerate}
  \item Un corpo sale da terra fino a un'altezza $h$. Il lavoro della forza peso è:
  \begin{enumerate}[label=\Alph*.]
    \item positivo.
    \item negativo.
    \item $mgh$
    \item nullo
  \end{enumerate}
  \item L'energia meccanica di un corpo che si trova a un'altezza $h$
  \begin{enumerate}[label=\Alph*.]
    \item è negativa
    \item è nulla.
    \item è $mgh$.
    \item dipende dalla velocità del corpo.
  \end{enumerate}
  \item Sia \vec{F} una forza generica, che agisce su un corpo che effettua uno spostamento $\Delta \vec{s}$. Allora
  \begin{enumerate}[label=\Alph*.]
    \item il suo lavoro ha la stessa direzione della forza.
    \item il suo lavoro ha la stessa direzione dello spostamento.
    \item il suo lavoro ha una direzione data dalla regola del parallelogramma.
    \item il suo lavoro non ha una direzione perché è scalare.
  \end{enumerate}
  \item L'energia cinetica di un corpo fermo che si trova a un'altezza $h$
  \begin{enumerate}[label=\Alph*.]
    \item è nulla.
    \item dipende da $h$
    \item è $mgh$.
    \item si conserva.
  \end{enumerate}
  \item Sia $E_m$ l'energia meccanica, $K$ l'energia cinetica, $U$ quella potenziale e $W_{NC}$ il lavoro delle forze non conservative agenti. Quale delle seguenti relazioni è vera?
  \begin{enumerate}[label=\Alph*.]
    \item $\Delta K=W_{NC}.$
    \item $\Delta U=W_{NC}$.
    \item $\Delta K = \Delta U$.
    \item $\Delta E_m=W_{NC}$.
  \end{enumerate}
  \item Il lavoro della reazione vincolare
  \begin{enumerate}[label=\Alph*.]
    \item è nullo solo se il percorso è chiuso.
    \item è sempre diverso da zero.
    \item può assumere qualsiasi valore.
    \item è sempre nullo.
  \end{enumerate}
  \item Un corpo impatta una molla con una velocità $v$. La molla di costante elastica $k$, opponendosi all'avanzare del corpo, lo ferma. Quale è il suo allungamento nel momento in cui si ferma, trascurando l'attrito?
  \begin{enumerate}[label=\Alph*.]
    \item $\sqrt{\frac{m}{k}v}$.
    \item $\sqrt{\frac{m}{k}}v$.
    \item $\sqrt{\frac{k}{m}}v$.
    \item $\sqrt{\frac{k}{m}}v$.
  \end{enumerate}
  \item Il lavoro della forza peso
  \begin{enumerate}[label=\Alph*.]
    \item è nullo se il percorso ha gli estremi posti alla stessa altitudine.
    \item è sempre negativo.
    \item è nullo solo se il percorso è chiuso.
    \item è sempre positivo.
  \end{enumerate}
  \item Sia \vec{F} una forza non conservativa. Allora
  \begin{enumerate}[label=\Alph*.]
    \item il lavoro non dipende dagli estremi del percorso.
    \item esiste un percorso chiuso il cui lavoro non è nullo.
    \item il lavoro non dipende dal percorso.
    \item per ogni percorso aperto il lavoro è nullo.
  \end{enumerate}
  \item Sia \vec{F} una forza non conservativa. Allora
  \begin{enumerate}[label=\Alph*.]
    \item la sua energia potenziale è tale che lungo un percorso $\Delta U=W$.
    \item la sua energia potenziale non esiste.
    \item la sua energia potenziale è sempre negativa.
    \item la sua energia potenziale è tale che lungo un percorso $\Delta U=-W$.
  \end{enumerate}
  \item Un corpo di massa $m$ scivola su un piano con attrito, partendo da una velocità v. Il coefficiente di attrito è \mu. Dopo quanto spazio si ferma?
  \begin{enumerate}[label=\Alph*.]
    \item $\frac{1}{2}v^2+\mu g$.
    \item $\frac{1}{2}v^2-\mu g$.
    \item $\frac{v^2}{2g\mu}}$.
    \item $\frac{2v^2}{g\mu}}$.
  \end{enumerate}
  \item Un corpo ha inizialmente una velocità $v$ e dopo un certo tempo si ferma. La variazione di energia cinetica è
  \begin{enumerate}[label=\Alph*.]
    \item dipende dal tempo in cui si ferma.
    \item è negativa
    \item dipende dallo spazio percorso.
    \item è positiva.
  \end{enumerate}
  \item Un corpo, partendo da fermo, rotola giù dalla cima di un piano inclinato di un angolo $\alpha$, alto $h$. Quale è la sua velocità quando arriva in fondo?
  \begin{enumerate}[label=\Alph*.]
    \item $\sqrt{2gh\sin\alpha}$.
    \item $\sqrt{2gh}\sin\alpha$.
    \item $\sqrt{mgh\cos\alpha}$.
    \item $\sqrt{2gh}$.
  \end{enumerate}
  \item Un corpo sale da terra fino a un'altezza $h$. Il lavoro della forza peso è:
  \begin{enumerate}[label=\Alph*.]
    \item nullo
    \item negativo.
    \item positivo.
    \item $mgh$
  \end{enumerate}
  \item Sia \vec{F} una forza conservativa. Allora
  \begin{enumerate}[label=\Alph*.]
    \item il lavoro di \vec{F} può essere nullo anche se il percorso non è chiuso.
    \item il lavoro di \vec{F} è nullo solo se il percorso è chiuso.
    \item il lavoro di \vec{F} non dipende dal percorso considerato.
    \item il lavoro di \vec{F} è sempre nullo.
  \end{enumerate}
  \item Il lavoro della forza d'attrito
  \begin{enumerate}[label=\Alph*.]
    \item è nullo se il percorso è rettilineo.
    \item è sempre negativo.
    \item non può essere calcolato perché la forza non è conservativa.
    \item è sempre positivo.
  \end{enumerate}
  \item Un corpo di massa $m_1$ e un corpo di massa $m_2$ sono lanciati in contemporanea da una torre alta $h$. Quali sono le velocità dei due corpi appena prima di toccare terra?
  \begin{enumerate}[label=\Alph*.]
    \item v_1=0, v_2=0
    \item $v_1=\sqrt{\frac{2gh}{m_1}}, v_2=\sqrt{\frac{2gh}{m_2}}$.
    \item $v_1=\sqrt{2gh}, v_2=\sqrt{2gh}$.
    \item $v_1=\sqrt{2m_1gh}, v_2=\sqrt{2m_2gh}$.
  \end{enumerate}
\end{enumerate}








\newpage \maketitle \centering \textbf{Griglia di valutazione}. Eventuali mezzi punti saranno arrotondati all'intero precedente. \begin{table}[h]     \centering \begin{tabular}{|c|c|c|c|c|c|c|c|c|c|c|c|c|c|c|c|c|c|c|c|} \hline Punti &  $\leq 4$ & 5 & 6 & 7 & 8 & 9 & 10 & 11 & \textbf{12} & 13 & 14 & 15 & 16 & 17 & 18 & 19 & 20 \\ \hline Voto & 2 & 2.5 & 3 & 3.5 & 4 & 4.5 & 5 & 5.5 & \textbf{6} & 6.5 & 7 & 7.5 & 8 & 8.5 & 9 & 9.5 & 10 \\ \hline \end{tabular} \end{table}
\textbf{Codice}: prkwatgbyipfqyaflrffq


\begin{enumerate}
  \item Un corpo impatta una molla con una velocità $v$. La molla di costante elastica $k$, opponendosi all'avanzare del corpo, lo ferma. Quale è il suo allungamento nel momento in cui si ferma, trascurando l'attrito?
  \begin{enumerate}[label=\Alph*.]
    \item $\sqrt{\frac{m}{k}}v$.
    \item $\sqrt{\frac{m}{k}v}$.
    \item $\sqrt{\frac{k}{m}}v$.
    \item $\sqrt{\frac{k}{m}}v$.
  \end{enumerate}
  \item Se calcoliamo il lavoro di una forza lungo un percorso chiuso e troviamo zero, cosa possiamo concludere?
  \begin{enumerate}[label=\Alph*.]
    \item niente.
    \item la forza non è conservativa.
    \item la forza è nulla.
    \item la forza è conservativa.
  \end{enumerate}
  \item Un corpo di massa $m_1$ e un corpo di massa $m_2$ sono lanciati in contemporanea da una torre alta $h$. Quali sono le velocità dei due corpi appena prima di toccare terra?
  \begin{enumerate}[label=\Alph*.]
    \item v_1=0, v_2=0
    \item $v_1=\sqrt{\frac{2gh}{m_1}}, v_2=\sqrt{\frac{2gh}{m_2}}$.
    \item $v_1=\sqrt{2gh}, v_2=\sqrt{2gh}$.
    \item $v_1=\sqrt{2m_1gh}, v_2=\sqrt{2m_2gh}$.
  \end{enumerate}
  \item Il lavoro della forza peso
  \begin{enumerate}[label=\Alph*.]
    \item è sempre positivo.
    \item è nullo se il percorso ha gli estremi posti alla stessa altitudine.
    \item è sempre negativo.
    \item è nullo solo se il percorso è chiuso.
  \end{enumerate}
  \item Sia \vec{F} una forza non conservativa. Allora
  \begin{enumerate}[label=\Alph*.]
    \item esiste un percorso chiuso il cui lavoro non è nullo.
    \item il lavoro non dipende dal percorso.
    \item il lavoro non dipende dagli estremi del percorso.
    \item per ogni percorso aperto il lavoro è nullo.
  \end{enumerate}
  \item Un corpo sale da terra fino a un'altezza $h$. Il lavoro della forza peso è:
  \begin{enumerate}[label=\Alph*.]
    \item negativo.
    \item $mgh$
    \item nullo
    \item positivo.
  \end{enumerate}
  \item Sia \vec{F} una forza generica, che agisce su un corpo che effettua uno spostamento $\Delta \vec{s}$. Allora
  \begin{enumerate}[label=\Alph*.]
    \item il suo lavoro ha la stessa direzione dello spostamento.
    \item il suo lavoro ha una direzione data dalla regola del parallelogramma.
    \item il suo lavoro non ha una direzione perché è scalare.
    \item il suo lavoro ha la stessa direzione della forza.
  \end{enumerate}
  \item Un corpo di massa $m$ scivola su un piano con attrito, partendo da una velocità v. Il coefficiente di attrito è \mu. Dopo quanto spazio si ferma?
  \begin{enumerate}[label=\Alph*.]
    \item $\frac{1}{2}v^2+\mu g$.
    \item $\frac{2v^2}{g\mu}}$.
    \item $\frac{v^2}{2g\mu}}$.
    \item $\frac{1}{2}v^2-\mu g$.
  \end{enumerate}
  \item L'energia meccanica di un corpo che si trova a un'altezza $h$
  \begin{enumerate}[label=\Alph*.]
    \item è negativa
    \item è $mgh$.
    \item dipende dalla velocità del corpo.
    \item è nulla.
  \end{enumerate}
  \item Sia $E_m$ l'energia meccanica, $K$ l'energia cinetica, $U$ quella potenziale e $W_{NC}$ il lavoro delle forze non conservative agenti. Quale delle seguenti relazioni è vera?
  \begin{enumerate}[label=\Alph*.]
    \item $\Delta E_m=W_{NC}$.
    \item $\Delta K = \Delta U$.
    \item $\Delta U=W_{NC}$.
    \item $\Delta K=W_{NC}.$
  \end{enumerate}
  \item L'energia cinetica di un corpo fermo che si trova a un'altezza $h$
  \begin{enumerate}[label=\Alph*.]
    \item dipende da $h$
    \item è $mgh$.
    \item è nulla.
    \item si conserva.
  \end{enumerate}
  \item Sia \vec{F} una forza conservativa. Allora
  \begin{enumerate}[label=\Alph*.]
    \item il lavoro di \vec{F} è nullo solo se il percorso è chiuso.
    \item il lavoro di \vec{F} non dipende dal percorso considerato.
    \item il lavoro di \vec{F} può essere nullo anche se il percorso non è chiuso.
    \item il lavoro di \vec{F} è sempre nullo.
  \end{enumerate}
  \item Un corpo sale da terra fino a un'altezza $h$. Il lavoro della forza peso è:
  \begin{enumerate}[label=\Alph*.]
    \item positivo.
    \item nullo
    \item negativo.
    \item $mgh$
  \end{enumerate}
  \item Un corpo ha inizialmente una velocità $v$ e dopo un certo tempo si ferma. La variazione di energia cinetica è
  \begin{enumerate}[label=\Alph*.]
    \item è positiva.
    \item dipende dallo spazio percorso.
    \item è negativa
    \item dipende dal tempo in cui si ferma.
  \end{enumerate}
  \item Sia \vec{F} una forza non conservativa. Allora
  \begin{enumerate}[label=\Alph*.]
    \item la sua energia potenziale è tale che lungo un percorso $\Delta U=W$.
    \item la sua energia potenziale non esiste.
    \item la sua energia potenziale è sempre negativa.
    \item la sua energia potenziale è tale che lungo un percorso $\Delta U=-W$.
  \end{enumerate}
  \item Un corpo sale lungo un piano inclinato di un angolo $\alpha$ e alto $h$, partendo da terra e arrivando fino alla cima del piano. Il lavoro della forza peso è:
  \begin{enumerate}[label=\Alph*.]
    \item $-mgh$
    \item $mgh\sin\alpha$
    \item $mgh$
    \item $mgh\cos\alpha$
  \end{enumerate}
  \item Un corpo, partendo da fermo, rotola giù dalla cima di un piano inclinato di un angolo $\alpha$, alto $h$. Quale è la sua velocità quando arriva in fondo?
  \begin{enumerate}[label=\Alph*.]
    \item $\sqrt{2gh\sin\alpha}$.
    \item $\sqrt{mgh\cos\alpha}$.
    \item $\sqrt{2gh}\sin\alpha$.
    \item $\sqrt{2gh}$.
  \end{enumerate}
  \item Sia $E_m$ l'energia meccanica, $K$ l'energia cinetica, $U$ quella potenziale e $W$ il lavoro di tutte le forze agenti. Quale delle seguenti relazioni è vera?
  \begin{enumerate}[label=\Alph*.]
    \item $\Delta K=W.$
    \item $\Delta U=W$.
    \item $\Delta U=W$.
    \item $\Delta E_m=W$.
  \end{enumerate}
  \item Il lavoro della reazione vincolare
  \begin{enumerate}[label=\Alph*.]
    \item può assumere qualsiasi valore.
    \item è sempre nullo.
    \item è nullo solo se il percorso è chiuso.
    \item è sempre diverso da zero.
  \end{enumerate}
  \item Il lavoro della forza d'attrito
  \begin{enumerate}[label=\Alph*.]
    \item è sempre negativo.
    \item non può essere calcolato perché la forza non è conservativa.
    \item è nullo se il percorso è rettilineo.
    \item è sempre positivo.
  \end{enumerate}
  \item Un corpo viene lasciato cadere da fermo da un altezza $h$. Quando raggiunge il suolo ha una velocità $v$. Quale relazione esprime correttamente la conservazione dell'energia?
  \begin{enumerate}[label=\Alph*.]
    \item $\frac{1}{2}mv^2+mgh=0.$
    \item $\frac{1}{2}mh^2=\frac{1}{2}mv^2$.
    \item $mgh=\frac{1}{2}mv^2$.
    \item $mgh+\frac{1}{2}mv^2=\frac{1}{2}mv^2.$
  \end{enumerate}
\end{enumerate}








\newpage \maketitle \centering \textbf{Griglia di valutazione}. Eventuali mezzi punti saranno arrotondati all'intero precedente. \begin{table}[h]     \centering \begin{tabular}{|c|c|c|c|c|c|c|c|c|c|c|c|c|c|c|c|c|c|c|c|} \hline Punti &  $\leq 4$ & 5 & 6 & 7 & 8 & 9 & 10 & 11 & \textbf{12} & 13 & 14 & 15 & 16 & 17 & 18 & 19 & 20 \\ \hline Voto & 2 & 2.5 & 3 & 3.5 & 4 & 4.5 & 5 & 5.5 & \textbf{6} & 6.5 & 7 & 7.5 & 8 & 8.5 & 9 & 9.5 & 10 \\ \hline \end{tabular} \end{table}
Le risposte corrette erano  bad
Il voto è:  0.0


\end{document}
