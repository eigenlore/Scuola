\documentclass{article}
\usepackage{graphicx, amsmath, amsthm, float} % Required for inserting images
\theoremstyle{definition}
\newtheorem*{definition}{Definizione}
\newtheorem*{example}{Esempio}
\setlength{\parindent}{0pt}
\title{Moti nel piano}
\author{Lorenzo Tasca}
\date{Settembre 2024}

\begin{document}

\maketitle

\section{Moto armonico}
\subsection{Moto armonico 1D}
\begin{definition}
Un moto si dice armonico se l'accelerazione è proporzionale alla posizione, ossia se esiste una costante $\alpha$ tale che $$a=\alpha x.$$
\end{definition}
\begin{example}
Consideriamo una massa $m$ che si muove attaccata a una molla di costante elastica $k$. Le equazioni del moto sono $$F=ma.$$ Dalla legge di Hooke sappiamo che $$F=-kx,$$ che sostituendo fornisce $$-kx=ma,$$ ossia $$a=-\frac{k}{m}x.$$ Se poniamo $$\alpha=-\frac{k}{m},$$ otteniamo l'equazione del moto armonico $$a=\alpha x.$$ Pertanto il moto di una massa attaccata a una molla è armonico.
\end{example}
Vediamo ora quale è la legge oraria del moto armonico. Una legge oraria è una formula che ci fornisce la posizione del corpo $x$ come funzione del tempo $t$. La legge oraria è $$x(t)=A\sin(\omega t),$$ dove $A$ e $\omega$ sono opportune costanti che dipendono dal problema. $A$ ha le dimensioni di una lunghezza, mentre $\omega$ di un $\textup{tempo}^{-1}$. Si definisce inoltre periodo la grandezza $$T=\frac{2\pi}{\omega},$$ e la frequenza $$f=\frac{1}{T}.$$ Il periodo ha le dimensioni di un tempo. Andiamo a rappresentare la legge oraria su un grafico. 
%\begin{figure}[H]
 %   \centering
  %  \includegraphics[width=\textwidth]{leggeorariaarmonico.png}
%\end{figure}
La grandezza $A$ (anche nota come ampiezza) rappresenta l'altezza dei picchi rispetto all'asse delle ascisse. Il periodo invece rappresenta la distanza orizzontale tra due picchi successivi.\\ La frequenza consiste nel numero di oscillazioni che effettua il corpo in un secondo. Il periodo è il tempo impiegato ad effettuare un'oscillazione. L'ampiezza è lo spostamento massimo raggiunto dal corpo durante il suo moto. \\Il moto risulta essere spazialmente limitato, con la posizione massima raggiunta che coincide appunto con l'ampiezza.  
\end{document}
